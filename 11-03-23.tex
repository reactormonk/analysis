\subsection{Höhere (partielle) Ableitungen}
Sei
\[f:\Omega_{\subset\mb{R}}\to\mb{R}\]
Die partiellen Ableitungen von $f$:
\[\Part{f}{x_i}(x)=\Limo{\varepsilon}\frac{f(x+\varepsilon e_i)-f(x)}{\varepsilon}\]
\[e_i=(0,\cdots,i,\cdots,0)\]
\[\Part{f}{x_i}:\Omega\to\mb{R}\]
\[\Part{\left( \Part{f}{x_i} \right)}{x_j}(x)\left(=\frac{\partial^2f}{\partial x_j\partial x_i} \right)(x))\]
\[=\lim_{\varepsilon\downarrow 0}\frac{\Part{f}{x_i}(x+\varepsilon e_j)-\Part{f}{x_i}(x)}{\varepsilon}\]
\[\frac{\partial^3f}{\partial x_k\partial x_j \partial x_i}(x)\]
\[=\lim_{\varepsilon\downarrow 0}\frac{\frac{\partial^2f}{\partial x_i}(x+\varepsilon e_j)-\frac{\partial ^2f}{\partial x_j \partial x_i}(x)}{\varepsilon}\]
\[\left( \frac{\partial f}{\partial x_i \partial x_i} \right)=\frac{\partial^2 f}{\partial x_i^2}\]
\[\frac{\partial^3 f}{\partial x_i \partial x_i \partial x_i}=\frac{\partial^3 f}{\partial x_i^3}\]
\begin{Sat}
  (von Schwarz) Sei $f:\Omega\to\mb{R}$ eine Funktion die in einer Umgebung von $p\in\Omega$ die partielle Ableitungen $\Part{f}{x_i}$, $\Part{f}{x_j}$ und $\frac{\partial^2}{\partial x_i \partial x_j}$ besitzt. Falls $\frac{\partial^2 f}{\partial x_i \partial x_j}$ stetig in $p$ ist, dann existiert $\frac{\partial^2 f}{\partial x_j\partial x_i}(p)$ und
  \[\frac{\partial^2 f}{\partial x_i\partial x_j}(p)=\frac{\partial^2 f}{\partial x_j\partial x_i}(p)\]
\end{Sat}
\begin{Bsp}
  \[f(x_1,x_2)=\sum_{i=1}^{N_1}\sum_{j=1}^{N_2}a_{ij}x_1^ix_2^j\]
  \[\Part{f}{x_1}=\sum_{i=1}^{N_1}\sum_{j=1}^{N_2}ia_{ij}x_1^{i-1}x_2^j\]
  \[\frac{\partial^2 f}{\partial x_2\partial x_1}=\sum_{i=1}^{N_1}\sum_{j=1}^{N_2}ija_{ij}x_1^{i-1}x_2^{j-1}\]
  \[\Part{f}{x_2}=\sum_{i=1}^{N_1}\sum_{j=1}^{N_2}ja_{ij}x_1^ix_2^{j-1}\]
  \[\frac{\partial^2 f}{\partial x_1\partial x_2}=\sum_{i=1}^{N_1}\sum_{j=1}^{N_2}ija_{ij}x_1^{i-1}x_2^{j-1}\]
\end{Bsp}
\begin{Bsp}
  Sei $V:\mb{R}\to\mb{R}$.
  \[v:\mb{R}^2\to\mb{R}\]
  \[v(x_1,x_2)=V(x_2)\]
  \[\Part{f}{x_1}=0\]
  \[\frac{\partial^2f}{\partial x_2\partial x_+}=0\]
\end{Bsp}
\begin{Bew}
  Die Idee ist den Mittelwertsatz zu benutzen.
  \paragraph{Schritt 1} Von Dimension $n\to 2$
  \[f(x_1,\cdots,x_i,\cdots,x_j,\cdots,x_n)\]
  \[p=(p_1,\cdots,p_i,\cdots,p_j,\cdots,p_n)\]
  \[g:U_{\subset\mb{R}}\to\mb{R}\]
  \[g(y,z)=g(p_1,\cdots,p_{i-1},y,p_{i+1},\cdots,p_{j-1},z,p_{j+1},\cdots,p_n\]
  \[\Part{f}{x_i}(p)=\Part{g}{y}(p_i,p_j)\]
  \[\Part{f}{x_j}(p)=\Part{g}{y}(p_i,p_j)\]
  \[\frac{\partial f}{\partial x_j\partial x_i}=\frac{\partial^2 g}{\partial z\partial y}(p_i,p_j)\]
  \[\frac{\partial f}{\partial x_i\partial x_j}(p)=\lim_{\varepsilon\downarrow 0}\frac{\Part{f}{x_j}(p_1,\ldots,p_i+\varepsilon,\ldots,p_j,\ldots,p_n)-\Part{f}{x_j}(p)}{\varepsilon}\]
  \[\Part{g}{z}(p)=\lim_{\varepsilon\downarrow 0}\frac{\Part{f}{x_j}(p_i+\varepsilon p_j)-\Part{g}{z}(p)}{\varepsilon}\]
  \[=\frac{\partial g}{\partial y \partial z}(p_i,p_j)\]
  Falls
  \[\frac{\partial g}{\partial y \partial z}(p_i,p_j)\]
  existiert und
  \[\frac{\partial g}{\partial z \partial y}(p_i,p_j)\]
  gleicht, dann ist das Theorem bewiesen.
  \paragraph{Deswegen}
  Nun, 
  \[f:\Omega_{\subset\mb{R}^2}\to\mb{R}\]
  $\Part{f}{x_1}$, $\Part{f}{x_2}$ und $\frac{\partial^2 f}{\partial x_2\partial x_1}$ existieren in einer Umgebunv von $p=(a,b)$, dann ist $\frac{\partial^2 f}{\partial x_2\partial x_1}$ stetig auf $p$. Zu beweisen: $\frac{\partial^2 f}{\partial x_2\partial x_1}(p)$ existiert und
  \[\frac{\partial^2 f}{\partial x_2\partial x_1}(p)=\frac{\partial^2 f}{\partial x_2\partial x_1}(p)\]
  $p=(a,b)$, $h,k\in\mb{R}\setminus\left\{ 0 \right\}$, $Q=$ Rechteck mit Ecken $(a,b)$, $(a+h, b)$, $(a, b+k)$, $(a+h, b+k$.
  \[D_Qf=f(a+h,b+k)-f(a+h, b)-f(a,b+k)+f(a,b)\]
  \[\Limo{k} \Limo{h}\frac{D_Qf}{hk}\]
  \[=\Limo{k} \Limo{h}\frac{f(a+h,b+k)-f(a,b+k)}{hk}-\frac{f(a+h,b)-f(a,b)}{hk}\]
  \[=\Limo{k}\frac{\Part{f}{x_1}(a,b+k)-\Part{f}{x_1}(a,b)}{k}\]
  \[=\frac{\partial^2f}{\partial x_2\partial x_1}(a,b)\]
  \[\Limo{h}\left( \Limo{k}\frac{D_Qf}{hk} \right)\]
  \[=\Limo{h}\frac{\Part{f}{x_2}(a+h,b)-\Part{f}{x_2}(a,b)}{h}=?\]
  wenn der Limes existiert
  \[=\frac{\partial^2f}{\partial x_1\partial x_2}(a,b)\]
  \paragraph{Ziel}
  $\Limo{h}\Limo{k}\frac{D_Qf}{hk}$ existiert und gleicht $\Limo{k}\Limo{h}\frac{D_Qf}{hk}$. $\implies$ Satz von Schwarz
  \paragraph{Zuerst}
  Wir behaupten ($\forall h,k$ klein genug) die Existenz von einer Stelle $(\xi,\zeta)\in Q$ so dass
  \begin{equation}
    \label{e:1103231}
    \frac{D_Qf}{hk}=\frac{\partial^2f}{\partial x_2\partial x_1}(\xi, \zeta)
  \end{equation}
  \[\frac{D_Qf}{hk}\neq \frac{1}{h}\left\{ \frac{f(a+h,b+k)-f(a+h,b)}{k}-\frac{f(a,b+k)-f(a,b)}{k} \right\}\]
  \[=\frac{1}{h}\left\{ g(a+h)-g(a) \right\}\stackrel{\text{Mittelwertsatz}}{=}g'(\xi)\]
  $x\in ]a,a+h[$, $\zeta\in]b,b+k[$ OBdA: $h,k>0$
  \[g(z)=\frac{f(z,b+k)-f(z,b)}{k}\]
  \[g'(z)=\left( \Part{f}{x_1}(z,b+k)-\Part{f}{x_1}(z,b) \right)\frac{1}{k}\]
  \ldots
  \[=\frac{1}{k}\left( \Part{f}{x_1}(\xi,b+k)-\Part{f}{x_1}(\xi,b) \right)\]
  \[=\Part{f}{x_2}\left( \Part{f}{x_1} \right)(\xi,\zeta)\]
  Womit wir beim zweiten Teil von \ref{e:1103231} wären.
  \[\frac{D_Qf}{hk}-\frac{\partial^2f}{\partial x_2\partial x_1}(a,b)=\frac{\partial^2f}{\partial x_2\partial x_1}(\xi,\zeta)-\frac{\partial^2f}{\partial x_2\partial x_1}(a,b)\]
  \[\Limo{h}\Limo{k}\left( \frac{D_Qf'}{hk}-\frac{\partial^2f}{\partial x_2\partial x_1}(a,b) \right)\]
  \[=\Limo{h}\Limo{k}\left( \frac{\partial^2f}{\partial x_2\partial x_1}(\xi,\zeta)-\frac{\partial^2 f}{\partial x_2\partial x_1}(a,b) \right)\]
  $\forall \varepsilon$ $\exists \delta$ so dass wenn $\sqrt{h^2+k^2}<\delta$
  \begin{equation}
    \label{e:1103233}
    \implies \Abs{\frac{\partial^2f}{\partial x_2\partial x_1}(\xi,\zeta)-\frac{\partial^2f}{\partial x_2\partial x_1}(a,b)}<\varepsilon
  \end{equation}
  \[\limsup_{h\to 0}\Abs{\Limo{k}\frac{D_Qf}{hk}-\frac{\partial^2 f}{\partial x_2\partial x_1}(a,b)}\]
  \[\leq \sup_{h\in ]0,\frac{\delta}{2}[}\Abs{\Limo{k}\frac{D_Qf}{hk}-\frac{\partial f}{\partial x_2 \partial x_1}(a,b)}\]
  \[\leq \sup_{h\in ]0,\frac{\delta}{2}[}\sup_{k\in ]0,\frac{\delta}{2}[}\Abs{\frac{D_Qf}{hk}-\frac{\partial f}{\partial x_2 \partial x_1}(a,b)}\stackrel{\ref{e:1103233}}{\leq} \varepsilon\]
  \[\implies \limsup_{k\to 0}\cdots=\Limo{k}\cdots=0\]
  \[\implies\Limo{h}\Limo{k}\frac{D_+f}{hk}\]
  \begin{equation}
    \label{e:110323r}
    =\frac{\partial^2f}{\partial x_2\partial x_1}(a,b)
  \end{equation}
  \[\left( =\Limo{k}\Limo{h}\frac{D_Qf}{hk} \right)\]
  \[\Limo{h}\Limo{k}\frac{1}{hk}\left\{ f(a+h,b+k)-f(a+h,b)-f(a,b+h)+f(a,b) \right\}\]
  \[=\Limo{h}\\frac{1}{h}\left\{ \Limo{k} \frac{f(a+h,b+k)-f(a+h,b)}{k}-\frac{f(a,b+h)+f(a,b)}{k} \right\}\]
  \[\Limo{h}\frac{1}{h}\left\{ \Part{f}{x_2}(a+h,b)-\Part{f}{x_2}(a,b) \right\}\]
  \[=\Part{}{x_1}\left( \Part{f}{x_2} \right)(a,b)\]
  \begin{equation}
    \label{e:110323b}
    =\frac{\partial^2f}{\partial x_1\partial x_2}(a,b)
  \end{equation}
  \ref{e:110323r} = \ref{e:110323b}
\end{Bew}

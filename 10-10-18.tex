\subsection{Uneigentliche Konvergenz}
\begin{Def}
  Sei $a_n$ eine Folge von reellen Zahlen. Dann sagen wir:
  \begin{itemize}
    \item $a_n\to +\infty$ (oder $\lim_{n\to +\infty} a_n=+\infty$) falls $\forall M\in\mb{R}$ $\exists N\in\mb{R}:a_n\geq M$ $\forall n\geq N$ 
(d.h. $a_n\geq M$ für fast alle $n\in\mb{R}$)
    \item $a_n\to-\infty$ ($\lim_{n\to -\infty} a_n=-\infty$) 
falls $\forall M\in\mb{R}$, $a_n\leq M$ für fast alle $n$.
  \end{itemize}
\end{Def}
\begin{Ueb}
$\limsup_{n\to+\infty}a_n=+\infty$ (bzw. $\liminf_n a_n = -\infty$)
$\iff$ $\exists$ Teilfolge 
$\left\{ a_{n_k} \right\}_{k\in\mb{N}}$ mit 
$a_{n_k}\stackrel{k\to+\infty}{\to}+\infty$ (bzw. 
$a_{n_k}\stackrel{k\to+\infty}{\to}-\infty$).
\end{Ueb}
\begin{Bem}
  Sei $a_n$ eine wachsende (bzw. fallende) Folge. Dann:
  \begin{itemize}
    \item entweder konvergiert $a_n$
    \item oder $\lim_{n\to+\infty}a_n=+\infty$ (bzw. $\lim_{n\to+\infty}a_n=-\infty$)
  \end{itemize}
\end{Bem}
\section{Reihen}
\subsection{Konvergenz der Reihen}
\begin{Def}
  Sei $(a_n)_{n\in\mb{N}}$ eine Folge komplexer Zahlen. Wir setzen:
  \begin{align*}
    s_0=a_0\\
    s_1=a_0+a_1\\
    s_2=a_0+a_1+a_2\\
    \cdots
  \end{align*}
  \begin{equation*}
    s_k:=\sum^k_{i=0}a_i
  \end{equation*}
\end{Def}
\begin{Def}
  Die $(s_k)_{k\in\mb{N}}$ ist die Folge der Partialsummen. 
Die Reihe ist die Folge $(s_k)_{k\in\mb{N}}$. Falls der Limes 
von $s_k$ existiert, dann $\lim_{n\to+\infty}s_n$ ist der \ul{Wert der Reihe}. 
Und wir sagen dass $(s_k)$ eine \ul{konvergente Reihe} ist.
\end{Def}
\begin{Not}
$\sum^\infty_{i=0}a_i$ bezeichnet \ul{die Reihe} und \ul{den Wert der Reihe}
(wenn sie konvergiert).
 Wenn die Partialsumme eine Folge reeller Zahlen ist und 
$s_n\to+\infty$ (bzw. $-\infty$), dann wir schreiben
$\sum a_n=+\infty$ (bzw. $-\infty$).
\end{Not}
\begin{Bsp}
Sei $z$ eine komplexe Zahl. Dann ist die Reihe 
$\sum^\infty_{n=0}z^n$ die \ul{geometrische Reihe} (f\"ur $z=0$ gilt 
die Konvention dass $z^0=0$).

\medskip

{\bf F\"ur $\abs{z}<1$ die geometrische Reihe konvergiert.}
 
In der Tat:
  \begin{align*}
    &(1-z)(1+z+\cdots z^n)=1-z^{n+1}\\
    &s_n=\frac{1-z^{n+1}}{1-z}\\
    &\lim_{n\to+\infty}\frac{1-z^n}{1-z}=\lim_{n\to+\infty}\left( 
\frac{1}{1-z} \right)-\frac{1}{1-z}\underbrace{\left( \lim_{n\to+\infty}z^n 
\right)}_{=0 \text{ weil } \abs{z}<1}=\frac{1}{1-z}
  \end{align*}

\medskip

{\bf  Für $\abs{z}\geq 1$ die geometrische Reihe divergiert}.
In der Tat:
\begin{itemize}
\item Falls $z=1$, dann $s_n=1+1+\cdots+1=n+1$;
\item Falls $z\neq 1$ gilt die Formel
\[s_n=\frac{(1-z^{n+1})}{1-z}\] 
und $s_n$ konvergier genau dann, wenn $z^n$ konvergiert.
Aber $z^n$ konvergiert nicht, weil:
\begin{itemize}
\item F\"ur $z>1$ wir haben $|z|^n\to \infty$;
\item F\"ur $|z|=1 (z\neq 1)$ wir haben
\[z=\cos\theta+i\sin\theta\implies z^n=\cos(n\theta)+i\sin(n\theta)
\qquad \mbox{(siehe \"Ubung 4 Blatt 3)}\]
mit $\theta\in ]0, 2\pi[$
und es ist einfach zu sehen dass $z^{n+1}$ nicht konvergiert.
\end{itemize}
\end{itemize}

\end{Bsp}

\begin{Bsp}
Die bekannte armonische Reihe: $\sum^\infty_{n=1}\frac{1}{n}$.
In diesem Fall $s_{n+1}\geq s_n$: $s_n$ ist eine wachsende Folge
So, entweder $s_n$ konvergiert nach eine
reele Zahl oder $\lim_n s_n =+\infty$. Wir betrachten die
Teilfoge $s_{2^n-1}$:
\begin{eqnarray*}
s_{2^n-1}&=&1+\underbrace{\frac{1}{2}+\frac{1}{3}}+
\underbrace{\cdots}_{2^{k-1}\leq j\leq 2^k-1}+\cdots+
\underbrace{\cdots}_{2^{n-1}\leq j\leq 2^n-1}\\
&\geq &1+\frac{1}{4}+\cdots+\underbrace{\frac{1}{2^k}+\cdots+\frac{1}{2^k}}_{2^{k-1}}
+\cdots\\
&\geq &1+\frac{1}{2}+\frac{1}{2} = 1+\frac{n-1}{2}\\
\end{eqnarray*}
Deswegen $\lim_n s_{2^n-1}=+\infty$ und 
die ursprüngliche Folge $(s_n)$ konvergiert nicht! 
\[\implies \lim_{n\to+\infty}s_n=+\infty \qquad \mbox{d.h.}
\quad \sum\frac{1}{n}=+\infty\]
\end{Bsp}
\subsection{Konvergenzkriterien für reelle Reihen}
\begin{Bem} (gilt auch für komplexe Reihen!)
  \[\sum^\infty_{n=0}a_n \text{konvergiert}\implies a_n\to 0\]
\end{Bem}
\begin{proof}[Beweis] Da $a_n = s_{n+1}- s_n$
und $\lim_n s_n = \lim_n s_{n+1} = \sum a_n$, wir schliessen
$a_n\to \sum a_n - \sum a_n = 0$.
\end{proof}
\begin{Ueb} Beweise
ganz schnell dass die geometrische Reihe nicht konvergiert wenn $\abs{z}\geq 1$.
\end{Ueb}
\begin{Bem}
  $a_n\to 0$ impliziert {\bf NICHT}
dass $\sum^\infty_{n=0}a_n$ konvergiert! Bsp: $a_n=\frac{1}{n}$
\end{Bem}

\begin{Sat}
  Sei $\sum a_n$ eine Reihe mit reellen Zahlen $a_n\geq 0$. Dann:
  \begin{itemize}
    \item entweder ist die Folge $(s_n)$ beschränkt und die Reihe konvergiert
    \item oder $\sum^\infty_{n=0}s_n=+\infty$
  \end{itemize}
\end{Sat}

\begin{proof}[Beweis] Trivial: $(s_n)$ ist eine wachsende Folge.
\end{proof}

\begin{Sat}[Konvergenzkriterium von Leibnitz] Sei $(a_n)$ eine fallende Nullfolge. Dann konvergiert $\sum^\infty_{n=0}(-1)^na_n$ (eine alternierende Reihe).
\end{Sat}
\begin{proof}[Beweis]
  Betrachten wir 
  \[s_k-s_{k-2}=(-1)^{k-1}a_{k-1}+(-1)^ka_k = (-1)^k\overbrace{(a_k-a_{k-1})}^{\leq 0}\]
  \begin{itemize}
    \item $s_k-s_{k-2}\geq 0$ falls $k$ ungerade ist
    \item $s_k-s_{k-2}\leq 0$ falls $k$ gerade ist
  \end{itemize}
  Für $k$ ungerade:
  \[s_1\leq s_3\leq s_5\leq \cdots\]
  \[\underbrace{s_k}_{\text{gerade}}=\underbrace{s_{k+1}}_{\text{ungerade}}+
\underbrace{(-1)^{k+1}}_{\geq 0}
\underbrace{a_{n+1}}_{\geq 0}\leq s_{k+1}\]
  Für $k$ gerade:
  \[s_0\geq s_2\geq s_4\leq \cdots\]
(Beweis gleich wie für ungerade)\\
Deswegen: 
\begin{itemize}
\item $(s_{2k+1})$ ist eine wachsende Folge mit $s_{2k+1}\leq s_0$,
$\implies$ $(s_{2k+1})$ ist eine beschr\"ankte wachsende Folge.
\item $(s_{2k})$ ist eine fallende Folge mit $s_{2k}\geq s_1$,
$\implies$ $(s_{2k})$ ist eine monotone fallende Folge.
\end{itemize}
Deswegen die beid Folgen konvergieren.
Seien
\[\lim_{k\to+\infty} s_{2k}=S_g\qquad
 \lim_{k\to\infty} s_{2k+1} = S_u\, .
\]
Dann
\[S_u-S_g=\lim_{n\to+\infty}s_{2n+1}-\lim_{n\to+\infty}s_{2n}=
\lim_{n\to+\infty}(s_{2n+1}-s_{2n})\lim_{n\to+\infty}a_{2n+1}=0\]
\[\implies S_u=S_g\implies \lim_{n\to+\infty}s_n=S_u(=S_g)\]
\end{proof}
\begin{Kor}
  \[1-\frac{1}{2}+\frac{1}{3}-\frac{1}{4}+\cdots\]
  konvergiert
\end{Kor}
\subsection{Konvergenzkriterien für allgemeine (komplexe) Reihen}
\begin{Bem}[Cauchyches Kriterium]
  $\sum a_n$ konvergiert $\iff$ $(s_n)$ konvergiert $\iff$ $(s_n)$ 
ist eine Cauchyfolge. $\iff$ 
\begin{equation}\label{e:CauchyR}
\forall\varepsilon>0\;
\exists N: \qquad 
|a_n+ a_{n-1} + \ldots + a_{m+1}|= \abs{s_n-s_m}<\varepsilon
\quad \forall n\geq m\geq N\, .
\end{equation}
\end{Bem}
\begin{Kor}[Majorantenkriterium] Sei $\sum a_n$ eine 
Reihe komplexer Zahlen und $\sum b_n$ 
eine konvergente Reihe nichtnegativer reeller Zahlen. 
Falls $\abs{a_n}\leq b_n$ (d.h. $\sum b_n$ {\bf majorisiert} $\sum a_n$), 
dann ist $\sum a_n$ konvergent.
\end{Kor}
\begin{proof}[Beweis] Seien $s_n$ die Partialsummen von $\sum_{a_n}$
und $\sigma_n$ die Partialsummen von $b_n$. Da $\sum b_n$
konvergiert, gilt \eqref{e:CauchyR}:
\[\forall \varepsilon>0\; \exists N:\qquad
b_n+\ldots+b_{m+1} = \abs{\sigma_n-\sigma_m}<\varepsilon\quad\forall n\geq m\geq N\]
Deswegen, f\"ur $n\geq m\geq N$:
\[|s_n-s_m| \leq \abs{a_n}+\ldots+\abs{a_{m+1}}\leq
b_n + \ldots b_{m+1} \leq \varepsilon. \]
Aus dem Cauchychen Kriterium folgt dass $\sum a_n$ konvergiert.
\end{proof}

\begin{Lem}
  Für Treppenfunktionen $\phi, \psi$ und Zahlen $\alpha, \beta\in\mb{R}$ gilt:
  \begin{enumerate}
    \item $\alpha \phi +\beta\psi$ eine Treppenfunktion ist und
      \[\int^b_a(\alpha\phi + \beta\psi)\md x=\alpha\int_a^b\psi\md x+\beta\int^b_a\psi\md x\]
    \item $|\psi|$ eine Treppenfunktion ist und
      \[\left|\int_a^b\psi\md x\right|\leq \int_a^b\abs{\psi}\md x\leq (b-a)\max_{x\in [a,b]} \phi (x) \]
    \item Falls $\phi\leq\psi$ (d.h. $\phi(x)\leq \psi(x)\forall x\in [a,b]$), dann
      \[\int_a^b\phi\md x\leq\int_a^b\psi\md x\]
  \end{enumerate}
\end{Lem}
\begin{Bew} {\bf 1}
$\exists a = x_0<x_1<\cdots<x_n=b$ so dass $\phi|_{]x_k, x_{k+1}}\equiv \mbox{konst}$ 
und $\exists a = y_0<y_1<\cdots<y_n=b$ so dass $\phi|_{]y_k, y_{k+1}}\equiv \mbox{konst}$
Seien $a= z_0< z_1 < \ldots < z_N= b$ so dass
\[\left\{ x_0,\cdots,x_n,y_0,\cdots y_m \right\}=\left\{ z_0<z_1<\cdots<z_N \right\}\]
Dann: $\forall k\in \{1, \ldots, N\}$
\[\phi|_{]z_{k-1},z_k[}\equiv c_k\in\mb{R}\s\]
\[\psi|_{]z_{k-1},z_k[}\equiv d_k\in\mb{R}\s\]
$F := \alpha\phi+\beta\psi$ ist konstant in jedem $]z_{k-1}, z_k[$ und
das beweist dass $F$ eine Treppenfunktion ist. Ausserdem,
\[F|_{]z_{k-1},z_n[}=\alpha c_k + \beta d_k\, ,\]
\[\int_a^b\phi=\sum_{k=1}^N(z_k-z_{k-1})c_k\, ,\]
\[\int_a^b\psi=\sum_{k=1}^N(z_k-z_{k-1})d_k\, \]
und
\[\int_a^bF=\sum^N_{k=1}(z_k-z_{k-1})(\alpha c_k+\beta d_k)
=\alpha\sum_{k=1}^N(z_k-z_{k-1})c_k+\beta\sum_{k=1}^N(z_k-z_{k-1})c_k\]
\[=\alpha\int_a^b\phi+\beta\int_a^b\psi\]

\medskip

{\bf 2} Seien $a=x_0<x_1<\cdots<x_n=b$ mit
\[\phi|_{]x_{k-1},x_k}=c_k\in\mb{R}\]
Dann
\[\abs{\phi}|_{]x_{k-1},x_k}=\abs{c_k}\in\mb{R}\, .\]
$|\phi$ ist eine Treppenfunktion und
\[
\abs{\int_a^b\phi}=\left|\sum^n_{k=1}(x_k-x_{k-1})c_n\right|
\leq \sum^n_{k=1}(x_k-x_{k-1})\abs{c_k}= \int_a^b |\phi|\, .
\]

\medskip
{\bf 3} ist eine einfache Folgerung der gleichen Ideen.
\end{Bew}
\subsection{Regelfunktion}
\begin{Def}\label{d:1012011}
  Eine Abbildung $f:[a,b]\to\mb{R}$ heisst Regelfunktion falls 
$\exists f_n:[a,b]\to\mb{R}$ (Folge von Funktionen), so dass:
  \begin{itemize}
    \item Jede $f_n$ eine Treppenfunktion ist
    \item
      \[\Limi{k}\underbrace{\left( \sup_{x\in[a,b]}\abs{f_k (x)-f (x)} \right)}_{:=\Norm{f_k-f}}=0\]
  \end{itemize}
\end{Def}
\begin{Sat}\label{s:unab}
  Sei $f$ eine Regelfunktion. Seien $\left\{ f_n \right\}$ und $\left\{ g_n \right\}$ zwei Folgen von Treppenfunktionen, welche die zwei Bedingungen in efinition \ref{d:1012011} erfüllen. Dann:
  \[\Limi{k}\int_a^bf_k=\Limi{k}\int_a^bg_k\s(\in\mb{R})\]
\end{Sat}
\begin{Def}
  Sei $f$ eine Regelfunktion und $\left\{ f_k \right\}$ eine Folge welche die 
zwei Bedinungen in Definition \ref{d:1012011} erf\"ullen. Dann existieren die Grenzwerte
\[\int_a^b f(x)\md x \qquad \mbox{und} \qquad \Limi{k}\int_a^bf_k(x)\md x\, .\]
existieren. Ausserdem, sie sind gleich und geh\"oren zu $\mb{R}$.
\end{Def}

Der Satz \ref{s:unab} garantiert dass $\int_a^b f$ wohldefiniert ist!

\begin{Bem} Es ist leicht zu sehen dass
  \[ \begin{cases}
    \int_a^bf\geq 0&  \mbox{falls }f\geq 0\\
    \int_a^bf = -\int_a^b (-f) \leq 0& \mbox{falls } f\leq 0
  \end{cases}\]
Deswegen, wenn $f\leq 0$, der Inhalt von $G:=\{(x,y): x\in \in I \mbox{ und } f(x) \leq y \leq 0\}$
ist $- \int_a^b f$.
\end{Bem}
\begin{proof}[Beweis vom Satz \ref{s:unab}] Zuerst bemerken wir dass, $\forall k, i$,
\[\left|\int_a^bf_k-\int_a^bf_i\right|=\left|\int_a^b(f_k-f_i)\right|\leq\]
\[(b-a) \sup_{x\in[a,b]}\abs{f_k-f_i}(x)\leq (b-a) 
\sup_{x\in[a,b]}\left\{ \abs{f_k-f}(x)+\abs{f-f_i}(x) \right\}\]
  \[\leq (b-a) \left(\sup_{x\in[a,b]}\abs{f_k-f}(x)+\sup_{x\in[a,b]}\abs{f-f_i}(x)\right)\]
  \[=(b-a) \left(\Norm{f_k-f}+\Norm{f-f_i}\right) \]
Sein nun $\eps>0$. Dann, $\exists N$ so dass $\|f-f_j\| < \eps/(2(b-a))$ $\forall i\geq N$.
Dann, wenn $k,i\geq N$,
\[
\left|\int_a^bf_k-\int_a^bf_i\right| < \eps\, .
\]
$\implies (a_k)=\left( \int_a^bf_k \right)$ ist eine Cauchyfolge 
$\implies$ $\exists \Limi{k}\int_a^bf_k\in\mb{R}$

\medskip

(Wir bemerken hier eine wichtige Eigenschaft der Norm $\|\cdot\|$:
\begin{equation}\label{e:3eck}
\Norm{f+g}\leq \Norm{f}+\Norm{g}
\end{equation}
In der Tat 
\[\abs{\sup_xf(x)+g(x)}\leq \sup_x\abs{f(x)}+\abs{g(x)}\leq\sup_x\abs{f(x)}+\sup_x\abs{g(x)}\, .\]
Die \eqref{e:3eck} ist \"ahnlich zur Dreiecksungleichung $|a+b|\leq |a|+|b|$.)

\medskip

Ausserdem,
\[\left|\Limi{k}\int_a^bf_k-\Limi{k}\int_a^bg_k\right|
=\left|\Limi{k}\left( \int_a^bf_k-\int_a^bg_k \right)\right|
=\Limi{k}\left|\int_a^b(f_k-g_k)\right|\]
\[\leq \int_a^b |f_k-g_k| \leq\Limi{k} (b-a) \Norm{f_k-g_k}
\leq (b-a) \left(\Limi{k}\left\{ \underbrace{\Norm{f_k-f}}_{\to 0}+\underbrace{\Norm{g_k-f}}_{\to 0} \right\}\right)=0\]
  \[\implies \Limi{k}\int_a^bf_k=\Limi{k}\int_a^bg_k\, .\]
\end{proof}
\begin{Sat}
  Eine stetige Funktion $f:[a,b]\to\mb{R}$ ist eine Regelfunktion.
\end{Sat}
\begin{Bew}
  Sei $k\in\mb{N}\setminus\left\{ 0 \right\}$, $f$ stetig, $[a,b]$ kompakt. $f$ ist gleichmässig stetig.  
Wir setzten $\varepsilon=\frac{1}{k}$ in der Definition der gleichmässigen Stetigkeit.
  \[\implies\exists\delta>0\s\abs{x-y}<\delta\implies\abs{f(x)-f(y)}<\frac{1}{k}\]
  Seien
\[ x_0 := a\,, \quad x_1 := a+\delta, \quad \ldots \quad 
x_N := a+N\delta, \quad x_{N+1} = b\]
wobei $N=\max\left\{ k,a+k\delta<b \right\}$. 

Sei $y_j=\frac{x_{j-1}+x_j}{2}$ (der Mittelpunkt von $I=[x_{j-1},x_j]$). Wir definieren
  \[ f_k(x)= \begin{cases}
    f_k(x)=f(y_j)& x\in [x_{j-1},x_j[\\
    f_k(x)=f(y_{N+1})&x=b
  \end{cases}\]
Wir bemerken dass
  \[\Norm{f-f_k}=\sup_{x\in I}\abs{f_k(x)-f(x)}<\frac{1}{k}\]
In der Tat, falls $x\in I$, dann 
$x\in [x_{j-1},x_j[$ oder $x\in [x_N,x_{N+1}]$. Deswegen, 
$\abs{x-y_j}\leq \frac{\delta}{2}$ oder $\abs{x-y_{N+1}}\leq \frac{\delta}{2}$.
\[\implies \abs{f(x)-f_k(x)}=\abs{f(x)-f(y_j)}<\frac{1}{k}\]
  \[\text{oder}\s\abs{f(x)-f_k(x)}=\abs{f(x)-f(y_{N+1})}<\frac{1}{k}\]
  $\forall k$ ist $f_k$ eine Treppenfunktion und $\Norm{f_k-f}\to 0$ für $k\to+\infty$
\end{Bew}
\begin{Bem} Da $f_k$ eine Treppefunktion ist,
\[\int_a^bf_k=\sum_{j=1}^{N+1}(x_j-x_{j-1})f(y_j)\, .\]
Die Summe 
\begin{equation}\label{e:Riem}
\sum_{j=1}^{N+1}(x_j-x_{j-1})f(y_j) 
\end{equation}
konvergiert gegen $\int_a^b f$ wenn $N\to\infty$.

Es ist nicht n\"otig dass $y_j$ der Mittelpunkt des Intervalls $[x_{j-1}, x_j]$ 
ist. Die gleiche Konvergenz erreicht man f\"ur beliebige Stellen $y_j\in [x_{j-1}, x_j]$.
In diesem Fall heisst die Summe in \eqref{e:Riem} eine {\em Riemannsche Summe}. 
\end{Bem}
\begin{Kor}
  Eine ``stückweise stetige'' Funktion auf $[a,b]$ ist auch eine Regelfunktion. 
( Eine Funktion heisst st\"uckweis Stetig wenn $\exists a=x_0<x_1<\cdots<x_n=b$ s.d.
  \begin{itemize}
    \item $f$ ist stetig überall auf $]x_{j-1}, x_j[$
    \item $\forall j\in \left\{ 0,\cdots,n \right\}$
      \[\left.\lim_{x\downarrow x_j}f(x)\in\mb{R} \qquad \mbox{und} \qquad 
\lim_{x\uparrow x_j}f(x)\in\mb{R}\, .\right)\] 
  \end{itemize}
\end{Kor}
\begin{theorem}
  Seien $f,g:[a,b]\to\mb{R}$ Regelfunktionen und $\alpha,\beta\in\mb{R}$
  \begin{itemize}
    \item {\bf Linearität} $\alpha f + \beta g$ ist auch eine Regelfunktion und
      \[\int_a^b(\alpha + \beta g)=\alpha\int_a^bf+\beta\int_a^bg\]
    \item {\bf Dreiecksungleichung} 
      \[\left|\int_a^bf\right|\leq\abs{b-a}\Norm{f}\]
    \item {\bf Monotonie}
      \[\int_a^bf\leq\int_a^bg\s\text{falls}\s f\leq g\]
    \item $\forall a<c<b$:
\begin{equation}\label{e:c}
\int_a^bf=\int_a^cf+\int_c^bf
\end{equation}
\item {\bf Mittelwertsatz}
      Falls $f$ stetig ist, $\exists\xi ]a,b[$ so dass
      \[\int_a^bf=f(\xi)(b-a)\]
  \end{itemize}
\end{theorem}
\begin{Bew} {\bf Linearit\"at}. Seien $f_k, g_k$ Treppenfunktionen mit $\Norm{f_k} \to 0$, $\Norm{g-g_k}\to 0$. $\alpha f_k+\beta g_k$ ist auch eine Treppenfunktion und
      \[\Norm{(\alpha f+\beta g)-(\alpha f_k-\beta g_k)}\leq\abs{\alpha}\Norm{f-f_k}+\abs{\beta}\Norm{g-g_k}\to 0\]
      \[\int_a^b(\alpha f+\beta g)=\Limi{k}\int_a^b(\alpha f_k+\beta g_k)=\Limi{k}(\alpha \int_a^bf_k+\beta\int_a^b g_k)\]
      \[=\Limi{k}\alpha\int_a^b f_k+\Limi{k}\beta\int_a^b g_n=\alpha\Limi{k}\int_a^b f_k+\beta\Limi{k}\int_a^bg_k\]
      \[=\alpha\int_a^b f+\beta\int^a_b g\]

\medskip

{\bf Dreiecksungleichung} Sei $f_k$ wie oben. Dann $ - \|f-f_k\| \leq \|f\|-\|f_k\| \leq \|f-f_k\|$
(aus der Dreiecksungleichung f\"ur die Norm $\|\cdot\|$. Deswegen
\[
\left|\int_a^b f\right|= \lim_k \left|\int_a^b f_k\right|
\leq \lim_k (b-a) \|f_k\| = (b-a)\|f\|\, .
\]

\medskip

{\bf Monotonie} Seien $f_k$ und $g_k$ wie oben. Wir definieren $\tilde{f_k}:=f_k+\Norm{f-f_k}$ 
(deswegen $\tilde{f_k}\geq f$) und $\tilde{g_k}:=g_k+\Norm{g-g_k}$ (deswegen $g\geq\tilde{g_k}$). 
Dann, $\tilde{f_k}\geq f \geq g \geq \tilde{g_k}$ und
    \[\int_a^bf=\Limi{k}\int_a^b\tilde{f_k}\geq\Limi{k}\int_a^b\tilde{g_k}=\int_a^bg\]

\medskip

{\bf \eqref{e:c}} Sei $f_k$ wie oben. Die Identit\"at folgt aus der entsprechenden Identit\"aten
f\"ur die Funktionen $f_k$.

\medskip

{\bf Zwischenwertsatz} Die Monotonie impliziert:
\[(b-a)\min f\leq \int_a^b f\leq(b-a)\max f\]
Der Zwischenwertsatz f\"ur stetige Funktionen impliziert die Existenz einer Stelle $\xi\in ]a,b[$ mit 
\[
f(\xi)=\frac{\int_a^bf}{b-a}\, .
\]
\end{Bew}

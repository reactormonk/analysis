\subsection{Schrankensatz}

\begin{Def} sei $f:U\to\mb{R}^m$ eine Abbildung. Wir schreiben
  $f\in C^k(U,\mb{R}^m)$ falls die partielle Ableitungen jeder $f_i$ mit Ordnung $\leq k$ existieren und stetig sind ($f=(f_1,\cdots,f_m)$). 
\end{Def}

\begin{Sat}
  Sei $\Omega\subset\mb{R}^n$ eine offene Menge, $f\in C^1 (\Omega, \mb{R}^k)$
und $\gamma [a,b]\to\Omega$ eine $\mb{C}^1$ Kurve.
Dann:
  \[\Norm{f(\gamma(b))-f(\gamma(a))}\leq \left[ \sup_{t\in [a,b]}\Norm{\md f|_{\gamma(t)}}_O \right]\underbrace{\int_a^b\Norm{\dot \gamma(t)}\md t}_{\text{Länge der Kurve}}\]
\end{Sat}
Zur Erinnerung:  $\gamma:[a,b]\to\Omega\subset\mb{R}^n$, $\gamma=(\gamma_1,\cdots,\gamma_n)$, $\dot\gamma=(\gamma_1',\cdots,\gamma_n')$.
\begin{Bew}
  Sei $\phi:[a,b]\to\mb{R}^k$ die Funktion
  \[\phi(t):=f(\gamma(t))=f\circ\gamma\]
  Kettenregel
  \begin{equation}
    \label{e:1104131}
    \md \phi|_t=\md f|_{\gamma(t)}\md \gamma|_t
  \end{equation}
  \[\phi:[a,b]\to\mb{R}^k\]
  \[\md \phi|_t:\mb{R}\to\mb{R}^k\s\text{lineare Abbildung}\]
  $\phi=(\phi_1,\cdots,\phi_k)$
  \[ \begin{pmatrix}
    \Part{\phi_1}{t} \\ \vdots \\ \Part{\phi_k}{t} 
  \end{pmatrix} = \begin{pmatrix}
    \phi_1' \\ \vdots \\ \phi_k'
  \end{pmatrix} = \dot\phi \]
  Sei $A (x)$ die Jacobi-Matridx für $\md f|_x$ (d.h.  $A_{ij} (x) = \Part{f}{x_i} (x)$). 
Kettenregel:
  \[\underbrace{\dot\phi(t)=A(\gamma(t))\cdot \dot\gamma(t)}_{\text{Matrix-Darstellung von \eqref{e:1104131}}}\]
  \[f(\gamma(b))-f(\gamma(a))=\phi(b)-\phi(a)= \begin{pmatrix}
    \phi_1(b)-\phi_1(a) \\
    \vdots \\
    \phi_k(b)-\phi_k(a)
  \end{pmatrix}\]
  $\phi_i'$ ist eine stetige Funktion:
 \[
    \phi_i'(t)=\sum_{j=1}^nA_{ij}(\gamma(t))\gamma_j'(t) =
    \sum_{j=1}^n\Part{f_i}{x_j}(\gamma(t))\gamma_j'(t)
 \]
Nun
  \[\phi(b)-\phi(a)= \begin{pmatrix}
    \int_a^b\phi_1'(t)\md t\\
    \vdots \\
    \int_a^b\phi_k'(t)\md t
  \end{pmatrix} \]
und
 \[
    \Norm{f(\gamma(b))-f(\gamma(a))}= \Norm{\phi(b)-\phi(a)}^2
    =\sqrt{\sum^k_{i=1}\left( \int_a^b\phi_i'(t)\md t \right)^2}
\]
Wir brauchen nun die folgende``Dreiecksungleichun'':
\begin{equation}\label{e:int_drei}
\sqrt{\sum^k_{i=1}\left( \int_a^b\phi_i'(t)\md t \right)^2}
\leq \int_a^b \|\dot\phi (t)\| \md t\, .
\end{equation}
Diese Ungleichung folgt aus dem Lemma \ref{l:int_drei} unten.
Mit der schreiben wir
\begin{eqnarray*}
   \Norm{f(\gamma(b))-f(\gamma(a))}&\leq& \int_a^b \|\dot\phi (t)\| \md t
\;=\;  \int_a^b \| A (\gamma (t))\cdot \dot{\gamma} (t)\|\md t\\
&\leq& \int_a^b \|A (\gamma (t)\|_O \|\dot{\gamma} (t)\md t
\;=\ \int_a^b \|df|_{\gamma (t)}\|_O \|\dot{\gamma} (t)\md t\\
&\leq& \sup_{t\in [a,b]}\Norm{\md f|_{\gamma(t)}}_O
\end{eqnarray*}
\end{Bew}
\begin{Bem} In der Tat $\sup_{t\in [a,b]}\Norm{\md f|_{\gamma(t)}}_O$ ist
ein Maxim wegen der Stetigkeit der Abbildung $t\mapsto \Norm{\md f|_{\gamma(t)}}_O$.
\end{Bem}

 \begin{Lem}\label{l:int_drei}
  Sei $g:[a,b]\to\mb{R}^k$ eine stetige Funktion. Dann
  \[\sqrt{\sum_{i=1}^k\left( \int_a^bg_i \right)^2}\leq\int_a^b\Norm{g}\, .\]
  \ul{Dreiecksungleichung}
\end{Lem}
\begin{Bew}
  Sei $\varepsilon>0$ und Treppenfunktion $\alpha_i$ so dass $g_i-\varepsilon\leq\alpha_i\leq g_i+\varepsilon$, $\alpha_i-\varepsilon\leq g_i\leq \alpha_i+\varepsilon$. Dann
 \[\int_a^b\alpha_i-(b-a)\varepsilon
    \leq\int_a^b g_i \leq \int_a^b\alpha_i+(b-a)\varepsilon\, 
\]
d.h.
  \[\Abs{\int_a^bg_i-\int_a^b\alpha_i}\leq (b-a)\varepsilon\]
Deswegen
\begin{equation}\label{e:abs1}  
 \Abs{\sqrt{\sum_{i=1}^k(\int g_i)^2}-\sqrt{\sum_{i=1}^k \int\alpha_i^2}}
    \leq\sqrt{\sum_{i=1}^k \left(\int g_i-\int \alpha_i\right)^2}
    \leq \sqrt{k}(b-a)\varepsilon
  \end{equation}
Sei nun $\alpha = (\alpha_1, \ldots, \alpha_n)$. Dann
  \begin{equation}\label{e:abs2}
    \Abs{\int_a^b\Norm{g}-\int_a^b\Norm{\alpha}}
    \leq\int_a^b\Abs{\Norm{g}-\Norm{\alpha}}
    \leq\int_a^b\Norm{g-\alpha}
    \leq\int_a^b\sqrt{k}\varepsilon
    =\sqrt{k} (b-a)\varepsilon\, .
  \end{equation}
 Wir werden bewesein dass
    \begin{equation}
      \label{e:1104134}
      \sqrt{\sum\left(\int_a^b\alpha_i\right)^2}\leq\int_a^b\Norm{\alpha}
    \end{equation}
 \eqref{e:abs1}, \eqref{e:abs2} und \eqref{e:1104134} implizieren
  \begin{eqnarray*}
    \sqrt{\sum\left(\int_a^b g_i\right)^2}&\leq&
\sqrt{\sum\left(\int_a^b\alpha_i\right)^2}+(b-a)\sqrt{k}\varepsilon \\
    &\leq&\int_a^b\Norm{\alpha}+(b-a)\sqrt{k}\varepsilon
    \leq\int_a^b\Norm{g}+2(b-a)\sqrt{k}\varepsilon
  \end{eqnarray*}
  Wenn $\varepsilon\downarrow 0$:
  \[\sqrt{\sum\left( \int_a^bg_i \right)^2}\leq\int_a^b\Norm{g}\]

\medskip

{\bf Beweis von \eqref{e:1104134}.}  Ohne Beschränkung der Allgemeinheit: $\exists$ eine Zerteilung von $[a,b]$ 
  \[a=c_0<c_1<\cdots<c_N=b\]
so dass  jedes $\alpha_i$ ist konstant auf $[c_{j-1},c_j]=I_j$. Die Konstante ist $a_{i,j}$.
  \[\alpha = \begin{pmatrix}
    \alpha_1\\ \vdots \\ \alpha_k
  \end{pmatrix}\]
  ist konstant auf $I_j$ mit Wert
  \[a_j = \begin{pmatrix}
    a_{1,j}\\ \vdots \\ a_{k,j}
  \end{pmatrix}\]
  \[\sqrt{\sum_{i=1}^k\left( \int_a^b\alpha_i \right)^2}=\sqrt{\sum_{i=1}^k\left( \sum_{j=1}^N\abs{I_j}\alpha_{i,j} \right)^2}=\Norm{a}\]
wobei
\[
    a:=\sum_{j=1}^N\abs{I_j}a_j
    = \begin{pmatrix}
      \sum_{j=1}^N\abs{I_j}\alpha_{1,j}\\
      \vdots \\
      \sum_{j=1}^N\abs{I_j}\alpha_{1,j}\\
    \end{pmatrix}\, .
\]
Deswegen
\begin{eqnarray*}
\|a\|   &=& \Norm{\sum_{j=1}^N\abs{I_j}a_j}
\;\stackrel{\text{Dreiecksungleichung}}{\leq}\;\sum_{j=1}^N\Norm{\abs{I_j}a_j}\nonumber\\
& =&\sum_{j=1}^N\abs{I_j}\Norm{a_j}=\int_a^b\Norm{\alpha}
 \end{eqnarray*}
\end{Bew}

\begin{Kor}
  $f\in C^1(\Omega, \mb{R}^k)$ und $[p,q]\subset\Omega$. Dann:
  \[\Norm{f(p)-f(q)}\leq\max_{z\in [p,q]}\Norm{\md f|_z}_O \Norm{p-q}\]
\end{Kor}
\begin{Bew}
  Wenden den Schrankensatz an $f$ und $\gamma:[0,1]\to\Omega$ ist $\gamma(a)=(1-s)p+sq$.
Da $\dot \gamma=q-p$,
  \begin{eqnarray*}
    \Norm{f(p)-f(q)}\leq
    \max_{s\in [0,1]}\Norm{\md f|_{\gamma(s)}}_O \underbrace{\int_0^1\Norm{\dot\gamma(s)}\md s}_{\Norm{p-q}}
= \max_{z\in [p,q]}\Norm{\md f|_z}_O \Norm{p-q}\, .
  \end{eqnarray*}
\end{Bew}

\begin{Sat}
  Sei $\Omega\subset\mb{R}^n$ eine offene Menge, $f:\Omega\to\mb{R}^k$ (eine differenzierbare Abbildung mit stetigen partiellen Ableitungen) eine $\mb{C}^1$ Funktion und $\gamma [a,b]\to\Omega$ eine $\mb{C}^1$ Kurve.
\end{Sat}
\begin{Def}
  $f\in\mb{C}^k(U,\mb{R}^m)$ falls die partielle Ableitungen von $f_i$ mit Ordnung $\leq k$ existieren und stetig sind ($f=(f_1,\cdots,f_m)$). Dann:
  \[\Norm{f(\gamma(b))-f(\gamma(a))}\leq \left[ \sup_{t\in [a,b]}\Norm{\md f|_{\gamma(t)}}_0 \right]\underbrace{\int_a^b\Norm{\dot \gamma(t)}\md t}_{\text{Länge der Kurve}}\]
  $\gamma:[a,b]\to\Omega\subset\mb{R}^n$, $\gamma=(\gamma_1,\cdots,\gamma_n)$, $\dot\gamma=(\gamma_1',\cdots,\gamma_n')$
\end{Def}
\begin{Bew}
  Sei $\phi:[a,b]\to\mb{R}^k$ die Funktion
  \[\phi(t):=f(\gamma(t))=f\circ\gamma\]
  Kettenregel
  \begin{equation}
    \label{e:1104131}
    \md \phi|_t=\md f|_{\gamma(t)}\md \gamma|_t
  \end{equation}
  \[\phi:[a,b]\to\mb{R}^k\]
  \[\md \phi|_t:\mb{R}\to\mb{R}^k\s\text{lineare Abbildung}\]
  $\phi=(\phi_1,\cdots,\phi_k)$
  \[ \begin{pmatrix}
    \Part{\phi_1}{t} \\ \vdots \\ \Part{\phi_k}{t} 
  \end{pmatrix} = \begin{pmatrix}
    \phi_1' \\ \vdots \\ \phi_k'
  \end{pmatrix} = \dot\phi \]
  Sei $A$ die Jacobi-Matridx für $\md f$. Kettenregel:
  \[\underbrace{\dot\phi(t)=A(\gamma(t))\cdot \dot\gamma(t)}_{\text{Matrix-Darstellung von \ref{e:1104131}}}\]
  \[f(\gamma(b))-f(\gamma(a))=\phi(b)-\phi(a)= \begin{pmatrix}
    \phi_1(b)-\phi_1(a) \\
    \vdots \\
    \phi_k(b)-\phi_k(a)
  \end{pmatrix}\]
  $\phi_i'$ ist eine stetige Funktion:
  \begin{eqnarray*}
    \phi_i'(t)=\sum_{j=1}^nA_{ij}(\gamma(t))\gamma_j'(t)\\
    \sum_{j=1}^n\Part{f_i}{x_j}(\gamma(t))\gamma_j'(t)
  \end{eqnarray*}
  \[\phi(b)-\phi(a)= \begin{pmatrix}
    \int_a^b\phi_1'(t)\md t\\
    \vdots \\
    \int_a^b\phi_k'(t)\md t
  \end{pmatrix} \]
  \begin{eqnarray*}
    \Norm{f(\gamma(b))-f(\gamma(a))}^2 \\
    = \Norm{\phi(b)-\phi(a)}^2\\
    = \sum^k_{i=1}\left( \int_a^b\phi_i'(t)\md t \right)^2\\
    \stackrel{\text{Dreiecksungleichung}}{\leq} \int_a^b
  \end{eqnarray*}
  \begin{Lem}\ul{Dreiecksungleichung} Seien $f,g:[a,b]\to\mb{R}$ zwei stetige Funktionen. Dann:
    \[\Abs{\int_a^bfg\md t}^2\leq\int_a^bf^2\int_a^bg^2\]
  \end{Lem}
  \begin{Bew}
    \begin{align*}
      \int_a^b\left( tf-\frac{1}{t}g \right)^2\geq 0\s\forall t\in\mb{R}\setminus\left\{ 0 \right\}\\
      \int_A^bt^2f^2+\int_A^b\frac{1}{t^2}g^2-\int_a^b2fg\geq 0\\
      \int_A^bfg\leq\underbrace{\frac{t^2}{2}\int_a^bf^2+\frac{1}{2t^2}\int_A^bg^2}_{\Phi(t)=\frac{t^2}{2}\alpha+\frac[{\beta}{2t^2}}\s \alpha,\beta >0, \Phi''>0\\
      \Phi'(t)=t\alpha-\frac{\beta}{t^3}\\
      \Phi'(t)=0\iff t^4=\frac{\beta}{\alpha}\iff t^2=\sqrt{\frac{\beta}{\alpha}}\\
    \end{align*}
    Wähle
    \[t^2=\sqrt{\frac{\beta}{\alpha}}=\frac{1}{2}\sqrt{\frac{\beta}{\alpha}}\alpha+\frac{\beta}{\sqrt{\frac{\beta}{\alpha}}}=\frac{1}{2}\sqrt{\beta\alpha}+\frac{1}{2}\sqrt{\beta\alpha}\]
  \end{Bew}
  \begin{equation}
    \label{e:1104132}
    \implies \int_a^bfg\leq\int_a^bf^2\int_a^bg^2 \s(=\sqrt{\beta\alpha})
  \end{equation}
  Zur Erinnerung: $\alpha,\beta>0$ $\implies$ \ref{e:1104132}
  \begin{eqnarray*}
    \alpha=0\implies \int_A^bf^2=0\implies f^2\equiv 0\\
    \implies f\equiv 0\implies \ref{1104132}\s\text{ist trivial}\\
    \implies \ref{e:1104132}\s\text{gilt ohne Einschränkung}
  \end{eqnarray*}
\end{Bew}
\begin{Kor}
  Wenn wir das gleiche Argument mit $-f$ und $g$ anwenden, dann:
  \begin{equation}
    \label{e:1104133}
    \int_a^b(-f)g\leq\sqrt{\int_a^b(f)^2\int_a^bg^2}
  \end{equation}
  \begin{eqnarray*}
    \xRightarrow{\ref{e:1104132} + \ref{e:1104133}}\Abs{\int_a^bfg}\leq\sqrt{\int_a^bf^2}\sqrt{\int_a^bg^2}\\
    \implies(\int fg)^2\leq \int_a^bf^2\int_a^bg^2
  \end{eqnarray*}
\end{Kor}
\begin{Lem}
  \ul{Dreiecksungleichung (2te Version)} $f,g:[a,b]\to\mb{R}^k$ zwei stetige Funktionen. Dann:
  \[\Abs{\sum_{i=1}^k\int_A^bf_ig_i}\leq\int_a^b\Norm{f}^2\int_a^b\Norm{g}^2\]
\end{Lem}
\begin{Bew}
  Das Argument:
  \[\int_a^b\Norm{tf-\frac{1}{t}g}^2\geq 0\]
  \begin{eqnarray*}
    \int_a^b\sum_{i=1}^k(tf_i-\frac{1}{t}g_i)^2\\
    =\int_a^b(t^2\sum f_i^2+\frac{1}{t^2}\sum g_i^2-2\sum f_i g)\\
    \implies \frac{t^2}{2}\int_a^b\Norm{f}^2+\frac{1}{2t^2}\int_a^b\Norm{g}^2\\
    \geq \int_a^b\sum f_ig_i
  \end{eqnarray*}
  Optimierung von $t$ $\implies$ die Ungleichung
  \[\int_a^b\sum f_ig_i\leq\sqrt{\int_a^b\Norm{f}^2\int_a^b\Norm{g}^2}\]
\end{Bew}
\begin{eqnarray*}
  \Norm{f(\gamma(b))-f(\gamma(a))}\\
  =\Norm{\phi(b)-\phi(a)}\\
  =\sqrt{\sum\left( \int_a^b\phi'(s)\md s \right)}\\
  \stackrel{\text{Dreiecksungleichung}}{\leq}\int_a^b\overbrace{\sqrt{\sum(\phi'(s))^2}}^{\Norm{\dot\phi}}\md s\\
\end{eqnarray*}
\begin{Lem}
  Sei $g:[a,b]\to\mb{R}^k$ eine stetige Funktion.
  \[\sqrt{\sum\left( \int_a^bg_i \right)^2}\leq\int_a^b\Norm{g}\]
  \ul{Dreiecksungleichung}
\end{Lem}
\begin{Bew}
  Sei $\varepsilon>0$ und Treppenfunktion $\alpha_i$ so dass $g_i-\varepsilon\leq\alpha_i\leq g_i+\varepsilon$, $\alpha_i-\varepsilon\leq g_i\leq \alpha_i+\varepsilon$.
  \begin{eqnarray*}
    \int_a\alpha_i-(b-a)\varepsilon\\
    \leq\int_a^b g_i\\
    \leq \int_a^b\alpha_i+(b-a)\varepsilon\\
  \end{eqnarray*}
  \[\Abs{\int_a^bg_i-f_a^b\alpha_i}\leq (b-a)\varepsilon\]
  \begin{eqnarray*}
    \Abs{\sqrt{\sum(\int g_i)^2}-\sqrt{\sum\int\alpha_i^2}}\\
    \leq\sqrt{\sum(\int g_i-\int \alpha_i)^2}\\
    \leq\sqrt{k(b-a)^2\varepsilon^2}\\
    \leq \sqrt{k}(b-a)\varepsilon
  \end{eqnarray*}
  \begin{eqnarray*}
    \Abs{\int_a^b\Norm{g}-\int_a^b\Norm{\alpha}}\\
    \leq\int_a^b\Abs{\Norm{g}-\Norm{\alpha}}\\
    \leq\int_a^b\Norm{g-\alpha}\\
    \leq\int_a^b\sqrt{k}\varepsilon\\
    =(b-a)\varepsilon
  \end{eqnarray*}
  \begin{Beh}
    \begin{equation}
      \label{e:1104134}
      \sqrt{\sum\left(\int_a^b\alpha_i\right)^2}\leq\int_a^b\Norm{\alpha}
    \end{equation}
  \end{Beh}
  \begin{eqnarray*}
    \ref{e:1104134}\implies
    \sqrt{\sum\left(\int_a^b g_i\right)^2}\leq\sqrt{\sum\left(\int_a^b\alpha_i\right)^2}+(b-a)\sqrt{k}\varepsilon \\
    \stackrel{\ref{e:1104134}}{\leq}\int_a^b\Norm{\alpha}+(b-a)\sqrt{k}\varepsilon\\
    \leq\int_a^b\Norm{g}+2(b-a)\sqrt{k}\varepsilon
  \end{eqnarray*}
  Wenn $\varepsilon\downarrow 0$:
  \[\sqrt{\sum\left( \int_a^bg_i \right)^2}\leq\int_a^b\Norm{g}\]
  Ohne Beschränkung der Allgemeinheit: $\exists$ eine Zerteilung von $[a,b]$ 
  \[a=c_0<c_1<\cdots<c_N=b\]
  jedes $\alpha_i$ ist konstant auf $[c_{j-1},c_j]=I_j$. Die Konstante ist $a_{i,j}$
  \[\alpha = \begin{pmatrix}
    \alpha_1\\ \vdots \\ \alpha_k
  \end{pmatrix}\]
  ist konstant auf $I_j$ mit Wert
  \[a_j = \begin{pmatrix}
    a_{1,j}\\ \vdots \\ a_{k,j}
  \end{pmatrix}\]
  \[\sqrt{\sum_{i=1}^k\left( \int_a^b\alpha_i \right)^2}=\sqrt{\sum_{i=1}^k\left( \sum_{j=1}^N\abs{I_j}\alpha_{i,j} \right)^2}=\Norm{a}\]
  \begin{eqnarray*}
    a:=\sum_{j=1}^N\abs{I_j}a_j\\
    = \begin{pmatrix}
      \sum_{j=1}^N\abs{I_j}\alpha_{1,j}\\
      \vdots \\
      \sum_{j=1}^N\abs{I_j}\alpha_{1,j}\\
    \end{pmatrix}\\
    = \Norm{\sum_{j=1}^N\abs{I_j}a_j}\\
    \stackrel{\text{Dreiecksungleichung}}{\leq}\sum_{j=1}^N\Norm{\abs{I_j}a_j}\\
    =\sum_{j=1}^N\abs{I_j}\Norm{a_j}\\
    =\int_a^b\Norm{\alpha}
  \end{eqnarray*}
  $\Norm{\alpha}$ hat den Wert $\sqrt{\sum_{i=1}^k\alpha_i^2}$ und desw3egen auf $I_j$ ist dieser Wert
  \[\sqrt{\sum^k_{i=1}a_{i,j}^2}=\Norm{a_j}\equiv\Norm{\alpha}\]
\end{Bew}
\begin{Bew}
  vom Schrankensatz. $\phi=f\circ \gamma$, $\phi(t)=f(\gamma(t))$
  \begin{eqnarray*}
    f(\gamma(b))-f(\gamma(a))\\
    =\phi(b)-\phi(a)\\
    = \begin{pmatrix}
      \int_a^b\phi_1'(s)\md s\\
      \vdots \\
      \int_a^b\phi_k'(s)\md s\\
    \end{pmatrix}\\
    \xRightarrow{\text{Dreiecksungleichung}}\Norm{f(\gamma(b))-f(\gamma(a))}\\
    \leq \int_a^b\Norm{\dot\phi(s)}\md s\\
    \stackrel{\text{Kettenregel}}{=}\int_a^b\Norm{\md f|_{\gamma(\delta)}(\dot\gamma(s))}\md s\\
    \leq \int_A^b\Norm{\md f|_{\gamma(s)}}_O\Norm{\dot\gamma(s)}\md s\\
    \leq \max_{s\in [a,b]}\Norm{\md f|_{\gamma(s)}}_O\int_a^b\Norm{\dot\gamma(s)}\md s
  \end{eqnarray*}
  Vorsicht: $\Norm{\cdot}_O$ ist eine Norm, deswegen stetig!
\end{Bew}
\begin{Kor}
  $f:\Omega\to\mb{R}^k$ $\mb{C}^1$ Funktion, $[p,q]\subset\Omega$. Dann:
  \[\Norm{f(p)-f(q)}\leq\max_{z\in [p,q]}\Norm{\md f|_z}\Norm{p-q}\]
\end{Kor}
\begin{Bew}
  Wenden den Satz an $f$ und $\gamma:[0,1]\to\Omega$ ist $\gamma(a)=(1-s)p+sq$, $\dot \gamma=q-p$
  \begin{eqnarray*}
    \Norm{f(p)-f(q)}\leq\\
    \max_{s\in [p,q]}\Norm{\md f|_{\gamma(s)}}\underbrace{\int_0^1\Norm{\dot\gamma(s)}\md s}_{\Norm{p-q}}
  \end{eqnarray*}
\end{Bew}

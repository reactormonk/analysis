\section{Metrik und Topologie des euklidischen Raumes}
$\mb{R}^n=\left\{ \left( x_1,\cdots,x_n \right), x\in\mb{R} \right\}$.
Wir f\"uhren verschiedene neue Begriffe in $\mb{R}^n$ ein:
\begin{itemize}
  \item die Euklidische Norm
  \item der Euklidische Abstand
  \item die entsprechende Topologie.
\end{itemize}
Wir betrachten gleichzeitig die entsprechenden Verallgemeinerungen,
d.h. die ``Abstrakte Theorien'' der
\begin{itemize}
  \item Normierten Vektorr\"aume
  \item Metrischen R\"aume
  \item Topologischen R\"aume.
\end{itemize}
\begin{Def}
  Sei $x\in\mb{R}^n$ ($x=(x_1,\cdots,x_n)$, $x_i\in\mb{R}$). Die Euklidische Norm
von $x$ ist 
  \[\Norm{x}_e=\sqrt{x_1^2+\cdots+x_n^2}=\sqrt{\sum_{i=1}^nx_i^2}\]
(wir schreiben oft $\|x\|$ anstatt $\|x\|_e$).
\end{Def}

Intuitiv: $\Norm{x}=$''der Abstand zwischen $x$ und 0``.  In der Tat, wenn $n=2$,
das Pytaghoras Theorem zeigt dass $\|x\|_e$ die L\"ange des Segments mit Extrema
$x$ and $0$ ist. 

\begin{Lem}\label{l:norm}
  $\Norm{.}$ erf\"ullt die Regeln
  \begin{enumerate}
    \item $\Norm{x}\geq 0$ und $\Norm{x}=0\iff x=0$
    \item $\Norm{\lambda x}=\Abs{\lambda}\Norm{x}$ $\forall \lambda\in\mb{R}$, $\forall x\in\mb{R}$
    \item $\Norm{x+y}\leq\Norm{x}+\Norm{y}$ $\forall x,y\in\mb{R}$
  \end{enumerate}
\end{Lem}
\begin{Bew}
  \begin{enumerate}
    \item $\geq 0$ trivial
      \[x=0\implies \sum x_i^2=0\implies \Norm{x}=0\]
      \[x=0\Leftarrow x_i=0\;\; \forall i\Leftarrow \sum x_i^2=0\Leftarrow \Norm{x}=0\]
    \item \[\Norm{\lambda x}=\sqrt{\sum^n_{i=1}(\lambda x_i)^2} = \sqrt{\lambda^2\sum x^2}=\Abs{\lambda}\sqrt{\sum x^2}=\Abs{\lambda}\Norm{x}\]
%\[\Abs{\lambda}=\frac{\Norm{x}\Abs{\lambda}}{\Norm{x}}\]
    \item Diese Aussage ist \"aquivalent zu
      \[\iff \underbrace{\Norm{x+y}^2}\leq \Norm{x}^2+\Norm{y}^2+2\Norm{x}\Norm{y}\]
Wir rechnen
      \[\sum_{i=1}^n(x_i+y_i)^2=\sum_{i=1}^n\left( x_i^2+y_i^2+2x_iy_i \right)=\Norm{x}^2+\Norm{y}^2\overbrace{2\sum_i x_iy_i}^{Skalarprodukt}\]
Wir definieren 
\[\langle x,y\rangle := \sum_{i=1}^n  x_iy_i\]
Wir brauchen dann die ber\"uhmte Cauchy-Schwartz Ungleichung, d.h.
      \[\langle x,y\rangle\leq \Norm{x}\Norm{y}\, .\]
Diese Ungleichung ist der Inhalt des n\"achsten Satzes.
  \end{enumerate}
\end{Bew}
\begin{Sat}{Cauchy-Schwartzsche Ungleichung}
  \[\sum^n_{i=1}x_iy_i\leq\sqrt{\sum_{i=1}^nx_i^2}\sqrt{\sum_{i=1}^ny_i^2}\]
\end{Sat}
\begin{Bew}
  OBdA $y\neq 0$ ($y=0$ trivial)
  \[t\to g(t)=\sum_{i=1}^n(x_i+ty_i)^2=\left( \sum x_i^2 \right)+2t\sum x_iy_i+t^2\sum y_i^2\]
  \[=\Norm{x}^2+2t\seq{x,y}+\Norm{y}^2t^2\]
  Sei $t_0=\frac{\seq{x,y}}{\Norm{y}^2}$, dann
  \[0\leq g(t_0)=\Norm{x}^2-2\frac{\seq{x,y}^2}{\Norm{y}^2}+\Norm{y}^2\frac{\seq{x,y}^2}{\Norm{y}^4}
 =\Norm{x}^2-\frac{\seq{x,y}^2}{\Norm{y}^2}\]
  \[\implies \seq{x,y}^2\leq\Norm{x}^2\Norm{y}^2\implies \Abs{\seq{x,y}}\leq\Norm{x}\Norm{y}\]
\end{Bew}
\begin{Def}
  Ein normierter Vektorraum ist ein reeller Vektorraum $V$ mit einer Abbildung $\Norm . :V\to\mb R$ so dass:
  \begin{enumerate}
    \item $\Norm x\geq 0$ und $\Norm x=0\iff x=0$ (Nullvektor)
    \item $\Norm{\lambda x}=\abs\lambda\Norm x$ $\forall \lambda\in\mb R$, $\forall x\in V$
    \item $\Norm{x+y}\leq\Norm{x}+\Norm{y}$ $\forall x,y\in V$
  \end{enumerate}
\end{Def}
\begin{Bsp}
  $V=\mb{R}^n$
  \[\Norm{x}_p=\left(\sum\Abs{x_i}^p\right)^{\frac{1}{p}}\s p\geq 1\, .\]
$\|\cdot\|_2$ ist die Euklidische Norm.
\end{Bsp}
\begin{Def}
  Seien $x,y\in\mb{R}^n$. Die Euklidische Metrik ist $d(x,y):=\Norm{x-y}$.
\end{Def}
\begin{Lem}\label{l:euk_met}
  \begin{enumerate}
    \item $d(x,y)\geq 0$ und $d(x,y)=0\iff x=y$
    \item $d(x,y)=d(y,x)$
    \item $d(x,z)\leq d(x,y)+d(y,z)$ (Dreiecksungleichung)
  \end{enumerate}
\end{Lem}
\begin{Bew} Die erste Zwei Aussagen sind trivial. Um die letzte zu beweisen:
  \[\Norm{x-z}\leq\|\underbrace{x-y}_{=:v}\|+\|\underbrace{\Norm{y-z}}_{=:w}\|\, .\]
Aber $x-z=v+w$. Wir wenden die dritte Aussage von Lemma \ref{l:norm} an:
  \[d (x,z) = \Norm{v+w}\leq\Norm v+\Norm w = d (x,y)+d(y,z)\, .\]
\end{Bew}
\begin{Def}
  Ein metrischer Raum ist eine Menge $X$ mit einer Abbildung
  \[d:X\times X\to\mb{R}\s (x,y)\mapsto d(x,y)\in\mb{R}\]so dass
  \begin{enumerate}
    \item $d(x,y)\geq 0$ und $d(x,y)=0\iff x=y$ $\forall x,y\in X$
    \item $d(x,y)=d(y,x)$ $\forall x,y\in X$
    \item $d(x,z)=d(x,y)+d(y,z)$ $\forall x,y,z\in X$
  \end{enumerate}
\end{Def}
\begin{Lem}
  Sei ($V$, $\Norm .$) ein normierter Vektorraum. Dann sind $V$ und $d(x,y)=\Norm{x-y}$ ein metrischer Raum.
\end{Lem}
\begin{Bew} Wir nutzen das gleiche Argument vom Lemma \ref{l:euk_met}.
\end{Bew}
\begin{Def}
  Die offene Kugel mit Radius $r>0$ und Mittelpunkt $x\in\mb{R}^n$ ist die Menge
  \[K_r(x)=\left\{ y\in\mb{R}^n, d(x,y)<r \right\}\]
(Wir werden auch oft $B_r (x)$ statta $K_r (x)$ nutzen.)
\end{Def}
\begin{Def}
  Eine Menge heisst ''Umgebung`` von $x$, wenn $V$ eine offene Kugel mit Mittelpunkt $x$ enth\"alt.
\end{Def}
\begin{Def}
  Eine Menge $U\subset\mb{R}^n$ heisst offen falls $\forall x\in U$ ist $U$ eine Umgebung von $x$, d.h.
  \[\forall x\in U\s\exists \s\text{eine Kugel}\s K_r(x)\subset U\]
\end{Def}
\begin{Bem}
Die Dreiecksungleichung impliziert dass jede offene Kugel eine offene Menge ist. In der Tat,
sei $y\in K_r (x)$. Dann $\rho:=d(x,y) < r$. Sei $\tau:= r-\rho>0$. Falls $z\in K_\tau (y)$,
dann $d(x,z)\leq d(x,y) + d(y,z) = \rho + d (y,z) < \rho +\tau =r$. D.h., $K_\tau (y)\subset K_r (x)$.
Das beweist dass $K_r (x)$ eine Ungebung ihrer ganzen Elementen ist, d.h. $K_r (x)$ ist offen. 
\end{Bem}
\begin{Sat}
  \begin{enumerate}
    \item $\varnothing$ und $\mb{R}^n$ sind offen
    \item Der Schnitt \ul{endlich vieler} offener Mengen ist auch offen.
    \item Die Vereinigung einer \ul{beliebigen} Familie offener Mengen ist auch offen.
  \end{enumerate}
\end{Sat}
\begin{Bew}
  \begin{enumerate}
    \item $\mb{R}^n$ trivialerweise offen, auch $\varnothing$
    \item Sei $x\in U\cap\dots\cap U_N$
      \[\forall i\in\left\{ 1,\dots,N \right\}\s \quad \exists r_i>0 \;\;\mbox{so dass}\;\; K_{r_i}(x)\subset U_i\]
      Sei $r=\min\left\{ r_i,\dots,r_N \right\} > 0$;
      \[\implies K_r(x)\subset U_i\quad\forall i\implies K_r(x)\subset U_1\cap\dots\cap U_N\]
    \item $\left\{ U_\lambda \right\}_{\lambda\in \Lambda}$. Sei $U=\bigcup_{\lambda\in\Lambda}U_\lambda$
      \[x\in U\implies x\in U_\lambda\s\text{f\"ur ein}\s\lambda\in\Lambda\]
      \[\implies \exists K_r(x)\subset U_\lambda\subset U.\]
  \end{enumerate}
\end{Bew}
\begin{Def}
  Ein topologischer Raum ist eine Menge $X$ und eine Menge $O$ von Teilmengen von $X$ so dass:
  \begin{enumerate}
    \item $\varnothing, X\in O$
    \item $U_1\cap\dots\cap_N\in O$ falls $U_i\in O$
    \item $\bigcap_{\lambda\in\Lambda}U_\lambda\in O$ falls $U_i\in O$
  \end{enumerate}
$O$ heisst die {\em Topologie}.
\end{Def}
\begin{Sat}
  Sei $(X,d)$ ein metrischer Raum. Wir definieren die entsprechende offene Kugel mit Mittelpunkt $x\in X$
und Radius $r>0$:
  \[K_r(x)=\left\{ y=X: d(x,y)<r \right\}\]
  Umgebungen und offene Mengen sind wie im Euklidischen Fall definiert. $O=\left\{ \text{offene Menge} \right\}$ definiert eine Topologie.
\end{Sat}

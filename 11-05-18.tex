\begin{Def}
  Eine Teilmenge $\Omega\subset\mb{R}^n$ heisst normal falls $\exists$(ein System von Koordinaten) und zwei Funktionen
  \[g\leq f:\underbrace{\Omega'}_{\subset\mb{R}^{n-1}}\to\mb{R}\]
  so dass
  \[\Omega'=\left[ a_1,b_1 \right]\times\left[ a_2,b_2 \right]\times\cdots\times\left[ a_{n-1},b_{n-1} \right]\]
  und
  \[\Omega=\left\{ x:(x_1,\cdots,x_{n-1})\in\Omega'\s\text{und}\s g(x_1,\cdots,x_{n-1})\leq x_n\leq f(x_1,\cdots,x_{n-1}) \right\}\]
\end{Def}
\begin{Bsp}
  in zwei Dimensionen
  % TODO Skizze
\end{Bsp}
\begin{Bem}
  In diesem Fall
  \[\int_\Omega F=\int_{a_1}^{b_1}\left( \int_{a_2}^{b_2}\left( \cdots \int_{a_{n-1}}^{b_{n-1}}\overbrace{\left( \int^{f(x_1,\ldots,x_{n-1})}_{g(x_1,\ldots,x_{n-1})}F(x_1,\cdots,x_{n-1},x_n)\md x_n \right)}^{G(x_1,\ldots,x_{n-1})}\md x_{n-1} \right)\cdots \right)\md x_1\]
  \[=\int_{a_1}^{b_1}\cdots\int_{a_{n-1}}^{b_{n-1}}\int_{g(x_1,\ldots,x_n)}^{f(x_1,\ldots,x_n)}F(x_1,\cdots,x_{n-1},x_n)\md x_n\cdots\md x_1\]
\end{Bem}
\begin{Bem}
  Wichtig: die Funktion
  \[G(x_1,\cdots,x_{n-1})=\int_{f(x_1,\ldots,x_{n-1})}^{g(x_1,\ldots,x_{n-1})}F(x_1,\cdots,x_{n-1},x_n)\md x_n\]
  ist stetig (ohne Beweis)
\end{Bem}
\begin{Bem}
  in 2d mit $F=1$ ist die Formel
  \[\text{Inhalt von $\Omega$} = \int_{a_1}^{b_1}\left( \int_{g(x_1)}^{f(x_1)}1\md x_2 \right)\md x_1 \left( =\int_{a_1}^{b_1}\left( f(x_1)-g(x_1) \right) \right)\]
\end{Bem}
% TODO Skizze
$\forall$ Intervall $I_i$
\begin{eqnarray*}
  m_i=\min_{I_i}f\\
  m_i=\max_{I_i}f
\end{eqnarray*}
\[\underbrace{\cup_{i=1}^NI_ix[0,m_i[}_{\text{Mass}=\sum^N_{i=1}\frac{b-a}{N}m_i}\subset\Omega\subset \underbrace{\cup_{i=1}^NI_i\times [0,M_i[}_{\text{Mass}=\sum^N_{i=1}\frac{b-a}{N}M_i}\]
Sei $\varepsilon>0$ gegeben aus der gleichmässigen Stetigkeit von $f$ $\exists$ $N_0>0$ so dass
\[M_i-m_i<\varepsilon\s\forall \s\text{Zerlegung mit}\s N\geq N_0\]
So Mass Aussen - Mass Innen
\begin{eqnarray*}
  =\sum^N_{i=1}\frac{b-a}{N}(M_i-m_i)\\
  \leq \varepsilon \sum^N_{i=1}\frac{b-a}{N}\varepsilon(b-a)
\end{eqnarray*}
\[\Limi{N}(\text{Mass Aussen} - \text{Mass Innen})=0\]
% TODO Skizze
\begin{eqnarray*}
  \Omega=\left\{ (x_1,x_2,x_3):(x_1,x_2)\in\Omega - \sqrt{1-(x_1^2+x_2^2)}\leq x_3\leq \sqrt{1-(x_1^2+x_2^2)} \right\}\\
  \Omega'=\left\{ x_1^2+x_2^2\leq 1 \right\}=\left\{ x_1\in[-1,1],-\sqrt{1-x_1^2}\leq x_2\leq\sqrt{1-x_2^2} \right\}\\
  1=\left( \sqrt{x_1^2+x_2^2} \right)^2+f(x_1,x_2)^2\\
  f(x_1,x_2)=\sqrt{1-(x_1^2+x_2^2)}
\end{eqnarray*}
\begin{Def}
  Ultranormale Bereiche $\exists$ ein System von Koordinaten und $a_1\leq b_1$
  \begin{eqnarray*}
    \Omega_1=\left[ a_1,b_1 \right]\\
    f_2,g_2:\Omega_1\to\mb{R}\\
    \Omega_2=\left\{ (x_1,x_2):x_1\in \Omega_1, f_2(x_1)\leq x_2\leq g_2(x_1) \right\}\\
    f_3,g_3:\Omega_2\to\mb{R}\\
    \Omega_3=\left\{ (x_1,x_2,x_3):(x_1,x_2)\in \Omega_2, f_3(x_1,x_2)\leq x_3\leq g_3(x_1,x_2) \right\}\\
    \Omega=\Omega_n=\left\{ (x_1,\cdots,x_{n-1},x_n):(x_1,\cdots,x_{n-1})\in \Omega_{n-1}, f_n(x_1,\cdots,x_{n-1})\leq x_3\leq g_3(x_1,\cdots,x_{n-1}) \right\}\\
  \end{eqnarray*}
\end{Def}
\begin{Bem}
  In diesem Fall
  % TODO overfull
  \[\int_\Omega F=\int_{a_1}^{n_1}\int_{f_2(x_1)}^{g_2(x_1)}\int_{f_3(x_1,x_2)}^{g_3(x_1,x_2)}\cdots\int_{f_n(x_1,\ldots,x_{n-1})}^{g_n(x_1,\ldots,x_{n-1})}F(x_1,\cdots,x_{n-1},x_n)\md x_n\cdots\md x_3\md x_2\md x_1\]
\end{Bem}
\begin{Bsp}
  Inhalt einer 3d Kugel mit Radius 1
  \begin{eqnarray*}
    =\int_{-1}^1\int_{-\sqrt{1-x_1^2}}^{\sqrt{1-x_1^2}}\int_{-\sqrt{1-(x_1^2+x_2^2)}}^{\sqrt{1-(x_1^2+x_2^2)}} 1\md x_3\md x_2\md x_1 =\\
    \int_{-1}^1\int_{-\sqrt{1-x_1^2}}^{\sqrt{1-x_1^2}}(\sqrt{1-(x_1^2+x_2^2)}+\sqrt{1-(x_1^2+x_2^2)}) 1\md x_2\md x_1\\
    = \frac{4\pi}{3}
  \end{eqnarray*}
\end{Bsp}
\begin{Sat}
  Sei $\Omega$ eine Menge, die:
  \begin{enumerate}
    \item beschränkt ist
    \item dessen Rand eine $\mb{C}^1$-Mannigfaltigkeit ist
  \end{enumerate}
  Dann: $\Omega$ ist Peano-Jordan messbar $\exists$ eine Zerlegung von $\Omega$ in endlich viele ultranormale Bereiche.
  % TODO Skizze
  $\implies$ Formel $\int_\Omega F$ (wenn $F$ eine stetig Funktion ist)
\end{Sat}
\paragraph{Ziel}
den Inhalt $B_1(0)\subset\mb{R}^3$ zu berechnen
\begin{Bew}
  Sei $\Omega=\left\{ x_1^2+x_2^2\leq r^2 \right\}\subset\mb{R}^2$, $I=[a,b]\subset\mb{R}$ $\Omega\times I\subset\mb{R}^3$. $\Omega\times I$ ist messbar und Inhalt = $=\pi r^2(b-a)$ $\Omega$ ist ein normaler Bereich:
  \[\left\{ (x_1,x_2):x_1\in [-1,1], -\sqrt{1-x_1^2}\leq x_2\leq \sqrt{1-x_1^2} \right\}\]
  \[\Omega\times I=\left\{ (x_1,x_2,x_3):(x_1,x_2)\in\Omega\s\text{und}\s a\leq x_3\leq b \right\}\]
  \begin{eqnarray*}
    \text{Inhalt}\s =\int_{-1}^1\int_{-\sqrt{1-x_1^2}}^{\sqrt{1-x_1^2}}\int_a^b 1\md x_3\md x_2\md x_1\\
    =(b-a)\left( \int_{-1}^1\int_{-\sqrt{1-x_1^2}}^{\sqrt{1-x_1^2}}1\md x_2\md x_1 \right)\\
    =(b-a)=\int_\Omega 1=(b-a)\text{Inhalt}(\Omega)=(b-a)\pi r^2
  \end{eqnarray*}
  % TODO Skizze
  \begin{eqnarray*}
    I_i=[a_i,b_i]\times [-\sqrt{1-a_i^2},\sqrt{1-a_i^2}]
  \end{eqnarray*}
  Rotation: Zylinder
  \[ [a_i,b_i]\times \left\{ x_2^2+x_3^2<\overbrace{1-a_i^2}^{r^2} \right\}\]
  $b_i-a_i=\frac{1}{N}$ $2N$ Intervalle:
  \begin{eqnarray*}
    \underbrace{\cup_{i=1}^{2N}}_{\Omega_N}\subset B_i(0)\subset\cup_{i=1}^{2N}c_i^a
  \end{eqnarray*}
  Zylinder
  \[c_i^a=[a_i,b_i]\times \left\{ x_2^2+x_3^2=1-b_i^2 \right\}\]
  $\Omega_N$ ist messbar und
  \begin{eqnarray*}
    \text{Inhalt} I = \sum_{i=1}^{2N}(b_i-a_i)\pi\overbrace{(1-a_i^2)}^{f(a_i)}\\
    \text{Inhalt} A = \sum_{i=1}^{2N}(b_i-a_i)\pi\overbrace{(1-b_i^2)}^{f(b_i)}
  \end{eqnarray*}
  \begin{eqnarray*}
    \Limi{N}\text{Inhalt} I=\int_{-1}^1\pi(1-x^2)\md x=\pi(x-\frac{x^3}{3})|_{-1}^1=\pi\left( \frac{2}{3}\left( \frac{2}{3} \right) \right)=\frac{4\pi}{3}\\
    \Limi{N}\text{Inhalt} I=\int_{-1}^1\pi(1-x^2)\md x=\pi = \cdots =\frac{4\pi}{3}\\
  \end{eqnarray*}
  Wenn die Kugel Radius $r$ hat
  \[\int_{-r}^r\pi(r^2-x^2)\md x=\pi\left( xr^2-\frac{x^3}{3} \right)|_{-r}^r=\frac{4}{3}\pi r^3\]
  Der Inhalt von 
  \[B_1(0)=\overbrace{\sup_{\text{Innere Approximation}}(\text{Inhalt $I$})}^S = \overbrace{\inf_{\text{Äussere Approximation}}\text{Inhalt} A}^I\]
  \[S\leq \frac{4\pi}{3}\]
  der Limes von eriner Folge von äusseren Approximation. ``Konkrete'' innere Approximation:
  \[S\geq\Limi{N}\text{Inhalt} I=\frac{4\pi}{3}\]
  $S=\frac{4\pi}{3}$, $I=\frac{4\pi}{3}$ $\implies$ Messbarkeit von $B_1(0)$ und Formel $=\frac{4\pi}{3}$
\end{Bew}
\begin{Bew}
  ``Richtiger'' $S=\frac{4\pi}{3}$ Sei nun $\cup_{i=1}^MW_i\subset\cup_{i=1}^{2N}C_i^a$ eine Innere Approximation.
  \[\text{Inhalt}(\cup_{i=1}^MW_i)=\sum^M_{i=1}\abs{W_i}\leq \Limi{N}\underbrace{\sum_{j=1}^{2N}\text{Inhalt}(C_j^a)}_{A_N}\leq\frac{4\pi}{3}\]
  $\implies$ $S\leq \frac{4\pi}{3}$
  \[I_N:=\sum_{i=1}^{2N}\text{Inhalt}(C_i)=\sup_{\text{Erlaubte innere Approximation von}\s \cup_{i=1}^{2N}C_i}\]
  $\implies$ $\exists$ eine innere Approximation mit Würfel von $\cup_{i=1}^{2N}C_i$ $W_1,\cdots,W_N$ so dass
  \[\sum_{i=1}^N\text{Inhalt}(W_i)\geq I_N-\frac{1}{N}\]
  Deswegen
  \[S\geq I_n\frac{1}{N}\]
  weil $\cup_{i=1}^{N'}W_i$ eine innere Approximation für $B_1(0)$ ist
  \[S\geq \Limi{N}\left( I_N-\frac{1}{N} \right)=\frac{4\pi}{3}\]
\end{Bew}

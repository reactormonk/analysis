\subsection{Die Exponentialfunlktion auf der reellen Gerade}

%\begin{itemize}
%  \item $f(z+w)=f(z)f(w)$
%  \item $\Limo{z}\frac{l^z-1}{z}=1$
%\end{itemize}
\begin{Sat} Die Funtion $\mb{R}\ni x \mapsto e^x \in \mb{R}$ ist
  \begin{enumerate}
    \item positiv
    \item monoton steigend
    \item bijektiv (falls $\mb{R}$ durch $\mb{R}^+$ ersetzt wird).
  \end{enumerate}
\end{Sat}
\begin{proof}[Beweis]
  \begin{enumerate}
    \item \[e^x=e^{\frac{x}{2}+\frac{x}{2}} = (e^\frac{x}{2})^2 \geq 0\]
$e^x = 0$ ist nicht m\"oglich, sonst w\"are $e^{xq}=0$ f\"ur alle 
$q\in \mb{Q}$ und, wegen der Dichtheit der rationalen Zahlen und
der Stetigkeit von $f$, $e^x\equiv 0$.
    \item \[\frac{e^{x+h}}{e^x}=e^h=1+\frac{h}{1!}+\cdots > 1\]
    \item z.z.: $\forall y\in\mb{R}^+$
      \[\exists x: e^x=y\]
      Falls $y\geq 1$
      \[e^0=1\leq y\leq e^y\stackrel{\text{ZWS}}{\implies}\exists x:e^x=y\]
      Falls $0<y<1$, dann betrachte $\frac{1}{y}>1$
      \[\exists x: e^x=\frac{1}{y}\implies e^{-x}=y\]
  \end{enumerate}
\end{proof}
\begin{Sat}[vom Wachstum]\label{s:wachstum1}
  \[\Limi{x}\frac{e^x}{x^n}=+\infty\]
  \[\lim_{x\to-\infty}x^ne^x=0\]
\end{Sat}
\begin{proof}[Beweis]
  \[e^x>\frac{x^{n+1}}{(n+1)!}\implies\frac{e^x}{x^n}>\frac{x}{(n+1)!}\stackrel{x\to\infty}{\to}\infty\]
  \[x^ne^x=\frac{x^n}{e^{-x}}=(-1)^n\frac{(-x)^n}{e^{-x}}\stackrel{x\to\infty}{\to}\infty\]
\end{proof}
\subsection{Natürlicher Logarithmus}
\begin{Def}
  $\ln:\mb{R}^+\to\mb{R}$ ist die Inverse der exponentiellen Funktion.
\end{Def}
\begin{Sat}
  \[\ln(xy)=\ln x + \ln y\]
\end{Sat}
\begin{proof}[Beweis]
  \[e^{\ln(xy)}=xy=e^{\ln x}e^{\ln y}=e^{\ln x + \ln y}\]
  \[\implies \ln(xy)=\ln x + \ln y\]
\end{proof}
\begin{Sat}[vom Wachstum 2] \label{s:wachstum2}
  \[\Limi{x}\frac{\ln x}{\sqrt[n]{x}}=0\]
\end{Sat}
\begin{proof}[Beweis]
  \[\frac{\ln x}{\sqrt[n]{x}}=\frac{\ln e^{ny}}{\sqrt[n]{e^{ny}}}=\frac{ny}{e^y}\]
  \[\exists y:x=e^{ny}\]
  \[\underbrace{y\to\infty}_{\iff x\to\infty}\implies \frac{\ln x}{\sqrt[n]{x}}\to 0\]
\end{proof}
\begin{Sat}
  \[\Limo{x}\frac{\ln(1+x)}{x}=1\]
\end{Sat}
\begin{proof}[Beweis]
  \[\Limo{x}\frac{\ln(1+x)}{x}=\Limo{y}\frac{\ln e^y}{e^y-1}=\Limo{y}\frac{y}{e^y-1}=1\]
\end{proof}
\begin{Bem}
  $\ln:\mb{R}^+\to\mb{R}$ ist stetig
\end{Bem}
\begin{Bem}
  $y=\frac{m}{n}\in\mb{Q}, n\in\mb{N}, a>0$
  \[\sqrt[n]{a^n}=\frac{m}{n}=e^{y\ln a}\]
  Warum?
  \[f(1)=e^{\ln a} =a\]
  \[f(q+r)=f(q)f(r)\]
  \[f:\mb{Q}\to\mb{R}\]
  \[\implies a^y=e^{y\ln a}\]
\end{Bem}
\begin{Def}
  $a>0$, $z\in\mb{C}$
  \[a^z:=e^{z\ln a}\]
\end{Def}
\begin{Sat}
  \begin{enumerate}
    \item \[a^{x+y}=a^xa^y\s(x,y\in\mb{C})\]
    \item \[(a^x)^y=a^{xy}\s(x,y\in\mb{R})\]
    \item \[(ab)^x=a^xb^x\]
  \end{enumerate}
\end{Sat}
\begin{proof}[Beweis]
  \begin{enumerate}
    \item \[a^{x+y}=w^{(x+y)\ln a}=e^{x\ln a+y\ln a}=e^{x\ln a}e^{y\ln a}=a^xa^y\]
    \item ähnlich
    \item \"ahnlich
  \end{enumerate}
\end{proof}
\begin{Sat}
  \begin{enumerate}
    \item \[\Limi{x}x^a= \begin{cases}
      \infty&a>0\\
      1&a=0\\
      0&a<0\\
    \end{cases}\] \label{i:101110a}
    \item \[\Limo{x}x^a= \begin{cases}
      0&a>0\\
      1&a=0\\
      \infty&a<0\\
    \end{cases}\]  \label{i:101110b}
    \item \[\Limi{x}x^a= \begin{cases}
      +\infty&a\geq0\\
      0&a<0\\
    \end{cases}\] \label{i:101110c}
    \item \[x^ae^x=+\infty\] \label{i:101110d}
    \item \[\Limo{x}\frac{a^x-1}{x}\ln a\] \label{i:101110e}
  \end{enumerate}
\end{Sat}
\begin{proof}[Beweis]
  \begin{enumerate}
    \item 
      \[\text{Bild}(x\mapsto x^a)=\mb{R}^+\implies\Limi{x}x^a=+\infty\]
      $a=0$ trivial
      $a>0$
      \[x^a=\frac{1}{x^{-a}}\to0\]
      (Wegen $-a>0$ und $x^{-a}\to\infty$)
    \item folgt aus \ref{i:101110a} durch die Substitution $x\mapsto \frac{1}{x}$. \ref{i:101110a} Falls $a>0$, $x^a$ monoton wachsend.
    \item $a\geq 0$ offensichtlich, $a<0$: $\exists n\in\mb{N}$, $a<-\frac{1}{n}$, $-a>\frac{1}{n}$
      \[x^a\ln x=\frac{\ln x}{x^{-a}}<\frac{\ln x}{x^{\frac{1}{n}}}\stackrel{\text{Satz \ref{s:wachstum2}}}{\to} 0\]
    \item $a>0$ trivial, $a<0$, $\exists n\in\mb{N}$ so dass $a>-n$ ($-a<n$)
      \[x^ae^x=\frac{e^x}{a^{-a}}>\frac{e^x}{x^n}\stackrel{\text{Satz \ref{s:wachstum1}}}{\to}\infty\]
    \item $\Limo{x}\frac{a-1}{x}=\ln a$
      \[\frac{a^x-1}{x}=\frac{e^{x\ln a}-1}{x}=\overbrace{\frac{e^{x\ln a}-1}{x\ln a}}^{\to 1}\ln a\to_{x\to 0} \ln a\]
  \end{enumerate}
\end{proof}
\subsection{Trigonometrische Funktionen}
\begin{Def} Falls $\phi$ die Gr\"osse eines Winkels (in Radianten) ist,
dann $\cos (\phi)$ und $\sin (\phi)$ sind die entsprenchenden Werte des
Cosinus und Sinus. Wir erweitern diese Funktionen auf der ganzen reellen
Gerade:
  \[\cos(\phi):=\cos(\phi-2\pi n) \qquad \mbox{falls } 2\pi n \leq \phi < 2\pi (n+1)\]
  \[\sin(\phi):=\sin(\phi-2\pi n) \qquad \mbox{falls } 2\pi n \leq \phi < 2\pi (n+1)\]
\end{Def}
\begin{Sat}
  Für $\phi$ klein genug gilt:
  \begin{enumerate}
    \item 
      \[\abs{\sin\phi}\leq\abs{\phi}\leq\frac{\abs{\sin\phi}}{\cos\phi}\]
    \item
      \[1-\cos\phi\leq\phi^2\]
  \end{enumerate}
\end{Sat}
\begin{proof}
  \begin{enumerate}
    \item Die Gr\"osse des Winkelns in Radianten ist die L\"ange des
entsprechenden Kreissektors (auf einem Kreis mit Radius 1): diese ist
gr\"osser als die L\"ange des (kleineren) Katheten.
    \item
      \[1-\cos\phi=\frac{(1-\cos\phi)(1+\cos\phi)}{1+\cos\phi}=\frac{1-(\cos\phi)^2}{1+\cos\phi}\leq\frac{\sin^2\phi}{1}\leq\phi^2\]
  \end{enumerate}
\end{proof}
\begin{Kor}
  \begin{enumerate}
    \item \[\Limo{\phi}\frac{\sin\phi}{\phi}=1\]
    \item \[\Limo{\phi}\frac{1-\cos\phi}{\phi}=0\]
    \item $\sin$ und $\cos$ sind stetig.
  \end{enumerate}
\end{Kor}
\begin{proof}[Beweis]
  \begin{enumerate}
    \item \[\frac{1}{\cos\phi}\leq\frac{\abs{\sin\phi}}{\phi}\leq 1\]
    \item \[0\leq\frac{1-\cos\phi}{\abs{\phi}}\leq\abs{\phi}\]
    \item Additionsregeln
      \[\cos(x+y)=\cos x\cos y-\sin x\sin y\]
      \[\sin(x+y)=\sin x\cos y+\cos x\sin y\]
  \end{enumerate}
\end{proof}
\begin{Sat}[von Euler]\label{s:Euler}
  \[e^{x+iy}=e^x(\cos x+\sin y) \qquad \forall x,y\in \mb{R}\]
\end{Sat}
\begin{Bew}
  Definiere $f(z)=e^x(\cos x+\sin y)$.
  $f$ erfüllt (AT) und (WT) im Satz \ref{s:Exp}% TODO ref
  \begin{itemize}
    \item[(AT)] folgt aus den Additionsregeln
    \item[(WT)] 2 Spezialfälle:
      \begin{itemize}
        \item $z=x\in\mb{R}$
          \[\Limo{z}\frac{f(z)-1}{z}=\Limo{x}\frac{e^x-1}{x}=1\]
        \item $z=iy$
          \[\Limo{z}\frac{f(z)-1}{z}=\Limo{y}\frac{\cos y+i\sin y-1}{iy}\]
          \[=\frac{1}{i}\Limo{y}\frac{\cos y-1}{y}+\frac{1}{i}\Limo{y}\frac{i\sin y}{y}=\frac{1}{i}0+\frac{1}{i}i=q\]
      \end{itemize}
      Der allgemeine Fall wird im Übungsblatt behandelt.
  \end{itemize}
\end{Bew}
\begin{Bem}(Was hat Euler gemacht?) Wegen der Taylor'schen Reihen:
  \[\cos y=\sum_{k=0}^\infty(-1)^k\frac{y^{2k}}{(2k)!}\]
  \[\sin y=\sum_{k=0}^\infty(-1)^k\frac{y^{2k+1}}{(2k+1)!}\]
das wussten die Mathematikern schon vor Euler seinen Satz entdeckte:
man kann diese Reihen mit der Differentialrechnung bestimmen
(und werden wir sp\"ater lernen).
Wenn man die Formel
  \[e^z:=\sum^\infty_{k=0}\frac{z^k}{k!}\]
f\"ur $z=iy$ anwendet:
  \[e^{iy}=\sum^\infty_{k=0}\frac{(iy)^k}{k!}=\underbrace{\sum_{k=0}^\infty(-1)^k\frac{y^{2k}}{(2k)!}}_{\cos y}+i\underbrace{\sum_{k=0}^\infty(-1)^k\frac{y^{2k+1}}{(2k+1)!}}_{\sin y}\]
  \[\implies e^{iy}=\cos y+i\sin y\]
  $e^{i\pi}=-1$ $\to$ die berühmte Formel von Euler.
\end{Bem}

\subsection{Noch andere spezielle Funktionen}
Wir definieren zuerst den Tangens:
\[\tan=\frac{\sin}{\cos}\]
Der Definitionsbereich dieser Funktion ist die reele Gerade ohne die Nullstellen des Cosinus,
\[\tan :\mb{R}\setminus\underbrace{\left\{ \frac{\pi}{2}+k\pi; k\in\mb{N} \right\}}
_{\text{Die Nullstellen des Cosinus}}\quad\to\quad\mb{R}\]
Geometrisch leicht zu sehen: 
\[\sin:\left[ -\frac{\pi}{2},\frac{\pi}{2} \right]\to\left[ 1,1 \right]
\qquad \mbox{ist injektiv und surjektiv.}\]
 Die Umkehrfunktion heisst arcsin:
\[\arcsin:\left[ -1,1 \right]\to\left[ -\frac{\pi}{2},\frac{\pi}{2}\,  \right]\, .\]
Analog ist
\[\cos:\left[ 0,\pi \right]\to\left[ -1,1 \right]\qquad \mbox{bijektiv.}\]
Die Umkehrfunktion heisst arccos:
\[\arccos:\left[ -1,1 \right]\to\left[ 0,\pi \right]\]
Auch der Tangens eingeschr\"ankt auf dem geeigneten Intervall ist bijektiv:
  \[\tan:\left] -\frac{\pi}{2},\frac{\pi}{2} \right[\to\mb{R}\]
Die Surjektivit\"at folgt aus der Stetigkeit 
und  
\[\lim_{x\to\pm\frac{\pi}{2}}\tan(x)=\pm\infty\, .\]
Die Injektivität werden wir später sehen (siehe Bemerkung \ref{b:tan_in}).

\[\arctan:\mb{R}\to\left] -\frac{\pi}{2},\frac{\pi}{2} \right[\]
ist die Umkehrfunktion.

Endlich wir definieren die hyperbolischen Funktionen:
\[\sinh(t)=\frac{e^t-e^{-t}}{2}\]
\[\cosh(t)=\frac{e^t+e^{-t}}{2}\]
\[\tanh(t)=\frac{\sinh (t)}{\cosh (t)}\]
(NB: Der Definitionsbereich von $\tanh$ ist die ganze reelle Gerade,
weil $\cosh (t)>0$ $\forall t\in \mb{R}$).
\begin{Bem}
  \[\cosh^2(t)-\sinh^2(t)=1\]
Nun, $\forall t\in\mb{R}$, $\left( \cos t, \sin t \right)$ geh\"ort dem Kreis mit Radius 1 und Mittelpunkt 0.
Die Punkten $\left( \cosh t, \sinh t \right)$ liegen auf einer Hyperbel.
\end{Bem}
\section{Differentialrechnung}
Eine affine Funktion $f:\mb{R}\to\mb{R}$ hat die Gestalt:
\[f(t)=c_0+m_0x\, .\]
Die Konstante $m_0$ (die Steigung der Gerade) ist leicht zu rechnen:
\[m_0=\frac{f(t_2)-f(t_1)}{t_2-t_1}\, ,\]
wobei $t_1\neq t_2$ sind zwei beliebige reelle Zahlen.
$f$ heisst linear wenn $c_0=0$.
\subsection{Die Ableitung}
Wir suchen
die beste Approximation von $f$ in der Nähe von einer Stelle $x_0$ 
mit einer affinen Funktion $g$, d.h. die Tangente zum Graphen von $f$
im Punkt $(x_0, f(x_0))$. Manchmal gibt es keine gute
Approximation mit einer affinen Funktion (z.B.
$f(x)=\abs{x}$ und $x_0=0$). Falls $\xi$ eine andere Stelle im
Definitionsbereich von $f$ ist, die Gerade 
\[
 g(x) = f (x_0) + \frac{f(\xi)-f(x_0)}{\xi- x_0} (x-x_0)
\]
enth\"alt die Punkten $(x_0, f(x_0))$ und $(\xi, f(\xi))$. 
Wenn $\xi-x$ sehr klein ist, diese Gerade ist ``fast'' die Tangente
im $(x_0, f(x_0))$.

\begin{Def}
  Sei $f:\left[ a,b \right]\to\mb{R}(\mb{C})$. Die Ableitung an der Stelle $x_0$ von $f$ ist
  \[f'(x)=\lim_{h\downarrow 0}\frac{f(x_0+h)-f(x_0)}{h}
\quad \left(= \frac{f(x_0+h)-f(x_0)}{(x_0+h)-x_0}\right)\, ,\]
falls der Limes exisitiert.  
Die Funktion heisst differenzierbar an der Stelle $x_0$, wenn die Ableitung $f'(x_0)$ existiert.
\end{Def}
\begin{Sat}
  $f:I\to\mb{C}$ ist in $x_0$ genau dann differenzierbar, wenn $\exists L:\mb{R}\to\mb{C}$ linear so dass
  \[\Limo{h}\frac{f(x_0+h)-f(x_0)-L(h)}{h}=0\]
\end{Sat}
\begin{proof}[Beweis]
  \[L\s\text{linear}\iff\exists m_0\in\mb{C}: L(h)=m_0h\s\forall h\in\mb{R}\]
Wir betrachten 
  \begin{equation}
    \label{e:differential151}
    \Limo{h}\frac{f(x_0+h)-f(x_0)-L(h)}{h} =
\Limo{h} \left(\frac{f(x_0+h)-f(x_0)}{h}-m_0\right)
  \end{equation}
und
  \begin{equation}
    \label{e:differential152}
    \Limo{h}\frac{f(x_0+h)-f(x_0)}{h}
  \end{equation}
Der Limes in \eqref{e:differential151} existiert und verschwindet genau
dann, wenn der Limes in \eqref{e:differential152} existitiert (d.h.
$f$ differenzierbar im $x_0$ ist) und gleicht $m_0$ (d.h. $m_0=f'(x_0)$).
\end{proof}
\begin{Sat}
  $f:I\to\mb{C}$ ist in $x_0\in I$ genau dann differenzierbar, wenn es ein $\phi:I\to\mb{C}$ gibt so dass
  \begin{itemize}
    \item $\phi$ ist stetig in $x_0$
    \item $f(x)-f(x_0)=\phi(x)(x-x_0)$
  \end{itemize}
Ausserdem, $\phi (x_0)= f'(x_0)$.
\end{Sat}
\begin{proof}[Beweis]
  $\exists\phi$ $\implies$ die Differenzierbarkeit von $f$.
  \[\phi(x_0)\stackrel{\mbox{(Stetigkeit)}}{=} \lim_{x\to x_0} \phi (x) =\lim_{x\to x_0}\frac{f(x)-f(x_0)}{x-x_0}\]
  \[=\Limo{h}\frac{f(x_0+h)-f(x_0)}{h}=f'(x_0)\, .\]
  Die Differenziberkaeit $\implies$ $\exists \phi$. Wir setzen:
    \[\phi (x)=\begin{cases}
      f'(x_0)&\qquad \mbox{falls } x=x_0\\
      \frac{f(x)-f(x_0)}{x-x_0}&\qquad \mbox{falls } x\neq x_0
    \end{cases}\, .\]
$\phi$ erfüllt die Bedingungen.
\end{proof}
\begin{Bsp}
  $f(x)=x^n$
  \[f'(x_0)=\Limo{h}\frac{(x_0+h)^n-x_0^n}{h}\]
\[=\Limo{h}\frac{\left\{ \left( x_0^n+\binom{n}{1}x_0^{n-1}+\binom{n}{2}x_0^{n-2}h^2+\cdots + h^n \right)-x_0^n \right\}}{h}\]
  \[\Limo{h}\left[ nx_0^{n-1}+\left\{ \binom{n}{2}x_0^{n-2}h+\cdots+h^{n-1} \right\} \right]=nx_0^{n-1}\]
\end{Bsp}
\begin{Bsp}
  $f(x)=e^x$
  \[f'(x_0)=\Limo{h}\frac{e^{x_0+h}-e^{x_0}}{h}\]
  \[=e^{x_0}\Limo{h}\frac{e^h-1}{h}=e^{x_0}\]
\end{Bsp}
\begin{Ueb}
  $f(x)=a^x$
  \[f'(x_0)=\ln(a) a^x\]
\end{Ueb}
\begin{Bsp}
  $f(x)=\ln x$
  \[f'(x_0)=\frac{\ln(x_0+h)-\ln(x_0)}{h}\]
  \[=\frac{\ln\left( \frac{x_0+h}{x_0} \right)}{h}\]
  \[=\left( \Limo{h}\frac{\ln\left( 1+\frac{h}{x_0} \right)}{\frac{h}{x_0}} \right)\frac{1}{x_0}
=\frac{1}{x_0}\]
\end{Bsp}
\begin{Bem}
  Falls $f$ in $x_0$ differenzierbar ist, dann ist $f$ auch stetig in $x_0$.
  \[f \mbox{ stetig im } x_0\iff\lim_{x\to x_0}f(x)=f(x_0)\iff\lim_{x\to x_0}\left( f(x)-f(x_0) \right)=0\]
  \[\Leftarrow\lim_{x\to x_0}\left( \frac{f(x)-f(x_0)}{x-x_0} \right)(x-x_0)=f'(x_0)
\cdot  0=0\]
\end{Bem}
\begin{Bem}
  Umgekehrt falsch. Sei $f(x)=\sqrt[n]{\abs{x}}$.
  F\"ur $n\geq 2$:
  \[\lim_{x\to 0}\frac{f(x)-f(0)}{x-0}=+\infty\]
  $n=1$:
  \[\lim_{x\to 0}\frac{f(x)-f(0)}{x-0}=\pm1\, .\]
$f$ ist nicht differenzierbar im $0$ 
(aber im $x_0\neq 0$ ist $\sqrt[n]{\abs{x}}$ differenzierbar).
\end{Bem}
\subsection{Rechenregeln}
\begin{Sat}
  Seien $f,g:I\to\mb{C}$ differenzierbar in $x_0$.
  \begin{itemize}
    \item $f+g$ ist auch differenzierbar in $x_0$: 
      \[(f+g)'(x_0)=f'(x_0)+g'(x_0)\]
    \item $fg$ ist auch differenzierbar in $x_0$: 
      \[(fg)'(x_0)=f'(x_0)g(x_0)+f(x)g'(x_0)\]
    \item $\frac{f}{g}$ ist in der Nähe von $x_0$ wohldefiniert wenn $g(x_0)\neq 0$. Ausserdem ist $\frac{f}{g}$ dort differenzierbar.
      \[\left( \frac{f}{g} \right)(x_0)=\frac{f'(x_0)g(x_0)-f(x_0)g'(x_0)}{g(x_0)^2}<\]
  \end{itemize}
\end{Sat}
\begin{proof}[Beweis]
  \begin{itemize}
    \item
      \[\Limo{h}\frac{(f+g)(x_0+h)-(f+g)(x_0)}{h}=\Limo{h}\left\{ \overbrace{\frac{f(x_0+h)-f(x_0)}{h}}^{f'(x_0)}+\overbrace{\frac{g(x_0+h)-g(x_0)}{h}}^{g'(x_0)}\right\}\]
    \item
      \[\Limo{h}\frac{(fg)(x_0+h)-(fg)(x_0)}{h}\]
      \[=\Limo{h}\frac{f(x_0+h)g(x_0+h)-f(x_0+h)g(x_0)+f(x_0+h)g(x_0)-f(x_0)g(x_0)}{h}\]
      \[=\Limo{h}\left\{ f(x_0+h)\frac{g(x_0+h)-g(x_0)}{h}+g(x_0)\frac{f(x_0+h)-f(x_0)}{h} \right\}\]
      \[=f(x_0)g'(x_0)+g(x_0)f'(x_0)\]
(NB: Wir haben benutzt dass $f$ und $g$ stetig im $x_0$ sind).
    \item
      \[\Limo{h}\frac{\frac{f(x_0+h)}{g(x_0+h)}-\frac{f(x_0)}{g(x_0)}}{h}=\Limo{h}\frac{f(x_0+h)g(x_0)-f(x_0)g(x_0+h)}{\left[ g(x_0)g(x_0+h) \right] h}\]
      \[=\Limo{h}\frac{1}{g(x_0)g(x_0+h)}\left\{ \frac{f(x_0+h)g(x_0)-f (x_0+h) g(x_0+h)}{h}+\frac{f(x_0+h)g(x_0+h)-f(x_0)g(x_0+h)}{h} \right\}\]
      \[=\Limo{h}\frac{1}{g(x_0)g(x_0+h)}\left\{ f(x_0+h)\left[ -\frac{g(x_0+h)-g(x_0)}{h}\right] +g(x_0+h)\frac{f(x_0+h)-f(x_0)}{h} \right\}\]
 \[ = \frac{1}{g(x_0)^2}\left\{ f (x_0) (-g'(x_0)) + g(x_0) f' (x_0)\right\}\] 
\end{itemize}
\end{proof}
\begin{Sat}[Kettenregel]\label{s:Ketten}
  Seien $I\stackrel{f}{\to}J\stackrel{g}{\to}\mb{C}$ (mit $I, J\subset\mb{R}$ Intervalle), 
differenzierbar an der Stelle $x_0$ und $f(x_0)$. Dann ist $g\circ f$ an der Stelle $x_0$ 
differenzierbar und
  \[(g\circ f)'(x_0)=g'(f(x_0))f'(x_0)\]
\end{Sat}
\begin{proof}[Beweis] Die Kernidee w\"are:
  \[\Limo{h}\frac{g\circ f(x_0+h)-g\circ f(x_0)}{x-x_0}\]
  \[=\Limo{h}\frac{g(f(x_0+h))-g(f(x_0))}{x-x_0}\]
  \[=\Limo{h}\frac{g (\overbrace{f(x_0+h)}^y)-g ( \overbrace{f(x_0))}^{y_0})}{\underbrace{f(x_0+h)}_y-\underbrace{f(x_0)}_{y_0}}\frac{f(x_0+h)-f(x_0)}{x-x_0}\]
  \[=g'(y_0)f'(x_0)=g'(f(x_0))f'(x_0)\]
Diese Gleichungen sind aber kein Beweis weil $y-y_0$ kann null werden. Lösung:
  \[f(x)-f(x_0)=\phi(x)(x-x_0)\]
  \[g(x)-g(x_0)=\gamma(x)(x-x_0)\]
  mit:
\begin{itemize}
 \item  $\phi$ stetig in $x_0$ und $\phi(x_0)=f'(x_0)$;
\item $\gamma$ stetig in $y_0$ mit $\gamma'(y_0)=g'(y_0)$.
\end{itemize}
Deswegen:
\[g(f(x))-g(f(x_0))=\gamma(f(x))(f(x)-f(x_0))=\underbrace{\gamma(f(x))\phi(x)}_{\Phi(x)}(x-x_0)\]
  $\Phi$ ist stetig an der Stelle $x_0$. $\implies$ $g\circ f$ ist differenzierbar in $x_0$.
Ausserdem,
  \[(g\circ f)'(x_0)=\Phi(x_0)=\gamma(f(x_0))\phi(x_0)=g'(f(x_0))f'(x_0)\, .\]
\end{proof}
\begin{Bsp}
  \[e^{it}=\cos t+i\sin t\]
  \[(\cos x)'=\left(\frac{e^{ix}+e^{-ix}}{2}\right)=
\frac{1}{2}\left( (e^{ix})'+(e^{ix})' \right)=\frac{i}{2} e^{ix}+\frac{i}{2}e^{-ix}\]
\[=-\frac{1}{2i}(e^{ix}-e^{-ix})=-\sin x\]
Analog
  \[(\sin x)'=\left(\frac{e^{ix}-e^{-ix}}{2i}\right)'=\cdots=\cos x\, .\]
NB: Die Identit\"at $(e^{ix})' = i e^{ix}$ ist nicht eine Folgerung des Satzes
\ref{s:Ketten} sein, weil die Werte der Funktion $f: x\mapsto ix$ sind nicht
in $\mb{R}$ enthalten (und im Satz \ref{s:Ketten} gibt es die Annahme
$f:I\to J\subset \mb{R}$). Es gibt tats\"achlich eine Erweiterung des Satzes 
\ref{s:Ketten} die auch den Fall $(e^{ix})'$ enth\"alt (siehe die Theorie
der holomorphen Funktionen). In unserem Fall folgt die Identit\"at 
$(e^{ix})' = i e^{ix}$ aus der Wachstum Identit\"at der Exponentialabbildung
(siehe (WT) im Satz \ref{s:Exp}):
\[
 \lim_{h\to 0} \frac{e^{i(x+h)}- e^{ix}}{h} 
= e^{ix} \lim_{h\to 0} \frac{e^{ih} - 1}{h}
= e^{ix} i \lim_{h\to 0} \frac{e^{ih} -1}{ih} = e^{ix}\, .
\]
\end{Bsp}
\begin{Bsp}
\[\tan'=\left( \frac{\sin}{\cos} \right)'=
\frac{\sin'\cos-\sin\cos'}{\cos^2}=\frac{\sin^2+\cos^2}{\cos^2}=\frac{1}{\cos^2}\]
\end{Bsp}

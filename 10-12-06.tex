\subsection{Der Hauptsatz der Differential- und Integralrechnung}
Das Integral ist eine Art ``Umkehrung'' der Ableitung!

\begin{Def} Wir setzen:
\[
\int_a^b f := - \int_b^a f \qquad \mbox{falls $b<a$}
\]
\[
\int_a^b f := 0 \qquad \mbox{falls $a=b$}
 \]
\end{Def}
\begin{theorem}[Hauptsatz der Differential- und Integralrechnung]
Sei $f:I\to\mb{R}$ eine stetige Funktion und $a\in I$ eine beliebige Stelle.
Wir definieren $F(x):=\int_a^xf(y)\md y$. Dann ist $F$ differenzierbar und 
$F'(x)=f(x)$ $\forall x\in I$ (d.h. $F$ ist \ul{eine} {\em Stammfunktion} von $f$).
\end{theorem}
\begin{Bem}
  $F$ Stammfunktion von $f$ $\implies$ $F+c_0$ ist auch eine Stammfunktion von $f$:
Stammfunktionen sind nicht eindeutig!
\end{Bem}
\begin{Bew} Sei $x\in I$. Wir wollen zeigen dass
\begin{equation}\label{e:goal}
f(x) = \Limo{h}\frac{F(x+h)-F(x)}{h}\, .
\end{equation}
Falls $h>0$,
  \[\frac{F(x+h)-F(x)}{h}=\frac{1}{h}\left( \int_a^{x+h}f(y)\md y-\int_a^xf(y)\md y \right) =\frac{1}{h}\int_x^{x+h}f(y)\md y\, .\]
 Ausserdem,
  \[
\frac{F(x+h)-F(x)}{h}-f(x) = \frac{1}{h} \int_x^{x+h} f(y) \md y 
- \frac{1}{h} hf(x)
\]
\[
= \frac{1}{h} \int_x^{x+h} f(y)\md y - \frac{1}{h} \int_x^{x+h} f(x)\md y
= \frac{1}{h} \int_x^{x+h} (f(y)-f(x))\md y\, .
\]
Sei $\varepsilon>0$: $\exists \delta>0$ so dass 
$\abs{y-x}<\delta\implies\abs{f(y)-f(x)}<\varepsilon$. Für $h<\delta$:
\[\left|\frac{1}{h}\int_x^{x+h} (f(y)-f(x))\md y\right|
\leq \frac{1}{h}\int_x^{x+h}\abs{f(y)-f(x)}\md y\leq 
\frac{1}{h}\int_x^{x+h}\varepsilon\md y = \eps\]
\[\implies \lim_{h\downarrow 0}\frac{1}{h}\int_x^{x+h} (f(y)-f(x))\md y=0 \implies \eqref{e:goal}\]
Im Fall $h<0$, wir setzen $h=-k$ ($k>0$) und schliessen
  \[\frac{1}{h}(F(x+h)-F(x))=\frac{1}{-k}(F(x-k)-F(x))=\frac{F(x)-F(x-k)}{k}\]
  \[=\frac{1}{k}\left\{ \int_a^xf(y)\md y-\int_a^{x-k}f(y)\md y \right\}=\frac{1}{k}\int_{x-k}^xf(y)\md y\]
  Gleiche Idee wie oben
  \[\implies \lim_{k\downarrow 0}\frac{1}{k}\int_{x-k}^xf(y)\md y=f(x)\]
\end{Bew}
\begin{Bem}
  Sei $f$ eine Funktion $f:[a,b]\to\mb{R}$. Seien $F,G:[a,b]\to \mb{R}$ zwei Stammfunktionen von $f$.
  \[(F-G)'=F'-G'=f-f=0\]
  \[\implies F-G=\text{konstant}\]
\end{Bem}
\begin{Kor}
  Sei $f:[a,b]\to\mb{R}$ eine stetige Funktion. Se $G:[a,b]\to\mb{R}$ eine Stammfunktion. Dann:
  \[\int_a^bf(x)\md x=G(b)-G(a)\left( =:G|_a^b \right)\]
\end{Kor}
\begin{Bew} $F(x)=\int_a^xf(y)\md y$, $x>a$. 
\[F'(x)=f(x)\s\forall x>a\]
(NB: Der Beweis oben impliziert die Differenzierbarkeit auch an der Stelle $a$, 
wobei (wegen unserer Konvention) $F(a)=0$. Ausserdem, die Stetigkeit an dieser Stelle
kann man wir folgt sehen  
  \[\abs{\int_a^xf(y)\md y}\leq \int_a^x\abs{f(y)}\md y\leq M(x-a)\]
  Deswegen $F(a):=0$. ( $\int_a^af(x)\md x:= 0$ und $\int_a^xf(y)\md y=-\int_x^af(y)\md y$ )
\paragraph{Zusammenfassung}
\begin{itemize}
  \item $F-G$ ist differenzierbar auf $[a,b]$
  \item $(F-G)'=f-f=0$
\end{itemize}
\[\implies F(x)=G(x)+c\]
\[\implies \int_a^bf(y)\md y=F(b)-F(a)=(F(b)-c)-(F(a)-c)=G(b)-G(a)\]
\end{Bew}
\begin{Bsp}
  $f(x)=x^2$. $A:= \{(x,y): |x|leq 1, x^2\leq y\leq 1\}$ 
und $B:= \{(x,y): |x|leq 1, 0\leq y \leq x^2\}$.
  Inhalt von $A$ = $2-\underbrace{\text{Inhalt von }B}_{ \abs{ B } }$
  \[f(x)=x^2\]
  \[G(x)=\frac{x^3}{3}\]
  \[G'(x)=x^2=f(x)\]
  \[\abs{B}=\int_{-1}^1f(x)\md x=\left.\frac{x^3}{3}\right|_{-1}^1=\frac{1}{3}-\left( -\frac{1}{3} \right)=\frac{2}{3}\]
\end{Bsp}
\begin{Bsp}\label{b:Kreis} Wir wollen den Inhalt des Kreis $K$ mit Mittelpunkt $(0,0)$ und Radius 1 
rechnen. Sei $f(x)=\sqrt{1-x^2}$ und $A:= \{(x,y): 0\leq x\leq 1, 0\leq y\leq f(x)$. Dann
\begin{equation}\label{e:Kreis}
|K| = 4 |A|= 4 \int_0^1\sqrt{1-x^2}\md x
 \end{equation}
Das Ingeral in \eqref{e:Kreis} ist nicht so einfach zu bestimmen. Wir bemerken
dass die Ableitung des Arcsinus ``fast'' $f$ ist: 
  \[\arcsin'(x)=\frac{1}{\sqrt{1-x^2}}\, .\]
In der naechsten Kapitel werden wir diese Bemerkung nuzten um das Integral
in \eqref{e:Kreis} zu bestimmen. Daf\"ur brauchen wir eine wichtige Methode
der Integrationsrechnung.
\end{Bsp}
\subsection{Integrationsmethoden}
\begin{itemize}
  \item Partielle Integration
  \item Substitutionsregel
\end{itemize}
\begin{Sat}\label{s:PI}[Partielle Integration]
  Seien $f,g:[a,b]\to\mb{R}$ stetig und $F,G$ entsprechende Stammfunktionen.
  \begin{equation}\label{e:1012061}
    \int_a^bF(x)g(x)\md x=FG|_a^b-\int_a^bf(x)G(x)\md x
  \end{equation}
\end{Sat}
\begin{Bew}
  \begin{equation}\label{e:1012062}
    \eqref{e:1012061}\iff\int_a^b(\underbrace{F(x)g(x)+f(x)G(x)}_{h(x)})\md x = FG|_a^b
  \end{equation}
  \[(FG)(x)=F(x)G'(x)+F'(x)G(x)=F(x)g(x) + f(x)G(x)\]
  $FG(x)$ ist eine Stammfunktion von $h$. $h$ ist stetig!
  Hauptsatz der Differential- und Integralrechnung $\implies$ \eqref{e:1012062}
\end{Bew}
\begin{Bsp} Wir rechnen nun das Integral in \eqref{e:Kreis}.
  \[\int_0^1\sqrt{1-x^2}\md x=\int_0^1\underbrace{\sqrt{1-x^2}}_{F(x)}\underbrace{g(x)}_1\md x\]
  \[F(x)=\sqrt{1-x^2}\]
  \[f(x)=F'(x)=\frac{-\not 2 x}{\not 2\sqrt{1-x^2}}=-\frac{x}{\sqrt{1-x^2}}\]
  \[g(x)=1\]
  \[G(x)=x\]
Leider, k\"onnten wir den Satz \ref{s:PI} nicht direkt anwenden, weil
$f(x)$ nicht definiert in $x=1$ ist (in der Tat $\lim_{x\to 1} f(x) = -\infty$ und
deswegen besitzt $f$ keine stetige Fortsetzung auf $[0,1]$).

Aber:
  \[\int_0^1 \sqrt{1-x^2} \md x = \lim_{\varepsilon\downarrow 0}\int_0^{1-\varepsilon}\sqrt{1-x^2}\md x\]
  \[\stackrel{Satz \ref{s:PI}}{=} \lim_{\varepsilon\downarrow 0}\left[ \left.\sqrt{1-x^2}x\right|_0^{1-\varepsilon}+\int_0^{1-\varepsilon}\frac{x^2}{\sqrt{1-x^2}}\md x \right]\]
  \[=\underbrace{\left.\sqrt{1-x^2}x\right|_0^1}_{=0} + \lim_{\varepsilon\downarrow 0}\int_0^{1-\varepsilon}\frac{x^2}{\sqrt{1-x^2}}\md x\]
  \[= \lim_{\varepsilon\downarrow 0}\int_0^{1-\varepsilon}\left( \left( \frac{x^2-1}{\sqrt{x^2-1}}+\frac{1}{\sqrt{x^2-1}} \right) \right)\md x\]
  \[= \lim_{\varepsilon\downarrow 0}\int_0^{1-\varepsilon}-\sqrt{1-x^2}\md x+\lim_{\varepsilon\downarrow 0}\int_0^{1-\varepsilon}\frac{\md x}{\sqrt{1-x^2}}\]
  \[=-\int_0^1\sqrt{1-x^2}+\lim_{\varepsilon\downarrow 0}\arcsin\Big|_0^{1-\varepsilon}\]
  \[=-\int_0^1\sqrt{1-x^2}+ \arcsin\Big|_0^{1}\]
  \[=-\int_0^1\sqrt{1-x^2}+\left( \frac{\pi}{2}-0 \right)\]
  \[=-\int_0^1\sqrt{1-x^2}+\frac{\pi}{2}\]
Wir schliessen
\[\int_0^1\sqrt{1-x^2}\md x=-\int_0^1\sqrt{1-x^2}\md x+\frac{\pi}{2}\]
\[\implies  2\int_0^1\sqrt{1-x^2}\md x=\frac{\pi}{2}\]
Deswegen, der Inhalt des Kreis mit Mittelpunkt $(0,0)$ und Radius $1$ ist $\pi$
(siehe Beispiel \ref{b:Kreis})!
\end{Bsp}
\begin{Sat}[Substitutionsregel]
  Seien $f:[a,b]\to\mb{R}$ und $g:\underbrace{f([a,b])}_{[m,M]}\to\mb{R}$ zwei stetige Funktionen ($m=\min_{[a,b]}f$, $M=\max_{[a,b]}f$). Falls $f$ differenzierbar ist mit $f'$ stetig, dann
  \[\int_a^bg(f(x))f'(x)\md x=\int_{f(a)}^{f(b)}g(y)\md y\]
\end{Sat}
\begin{Bew}
  Sei $G$ eine Stammfunktion von $g$. (Später: warum gibt es eine solche Stammfunktion?)
  \[\int_a^bG'(f(x)f'(x)\md x=\int_a^b(G(f(x)))'\md x=G\circ f\Big|_a^b\]
  \[\implies \int_A^bg(f(x))f'(x)\md x=G(f(b))-G(f(a))\]
  \[=G\Big|_{f(a)}^{f(b)}=\int_{f(a)}^{f(b)}g(y)\md y\]

  Zur Existenz der Stammfunktion. Diese wird vom Hauptsatz der Differential- und Integralrechnung
garantiert! In der Tat ist $G(x):= \int_m^x g(y) dy$ eine Stammfunktion von $g$.
\end{Bew}
\begin{Bem} Der Hauptsatz der Differential- und Integralrechnung ent\"ahlt das erste
Beispiel eines Existenzsatzes f\"ur Differentialgleichungen, d.h.
die Existenz der Lösungen dieser Differentialgleichung:
  \[\underbrace{F'}_{\text{Die Unbekannte}}=\underbrace{f}_{\text{bekannt}}\]
\end{Bem}
\subsection{Uneigentliche Integrale}
\begin{Def}
  Sei $I=]a,b[$ (wobei $-\infty\leq a<b\leq +\infty$; z.B. $I=]-\infty,\infty[=\mb{R}$ ist eine
M\"oglichkeit). Sei $f:I\to\mb{R}$ so dass
  \[\forall a<\alpha<\beta<b\s f|_{[\alpha,\beta]}\s\text{eine Regelfunktion ist}\]
  Falls $c\in I$ und
  \[\lim_{\beta\uparrow b}\int_c^\beta\in\mb{R}\qquad \mbox{und}\qquad
\lim_{\alpha\downarrow \alpha}\int_c^\alpha\in\mb{R}\]
  dann definieren wir
  \[\int_a^bf:=\lim_{\beta\uparrow b}\int_c^\beta f+\lim_{a\downarrow 0}\int_\alpha^cf\]
\end{Def}
\begin{Bem} $\int_a^b f$ h\"angt nicht von $c$ ab! In der Tat, sei $\tilde{c}$ eine andere
 Stelle in $I$. Dann:
\[\lim_{\beta\uparrow b}\int_{\tilde c}^bf
=\lim_{\beta\uparrow b}\left( \int_c^\beta f-\int_c^{\tilde c}f \right)
=\lim_{\beta\uparrow b}\int_c^\beta f-\int_c^{\tilde c}f\]
und analog  
\[\lim_{\alpha\downarrow a}\int_\alpha^{\tilde c}f=
\lim_{\alpha\downarrow a}\int_\alpha^c f+\int_c^{\tilde c}f\]
Deswegen
\[
\lim_{\beta\uparrow b}\int_{\tilde c}^bf +  \lim_{\alpha\downarrow a}\int_\alpha^{\tilde c}f
 =\lim_{\beta\uparrow b}\int_c^\beta f-\int_c^{\tilde c}f + \lim_{\alpha\downarrow a}\int_\alpha^c f+\int_c^{\tilde c}f
\]
\[
 =\lim_{\beta\uparrow b}\int_c^\beta f+\lim_{a\downarrow 0}\int_\alpha^cf
\]


\end{Bem}
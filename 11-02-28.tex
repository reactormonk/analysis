\subsection{Stetigkeit}
\begin{Def}
  Sei $f:\Omega_{\subset\mb{R}^n}\to\mb{R}^k$. $f$ ist stetig an der Stelle $x\in\Omega$ falls $\forall \left\{ x_k \right\}\subset\Omega$ mit $x_k\to x$.
  \[\Limi{k} f(x_k)=f(x)\]
\end{Def}
\begin{Lem}
  \label{l:1102282}
  Eine equivalente Definition:
  \[\forall \varepsilon>0\s\exists \delta>0: f\left( K_\delta(x)\cap \Omega \right)\subset K_\varepsilon(f(x))\]
\end{Lem}
\begin{Bew}
  $\varepsilon$-$\delta$ $\implies$ Folgendefinition. Sei $x_k\to x$. Ziel: $f(x_k)\to f(x)$
  \[\forall \varepsilon>0\s\exists \s\text{mit}\s \underbrace{\Norm{f(x_k)-f(x)}}_{\underbrace{d(f(x_k),f(x))}_{f(x_k)\in K_\varepsilon(f(x))}}<\varepsilon\s\forall k\geq N\]
  \[\exists \delta>0\s \underbrace{f(K_\delta(x))\subset K_\varepsilon(f(x))}\]
  \[\exists\s \Norm{x_k-x}<\delta\s k\geq N\]
  \[x_k \in K_\delta(x)\implies f(x_k)\in K_\varepsilon(f(x))\]
  Folgendefinition $\implies$ ($\varepsilon$-$\delta$)-Defintion. Widerspruchsannahme: 
  \[\exists \varepsilon>0: f(K_\delta(x)\cap \Omega)\not\subset K_\varepsilon(f(x))\s\forall \delta>0\]
  \[\implies\forall \delta>0\s\exists y_\delta\in K_\delta(x)\s\text{und}\s \Norm{f(y_\delta)-f(x)}\geq \varepsilon\]
  Nehmen wir $\delta=\frac{1}{j}$ und $x_j=\frac{y_1}{j}$
  \[\Norm{x_j-x}<\frac{1}{j}\s(\text{weil}\s x_j\in K_{\frac{1}{j}}(x))\]
  \[\Norm{f(x_j)-f(x)}=\Norm{f(y_{\frac{1}{j}}-f(x))}\geq\varepsilon\]
  $x_j\to x$ aber $f(x_j)\not\to f(x)$
\end{Bew}
\begin{Def}
  Die allgemeine Defintion der Stetigkeit für metrische Räume: Seien $(X,d)$ und $(Y,\ol d)$ zwei metrische Räume. Sei $f:X\to Y$. $f$ ist stetig an der Stelle $x$ falls:
  \[\forall \varepsilon>0\s\exists \delta>0\s\text{mit}\s d(y,x)<\delta\implies d(f(y),f(x))<\varepsilon\]
  \[\iff f(K\delta(x))\subset K_\varepsilon(f(x))\]
\end{Def}
\begin{Def}
  Eine $f:X\to Y$ heisst stetig falls $f$ stetig an jeder Stelle $x\in X$ ist.
\end{Def}
\begin{Sat}
  Sei $f:X\to Y$ ($(X,d), (Y\ol d)$ metrische Räume) Dann:
  \begin{enumerate}
    \item Die Stetigkeit in $x$ $\iff$ $\forall$ Umgebung $U$ von $f(x)$ ist $f^{-1}(U)$ eine Umgebung von $x$.
    \item Stetigkeit von $f$ $\iff$ $f^{-1}(U)$ ist offen $\forall U$ offen.
  \end{enumerate}
\end{Sat}
\begin{Bew}
  \begin{enumerate}
    \item
      \begin{itemize}
        \item Stetigkeit $\implies$ Umgebung.
          $U$ Umgebung von $f(x)\implies \exists \delta>0$ mit $K_\delta(f(x))\subset U$
          \[\implies \exists \varepsilon>0:f(K_\varepsilon(x))\subset K_\delta(f(x))\]
          \[\implies f^{-1}(U)\supset f^{-1}(K_\delta(f(x)))\supset K_\varepsilon(x)\implies f^{-1}(U)\s\text{Umgebung von}\s U\]
        \item Umgebung $\implies$ Stetigkeit. Sei $\delta>0$ $U=K_\delta(f(x))$. $U$ Umgebung von $f(x)$. $f^{-1}(U)$ ist eine Umgebung von $x$.
          \[\implies \exists\varepsilon>0:K_\varepsilon(X)\subset f^{-1}(U)\]
          \[\implies f(K_\varepsilon(x))\subset U=K_\delta(f(x))\]
      \end{itemize}
    \item
      \begin{itemize}
        \item 
          Stetigkeit $\implies$ offen. Sei $U$ offen $\iff$ $\forall y\in U$ ist $U$ eine Umgebung von $y$
          \[f^{-1}U\ni x\implies f(x)\in U\stackrel{\text{Stetigkeit in}\s x}{\implies} f^{-1}(U)\s\text{ist eine Umgebung von}\s x\]
          \[\implies f^{-1}(U)\s\text{ist offen}\]
        \item offen $\implies$ Stetigkeit an jedem $x\in X$. Sei $x\in X$, $\delta>0$, $K_\delta(f(x))$ ist offen
          \[f^{-1}(K_\delta(f(x)))\s\text{ist offen}\implies x\in f^{-1}(K_\delta(f(x)))\]
          \[\implies \exists \varepsilon>0: K_\varepsilon(x)\subset f^{-1}(K_\delta(f(x)))\]
          \[f(K_\varepsilon(x))\subset K_\delta(f(x))\]
      \end{itemize}
  \end{enumerate}
\end{Bew}
\subsection{lineare Abbildungen}
\begin{Def}
  Eine Abbildung $L:V\to W$ ($V$, $W$ Vektoren) heisst linear, falls
  \[L(\lambda_1v_1+\lambda_2v_2)=\lambda_1L(v_1)+\lambda_2L(v_2)\s\forall v_1,v_2\in V,\s\forall \lambda_1,\lambda_2\in\mb{R}\]
  \[L:\mb{R}^n\to\mb{R}^k\iff \exists\s\text{eine Matrix}\s L_{ij}:\]
  \[L(x)=\left( \sum^n_{j=1}L_{1j}x_j,\sum^n_{j=1}L_{2j}x_j,\cdots,\sum^n_{j=1}L_{kj}x_j \right)\]
\end{Def}
\begin{Def}
  Sei $L_{ij}$ eine Matrix die die lineare Abbildung $L:\mb{R}^n\to\mb{R}^k$ darstellt. Die Hilbert-Schmidt Norm von $L$ ist
  \[\Norm{L}_{\text{HS}}=\sqrt{\sum^k_{i=1}\sum^n_{j=1}L_{ij}^2}\]
\end{Def}
\begin{Bem}
  $\left\{ L:(L_{ij} n\times k\s\text{Matrixen} \right\}\sim \mb{R}^{nk}$ $\Norm{.}_{\text{HS}}$ ist die euklidische Norm.
\end{Bem}
\begin{Bem}
  Sei $L:\mb{R}^n\to\mb{R}^k$ eine lineare Abbildung und $x\in\mb{R}^n$. Dann $\Norm{L(x)}\leq\Norm{x}\Norm{L}_\text{HS}$.
\end{Bem}
\begin{Kor}
  Sei $L$ wie oben, dann ist $L$ stetig.
\end{Kor}
\begin{Bew}
  Sei $x_k\to x$. Ziel $L(x_k)\to L(x)$
  \[\Norm{L(x_k)-L(x)}=\Norm{L(x_k-x)}\leq\Norm{x_k-x}\Norm{L}_\text{HS}\to 0\]
  \[\implies \Norm{L(x_k)-L(x)}\to 0\]
  \[\implies \text{Stetigkeit}\]
\end{Bew}
\begin{Bew}
  Beweis von \ref{l:1102282}: $L(x)=y$
  \[\Norm{L(x)}^2=\sum^k_{i=1}y_i^2\]
  \[=\sum^k_{i=1}\left( \sum^n_{j=1}L_{ij}x_j \right)^2\stackrel{\text{Cauchy-Schwartz}}{\leq}\sum^k_{i=1}\left( \sum^n_{j=1}L_{ij}^2 \right)\left( \sum_{j=1}x_j \right)^2\]
  \[=\sum^k_{i=1}\sum^n_{j=1}L_{ij}^2\Norm{x}^2=\Norm{x}^2\left( \sum^k_{i=1}\sum^n_{j=1}L_{ij}^2 \right)\]
  \[\Norm{x}^2\Norm{L}^2_\text{HS}\implies \Norm{L(x)}\leq\Norm{x}\Norm{L}_\text{HS}\]
\end{Bew}
\begin{Def}
  Sei $L:V\to W$ eine lineare Abbildung wobei $(V,\Norm{.}_V)$ und $(W,\Norm{.}_W)$ zwei endlich-dimensionierte Vektorräume sind. Die Operatornorm von $L$ ist:
  \[\Norm{L}_{L(V,W)}:=\sup_{\Norm{v}_V\leq 1}\Norm{L(v)}_W\]
\end{Def}
\begin{Sat}
  $\Norm{.}_{L(V,W)}$ ist eine Norm und
  \[\Norm{L(v)}_W\leq \Norm{L}_{L(V,W)}\Norm{v}_V\]
  Deswegen: jede lineare Abbildung $L:V\to W$ ist stetig.
\end{Sat}
\begin{Bew}
  Der Kern ist die folgende Eigenschaft:
  \[\Norm{L}_{L(V,W)}<+\infty\]
  Wenn das gilt dann:
  \begin{enumerate}
    \item 
      \[\underbrace{\Norm{L}_{L(V,W)}}_\text{Kern}\s\text{und}\s\Norm{L}_{L(V,W)}=0\iff L=0\]
      $\Leftarrow$ einfach. Sei $\Norm{L}_{L(V,W)}=0$. Dann sei $v\in V$.
      \[v=0\implies L(v)=0\]
      \[v\neq 0\s z\frac{v}{\Norm{v_V}}\implies \Norm{z}_V=1\]
      \[\Norm{L(z)}_W\leq\sup_{\Norm{y}_V\leq 1}\Norm{L(v)}_W=0\]
      \[\implies L(z)=0\implies L(v)=L\left( \Norm{v}_V z \right)=\Norm{v}_VL(z)=0\]
    \item
      \[\Norm{\lambda L}_{L(V,W)}=\abs{\lambda}\Norm{L}_{L(V,W)}\]
      \[\Norm{\lambda L}_{L(V,W)}=\sup_{\Norm{y}_V\leq 1}\Norm{\lambda L(v)}_W\]
      \[=\sup_{\Norm{y}_V\leq 1}\abs{\lambda}\Norm{L(v)}_W\]
      \[=\abs{\lambda}\sup_{\Norm{y}_V\leq 1}\Norm{L(v)}_W\]
      \[=\abs{\lambda}\Norm{L}_{L(V,W)}\]
    \item
      \[\Norm{L+L'}_{L(V,W)}\]
      \[=\sup_{\Norm{y}_V\leq 1}\Norm{(L+L')(v)}_{L(V,W)}\]
      \[=\sup_{\Norm{y}_V\leq 1}\Norm{L(v)+L'(v)}_{L(V,W)}\]
      \[\leq\sup_{\Norm{y}_V\leq 1}\left( \Norm{L(v)}_W+\Norm{L'(v)}_W\right)\]
      \[\leq\sup_{\Norm{y}_V\leq 1}\Norm{L(v)}_W+\sup_{\Norm{y}_V\leq 1}\Norm{L'(v)}_W\]
      \[=\Norm{L}_{L(V,W)}+\Norm{L'}_{L(V,W)}\]
      Wenn $v_1,\cdots,v_n$ Basis für $V$, $w_1,\cdots,w_k$ Basis für $W$. Die lineare Abbildung $E_{ij}(v_i)=w_j$, $E_{ij}(v_l)=0$ falls $l\neq i$ ist eine Basis für $L(V,W)\implies L=\sum_{i,j}\lambda_{ij}E_{ij}$
  \end{enumerate}
\end{Bew}

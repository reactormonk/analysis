\begin{Sat}[De L'Hospitalsche Regel]
  $f,g:]a,b[\to\mb{R}$ überall differenzierbar und mit $g(x), 
g'(x)\neq 0$ $\forall x\in ]a,b[$. In jeder dieser Situationen:
  \begin{enumerate}
    \item $f(x)\to 0, g(x)\to 0$ für $x\downarrow a$
    \item $f(x), g(x) \to +\infty$ (bzw. $-\infty$) für $x\downarrow a$
  \end{enumerate}
  Falls $\lim_{x\downarrow a}\frac{f'(x)}{g'(x)}$ existiert (oder $\pm \infty$ ist), dann
  \[\lim_{x\downarrow a}\frac{f(x)}{g(x)}=\lim_{x\downarrow a}\frac{f'(x)}{g'(x)}\]
Die entsprechenden Aussagen gelten auch f\"ur Grenzprozesse mit $x\uparrow b$ und
$x\to \pm \infty$.
\end{Sat}
Eine grobe Idee wie so dieser Satz gilt: nehmen wir an dass
die Funktionen $f$ und $g$ auch in $a$ definiert und differenzierbar sind, mit
$f(a)=g(a)=0$; dann, wenn $\abs{x-a}$ klein ist,
  \begin{eqnarray*}
    f(x)&=f'(a)(x-a)+R\\
    g(x)&=g'(a)(x-a)+R'
  \end{eqnarray*}
wobei $R$ und $R'$ ziemlich klein im vergleich mit $|x-a|$ sind. Deswegen,
  \[\frac{f(x)}{g(x)}\sim\frac{f'(a)}{g'(a)}\, .\]
Wenn die Ableitungen von $f$ und $g$ stetig w\"aren, dann
\[
 \frac{f(x)}{g(x)}\sim \frac{f'(x)}{g'(x)}.
\]


\begin{Bew}
  \begin{enumerate}
  \item 
      OBdA $f(a)=0$, $g(a)=0$ $\implies$ $f$ und $g$ sind stetig auf $[a,b[$. Verallgemeinerter Mittelwertsatz:
      \[\forall x\in ]a,b[\s\exists \xi\in ]a,x[:\]
      \[\frac{f(x)}{g(x)}=\frac{f(x)-f(a)}{g(x)-g(a)}=\frac{f'(\xi)}{g'(\xi)}\]
      \[x\to a\implies \xi\to a\]
      \[\lim_{x\downarrow a} \frac{f(x)}{g(x)}=\lim_{\xi\downarrow a} \frac{f'(\xi)}{g'(\xi)}\]
    \item Wir nehmen zus\"atzlich an dass 
\[\lim_{\xi\downarrow 0} \frac{f'(\xi)}{g'(\xi)}\in\mb{R}\, .
\]
 Sei $A:=\frac{f'(\xi)}{g'(\xi)}\in\mb{R}$. Wir schätzen $\abs{\frac{f(x)}{g(x)}-A}$ ab für $x$ 
in der Nähe von $a$. F\"ur jede $y<x$ mit $y\in ]a,b[$ schreiben wir
\begin{equation}\label{e:X1}
\frac{f(x)}{g(x)}=\frac{f(x)-f(y)}{g(x)-g(y)}\frac{1-\frac{g(y)}{g(x)}}{1-\frac{f(y)}{f(x)}}
\end{equation}
Sei $\varepsilon$ eine gegebene positive Zahl. Wählen wir ein $\delta>0$ so dass
\begin{equation}\label{e:X2}\left|\frac{f'(\xi)}{f'(\xi)}-A\right|<\varepsilon\qquad\forall 
\xi\in ]a,a+\delta[
\end{equation}
F\"ur jede $a<y<x<a+\delta$, sei $\xi$ die Stelle des Satzes von Cauchy. Dann:     
\[\left|\frac{f(x)}{g(x)}-A\right|\leq \underbrace{\left|
\frac{f(x)}{g(x)}-\frac{f(x)-f(a+\delta)}{g(x)-g(a+\delta)}\right|}_{=B}
+\underbrace{\left|\frac{f(x)-f(a+\delta)}{g(x)-g(a+\delta)}\right|}_{=C}\]
Aus \eqref{e:X2} folgt $C<\varepsilon$. Aus \eqref{e:X1}:
\[B=\left|\frac{f(x)-f(a+\delta)}{g(x)-g(a+\delta}\left( 
\frac{1-\frac{g(a+\delta)}{g(x)}}{1-\frac{f(a+\delta)}{f(x)}} \right)-\frac{f(x)-f(a+\delta)}
{g(x)-f(a+\delta)}\right|\]
      \[=\underbrace{\left|\frac{f(x)-f(a+\delta)}{g(x)-g(a+\delta)}\right|}
_{\leq \abs{A}+\varepsilon}\underbrace{\left|\frac{1-\frac{g(a+\delta)}{g(x)}}
{1-\frac{f(a+\delta)}{f(x)}}-1\right|}_{\to 0\s\text{für}\s x\downarrow a}\]
      $\implies$ $\exists \delta*$ so dass für $\abs{x-a}<\delta*$, $B<\varepsilon$. 
Sei nun $x$ s.d. $x-a<\min\left\{ \delta,\delta* \right\}$. Dann
      \[\left|\frac{f(x)}{g(x)}-A\right|<2\varepsilon\]
  \end{enumerate}
  Um den Beweis zu beenden, es bleibt zu tun:
  \begin{itemize}
    \item $x\downarrow a$, Situation 2., aber
      \[\lim_{x\downarrow a}\frac{f'(x)}{g'(x)}=+\infty (-\infty).\]
Der Beweis ist in diesem Fall ganz \"ahnlich zum obigen Beweis, aber anstatt
\[\left|\frac{f(x)}{g(x)}-A\right|<2\varepsilon \qquad \mbox{f\``ur $x-a$ klein genug}\]
ist das Ziel
\[\frac{f(x)}{g(x)} > M \qquad \mbox{f``ur $x-a$ klein genug}\]
(wobei $M$ eine beliebige reelle Zahl ist).
    \item $x\uparrow b$. Dieser Fall ist trivial.
\item $x\to +\infty$. In diesem Fall, setzen wir
  \[F(y)=f\left( \frac{1}{y} \right) \qquad \mbox{und}\qquad 
G(y)=g\left( \frac{1}{y} \right)\, .\]
Dann
\[\Limi{x}\frac{f(x)}{g(x)}=\lim_{y\downarrow 0 (=:a)}\frac{F(y)}{G(y)}
=\lim_{y\downarrow 0}\frac{F'(y)}{G'(y)}\]
\[=
\lim_{y\downarrow 0} \frac{f'\left(\frac{1}{y}\right)\left( -\frac{1}{y^2} \right)}
{g'\left( \frac{1}{y} \right)\left( -\frac{1}{y^2} \right)}
=\lim_{y\downarrow 0}\frac{f'\left(\frac{1}{y}\right)}{g'\left(
\frac{1}{y}\right)}=\Limi{x}\frac{f'(x)}{g'(x)}\]
\end{itemize}
\end{Bew}
\begin{Bsp}
  \[\Limi{x}\frac{e^x}{x}=\Limi{x}\frac{e^x}{1}=+\infty\]
\end{Bsp}
\begin{Bsp}
  \[\Limi{x}\frac{e^x}{x^n}=\Limi{x}\frac{e^x}{nx^{n-1}}=\Limi{x}\frac{e^x}{n(n-1)x^{n-2}}=\cdots=\Limi{x}\frac{e^x}{n!}=+\infty\]
\end{Bsp}
\begin{Bsp}
  \[\Limo{x}\left( \frac{1}{x}-\frac{1}{\sin x} \right)=\Limo{x}\frac{\sin x-x}{x\sin x}=\Limo{x}\frac{\cos x-1}{\sin x + x\cos x}\]
  \[=\Limo{x}\frac{\overbrace{-\sin x-0}^{\to 0}}{\underbrace{\cos x-x\sin x + \cos x}_{\to 2}}=0\]
\end{Bsp}
\subsection{Differentation einer Potenzreihe}
Aus den Rechenregeln f\"ur die Ableitung wissen wir:
\[P(x)=a_nx^n+a_{n-1}x^{n-1}+\cdots+a_0\]
\[P'(x)=na_nx^{n-1}+(n-1)a_{n-1}x^{n-2}+\cdots+a_1\]
Sei nun $f$ durch eine Potenzreihe definiert, mit einem nichttrivialen Konvergenzradius:
\[f(x)=\sum^\infty_{n=0}a_nx^n\]
K\"onnten wir schliessen dass $f$ auf ihrem Definitionsbereich differenzierbar ist? 
Ausserdem, gilt die Formel
\[f'(x)\stackrel{?}{=}\sum_{n=1}^\infty na_nx^{n-1}\]
\begin{Sat}\label{s:diff}
  Sei $\sum^\infty_{n=0}a_nx^n=f(x)$ eine Potenzreihe mit Konvergenzradius $R$ 
($> 0$, auch $R=+\infty$ m\"oglich). Falls $\abs{x_0}<R$, dann ist $f$ in $x_0$ differenzierbar und
  \[f'(x_0)=\sum^\infty_{n=1}na_nx^{n-1}\]
  (falls $R=+\infty$, $f$ ist überall differenzierbar, auf $\mb{R}$!)
\end{Sat}
\begin{Bem}\label{b:R=R'} Der Satz von Cauchy-Hadamard gibt
  \[R=\frac{1}{\limsup_{n\to\infty}\sqrt[n]{a_n}}\]
Nun, $\sum\infty_{n=1}n a_n x^{n-1}$ konvergiert für $x=0$ und 
für $x\neq 0$ konvergiert genau dann, wenn $\sum^\infty_{n=0} na_nx^n$ konvergiert. 
Der Konvergenzradius ist deswegen:
  \[R'=\frac{1}{\limsup_{n\to\infty}\sqrt[n]{na_n}}
= \frac{1}{\limsup_{n\to\infty} \sqrt[n]{a_n}} =R\, .\]
\end{Bem}

Wir wollen nun noch ein Mal das Lemma \ref{l:Abel} schauen. Dieses Lemma
sagt dass, wenn eine Potenzreihe an einer Stelle $x_0$ konvergiert, dann
konvergiert sie auch an jeder Stelle $x$ mit $|x|< |x_0|$. Aber die Kernidee
des Beweis dieses Lemma hat auch andere Konsequenzen.

\begin{Def}
 Sei $I=[a,b]$ ein abgeschlossenes Intervall und $f: I\to \mb{R}$ eine
stetige Funtion. Dann
\[
\|f\|_{C^0 (I)} \;=\; \max_{x\in I} |f(x)|\, .
\]
\end{Def}

\begin{Def}
Sei $I$ ein abgeschlossenese Intervall und $f_n: I \to \mb{R}$ eine Folge
von Funktionen. Falls $\sum_n f_n (x)$ an jeder Stelle $x\in I$ konvergiert,
koennten wir eine neue Funktion definieren:
\[
I\in x \quad \mapsto \quad f(x) = \sum_{n=0}^\infty f_n (x) \in \mb{R}\, . 
\]
F\"ur diese neue Funktion schreiben wir $f=\sum_n f_n$, d.h. eine
{\em Reihe von Funktionen}.

Falls jede $f_n$ stetig ist und
\[
\sum_{n=0}^\infty \|f_n\|_{C^0 (I)}<\infty\, ,
\]
dann sagen wir dass die Reihe $\sum_n f_n$ konvergiert normal.
\end{Def}

Eine Potenzreihe ist ein dann ein Beispiel einer Reihe von Funktionen.
Der Beweis vom Lemma \ref{l:Abel} impliziert dass eine Potenzreihe
im {\em Inneren} des Konvergenzkreis normal konvergiert.

\begin{Lem}\label{l:Abelbis}
  Sei $\sum a_nx^n$ eine Potenzreihe mit Konvergenzradius $R$. Sei $\rho<R$ und
$I= [-\rho, \rho]$. Die Potenzreihe konvergiert normal auf $I$.
\end{Lem}
\begin{Bew}
  $\rho <R$ Sei $x_0$ mit $\rho<|x_0|<R$. Dann
  \[\sum\abs{a_n}\abs{x_0}^n\quad\text{konvergiert}\]
Deswegen ist $|a_n||x_0|^n$ eine Nullfolge und 
  \[\exists M : \qquad \abs{a_n}\abs{x_0}^n\leq M\quad \forall n\]
Sei nun $f_n (x):= a_n x^n$.
Dann
\[
\|f_n\|_{C^0 (I)} \;=\; \max_{|x|\leq \rho} |f_n (x)|
\;=\; \max_{|x|\leq \rho} |a_n||x|^n = |a_n|\rho^n
\]
\[ 
\leq |a_n| |x_0|^n {\underbrace{\left(\frac{\rho}{|x_0|}\right)}_\gamma}^n
\leq M \gamma^n\, .
\]
Aber $\gamma<1$ und aus dem Majorantenkriterium folgt:
\[
\sum_{n=0}^\infty \|f_n\|_{C^0 (I)} <\infty\, .
\]
\end{Bew}
Sein nun $\sum_n f_n = \sum_n a_n x^n$ eine Potenzreihe wie
im Satz \ref{s:diff}. Sei $R$ der entsprechende Konvergenzradius 
und $\rho$ eine beliebige positive reelle Zahl mit $\rho<R$. 
Aus der Bemerkung \ref{b:R=R'} und dem Lemma \ref{l:Abelbis}
schliessen wir:
  \begin{enumerate}
    \item $\forall f_n$ ist differenzierbar
    \item $\sum f_n$ und $\sum f_n'$ sind beide normal konvergent auf $I=[\rho, \rho]$. 
  \end{enumerate}
Dann Satz \ref{s:diff} folgt aus der folgenden allgemeineren Aussage.

\begin{theorem}\label{t:C1Konv}
  Sei $\sum f_n$ eine Reihe von Funktionen auf einem abgeschlossenen
Intervall $I$. Falls:
  \begin{enumerate}
    \item $\sum f_n (x)$ $\forall x\in I$ konvergiert,
    \item $\sum f_n'$ normal konvergent ist
  \end{enumerate}
  dann ist $f$ überall differenzierbar mit $f'=\sum f_n'$.
\end{theorem}

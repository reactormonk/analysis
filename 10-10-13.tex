%\begin{Sat}
%  Seien $a_n\to a$ und $b_n\to b$ reelle Folgen. Falls $a_n\leq b_n$, dann $a\leq b$.
%\end{Sat}
%\begin{Bew}
%  Sei $\varepsilon>0$. Dann:
%  \begin{itemize}
%    \item 
%      \[\exists N:\abs{a_n-a}<\varepsilon\s\forall n\geq N\]
%    \item
%      \[\exists N:\abs{b_n-b}<\varepsilon\s\forall n\geq N'\]
% \end{itemize}
%  Sei $n\geq \max\left\{ N',N \right\}$.
%  \[b-a=b_n+(b-b_n)-a_n+(a_n-a)\geq (b-b_n)+(a_n-a)\]
%  \[\geq -\abs{a_n-a}-\abs{b_n-b}\geq -2\varepsilon\]
%  \[b-a\geq -2\varepsilon\stackrel{\forall \varepsilon>0}{\implies} b-a\geq 0\]
%  (wäre $b-a<0$: Sei
%  \[Sei \varepsilon=\frac{\abs{b-a}}{3}=\frac{-(b-a)}{3}\]
%  Widerspruch!)
%\end{Bew}
%\begin{Sat}
%  (Einschliessungsregel). Sei $c_n$ eine Folge reeller Zahlen. Seien $a_n\to a$ und $b_n\to a$ so dass $a_n\leq c_n\leq b_n$. Dann $c_n\to a$.
%\end{Sat}
%\begin{Bew}
%  Sei $\varepsilon>0$.
%  \begin{itemize}
%    \item 
%      \[\exists N:\abs{a_n-a}<\varepsilon\s\forall n\geq N\]
%    \item
%      \[\exists N:\abs{b_n-a}<\varepsilon\s\forall n\geq N'\]
%  \end{itemize}
%  Sei $n\geq \max\left\{ N,N' \right\}$.
%  \[a\varepsilon\leq a_n\leq c_n\leq b_n<a+\varepsilon\]
%  \[\implies \abs{c_n-a}<\varepsilon\]
%\end{Bew}
%\begin{Bsp}
%  \[\Limi{n}\sqrt[n]{n^k}=1\s k\in\mb{N}\]
%  \[\sqrt[n]{n^s}\s s\in\mb{Q}, s>0\]
%  \[\underbrace{1}_{a_n}\leq\underbrace{\sqrt[n]{n^s}}_{c_n}\leq\underbrace{\sqrt[n]{n^k}}_{b_n}\]
%  Einschliessungsregel: $\sqrt[n]{n^s}\to 1$.
%\end{Bsp}
\subsection{Monotone Folgen}
\begin{Def}
  Eine Folge $(a_n)$ reeller Zahlen heisst fallend (bzw. wachsend) 
falls $a_n\leq a_{n-1}$ $\forall n\in\mb{N}$ (bzw. $a_n\geq a_{n-1}\forall n\in\mb{N}$). 
Monoton bedeuted fallend oder wachsend.
\end{Def}
\begin{Sat}
  Eine monotone beschränkte Folge konvergiert.
\end{Sat}
\begin{proof}[Beweis]
  OBdA k\"onnen wir $(a_n)$ wachsend annehmen. (Sei $a_n$ fallend, 
dann $-a_n$ ist wachsend. Falls $-a_n \to L$, dann 
\[
a_n=(-1)(-a_n) \to \lim(-1)\lim(-a_n) = -L)\, .
\] 
Sei 
  \[s=\sup \underbrace{\left\{ a_n:n\in\mb{N} \right\}}_{=:M}\]
  Behauptung:
  \[s=\Limi{n} a_n\]
Da $a_n\geq a$, wir sollen beweisen dass:
\begin{equation}\label{e:zuBeweisen}
 \forall \varepsilon>0\s\exists N: \qquad
a_n> s-\varepsilon\quad\forall n\geq N\, .
\end{equation}
Sei $\varepsilon > 0$. Dann
  \[\exists a_j\in M:\qquad a_j>s-\varepsilon\]
(sonst w\"are $s-\varepsilon$ eine obere Schranke kleinere als
$s$).
  Die Folge wächst $\implies$ $a_n\geq a_j>s-\varepsilon$ $\forall n\geq j$.
\end{proof}
\begin{Bsp} Die Beschr\"akheit impliziert nicht die Konvergenz:
 \[a_n=(-1)^n\]
\end{Bsp}
\subsection{Der Satz von Bolzano-Weierstrass}
\begin{Def}
  Sei $(a_n)$ eine Folge. Eine Teilfolge von $(a_n)$ ist eine neue Folge 
$b_k:= a_{n_k}$, wobei $n_k\in\mb{N}$ mit $n_k>n_{k-1}$ (zum Beispiel:
\[\underbrace{a_0}_{b_0} \s a_1\s\underbrace{a_2}_{b_1}
\s a_3\s a_4 \s a_5 \s\underbrace{a_6}_{b_2} \quad \ldots )\, .\]
\end{Def}
\begin{Sat}[Bolzano-Weierstrass]
  Jede berschränkte Folge $(a_n)$ $(\subset\mb{R}, \mb{C})$ 
besitzt eine konvergente Teilfolge.
\end{Sat}
\begin{proof}[Beweis]
  {\bf Schritt 1}: Sei $(a_n)$ eine Folge reeller Zahlen. 
Sei $I$ und $M\in\mb{R}$ so dass $I\leq a_n\leq M$ 
$\forall n\in\mb{N}$. OBdA $I<M$, sonst ist $(a_n)$ eine konstante
Folge. Wir definieren $J_0 = [I, M]$ und teilen es in zwei Intervallen: 
 \[ J_0 = \left[ I,A_0 \right]\cup 
\left[ A_0, M \right]\s A_0=\frac{M-I}{2}+I=\frac{M+I}{2}\]
  mindestens ein Intervall enthält unendlich viele $a_n$. 
Nennen wir dieses Intervall $J_1$.

Rekursiv definieren wir eine Folge von Intervallen $J_k$ s.d.
  \begin{itemize}
    \item $J_{k+1}\subset J_k$;
    \item Die Länge $\ell_k$ von $J_k$ ist $(M-I)2^{-k}$;
    \item jedes Intervall ent\"alt unendlich viele gliedern der Folge
$(a_n)$.
  \end{itemize}
Diese Folge ist eine Intervallschachtelung und deswegen $\exists ! L$
mit $L\in J_i \;\forall i$.

Wir w\"ahlen $n_0\in \mb{N}$ s.d. $a_{n_0}\in J_0$. Da $J_1$ ent\"alt 
unendlich viele $a_n$, $\exists n_1>n_0$ mit $a_{n_1}\in J_1$.
Rekursiv definieren wir eine Folge nat\"urlicher Zahlen $(n_k)$
mit $n_{k+1}>n_k$ und $a_{n_k}\in J_k$. Die Folge $b_k:= a_{n_k}$
ist eine Teilfoge von $(a_n)$. Ausserdem
\[
|b_k-L|\leq \ell_k = 2^{-k} (M-I)\, ,
\]
weil $b_k, s\in J_k$. Deswegen $b_k\to L$.

\[\exists a\in J_k\s\forall k\in\mb{N}\]
  \[\exists n_0: a_{n_0}\in J_0\]
  $J_1$ enthält unendlich viele $a_n\implies \exists n_1>n_0$ mit $a_{n_1}\in J$. 

\medskip

{\bf Schritt 2.} Sei nun $a_k=\xi_k+i\zeta_k$ eine beschr\"ankte 
komplexe Folge. 
$(\xi_k)$ ist eine beschränkte Folge reeller Zahlen. 
Aus dem Schritt 1 $\exists (\xi_{k_j})$ Teilfolge die konvergiert.
$(\zeta_{k_j})$ ist auch eine beschränkte Folge reeller Zahlen 
und deswegen besitzt eine konvergente Teilfolge $(\zeta_{k_{j_n}})$.
Dann
\[b_n := a_{k_{j_n}}=\xi_{k_{j_n}}+i\zeta_{k_{j_n}}\]
ist eine konvergente Teilfolge!
\end{proof}
\begin{Def}
  Falls $(a_k)$ eine Folge ist und $a$ der Limes einer Teilfolge, dann heisst $a$ \ul{Häufungswert}.
\end{Def}
\begin{Lem}\label{l:char_hauf}
  Sei $(a_k)$ eine Folge. $a$ Häufungswert $\iff$ 
$\forall$ offenes Invervall mit $a\in I$ $\exists$ unendlich viele $a_k\in I$.
\end{Lem}
\begin{proof}[Beweis]
Trivial \end{proof}
\begin{Def}
Wenn die Menge der Häufungswerte von $(a_n)$ (relle Folge) ein Supremum 
(bzw. ein Infimum) besitzen, heisst dieses Supremum 
``Limes Superior'' (bzw. ``Limes Inferior'') und wir nuzten die Notation
\[\limsup_{n\to\infty} a_n 
\qquad (\mbox{bzw. } \liminf_{n\to\infty} a_n).\]
Wenn die Folge keine obere (bzw. untere) Schranke besitzt,
wir schreiben
\[ \limsup_{n\to \infty} a_n = \infty \quad
 (\mbox{bzw. } \liminf_{n\to\infty} a_n = -\infty.)
\]
\end{Def}

\begin{Bem}
Eine konvergente Folge hat genau einen H\"aufungswert, d.h.
der Limes der Folge! 
\end{Bem}

\begin{Lem}
Der Limes Superior (bzw. Inferior) ist das Maximum (bzw. Minimum) 
der Häufungswerte, falls er enldich ist. 
Ausserdem eine reelle Folge konvergiert genau
dann, wenn der Limes superior und der Limes inferior gleich und endlich
sind.
\end{Lem}
\begin{proof}[Beweis]
{\bf Teil 1} Sei $\limsup_n a_n= S\in \mb{R}$. Zu beweisen
ist dass $S$ ein H\"aufungswert ist. Sei $I=]a,b[$ ein Intervall mit
$S\in I$. Wir behaupten dass $I$
unendlich viele Gliedern von $(a_n)$ besitzt: es folgt dann
aus Lemma \ref{l:char_hauf} dass $S$ ein H\"aufungswert ist. Da $S$ das Supremum
der H\"aufungswerte ist, $\exists$ ein H\"aufungswert $h >a$. 
Aber dann $h\in I$, und aus Lemma \ref{l:char_hauf}
folgt dass $I$ unendlich viele $a_n$ enth\"alt. 

\medskip

{\bf Teil 2}. Sei $(a_n)$ eine Folge mit
\[
 \liminf_{n\to\infty} a_n = \limsup_{n\to \infty} a_n = L \in \mb{R}\, .
\]
Es folgt dass $(a_n)$ eine beschr\"ankte Folge ist. Falls
$a_n$ nicht nach $L$ konvergiert, dann $\exists \varepsilon >0$
und unendlich viele $a_n$ mit $|a_n-L|>\varepsilon$, d.h. eine
Teilfolge $b_k=a_{n_k}$ von $(a_n)$ mit $|b_k-L|>\varepsilon$.
Aus dem Satz von Bolzano-Weiestrass schliessen wir die Existenz
einer konvergenten Teilfoge von $(b_n)$ mit Limes $\ell\neq L$.
$\ell$ ist ein H\"aufungswert von $(a_n)$. Das ist ein Widerspruch
weil, nach der Definition von Liminf und Limsup, $L\leq \ell \leq L$.
\end{proof}



\subsection{Konvergenzkriterium von Cauchy}
\begin{Sat}
  Eine Folge komplexer Zahlen konvergiert genau dann, wenn:
\begin{equation}\label{e:Cauchy}
\forall \varepsilon>0\s\exists N:\abs{a_n-a_m}<\varepsilon\s\forall n,m\geq N\, .
\end{equation}
\end{Sat}

\begin{proof}[Beweis]
  {\bf Konvergenz $\implies$ Cauchy.} Sei $(a_n)$ s.d. 
$a_n\to a$. Sei $\varepsilon>0$. Dann 
  \[\exists N:\abs{a_n-a}<\frac{\varepsilon}{2}\s\forall n\geq N\]
Deswegen:
  \[\abs{a_n-a_m}=\abs{a_n-a+a-a_m}
\leq \abs{a_n-a}+\abs{a-a_n}<\frac{\varepsilon}{2}
+\frac{\varepsilon}{2}\s\forall n,m\geq N\]

\medskip

{\bf Cauchy $\implies$ Konvergenz}. Sei $(a_n)$ eine ``Cauchy-Folge''
(d.h. \eqref{e:Cauchy} gilt).   
\begin{Bem}\label{b:h_gen}
Falls $a$ ein Häufungswert ist, dann konvergiert die Ganze Folge nach $a$! 
\end{Bem}
In der Tat, sei $a_{n_k}$ eine Teilfoge die nach $a$ konvergiert.
\begin{equation}\label{e:est1}
\forall \varepsilon>0\s\exists K:\qquad 
k>K \quad \implies\quad \abs{a_{n_k}-a}<\frac{\varepsilon}{2}
\end{equation}
\begin{equation}\label{e:est2}
\exists N: \qquad \abs{a_{n_k}-a}<\frac{\varepsilon}{2}\quad \forall m,n>N\, .
\end{equation}
  Sei nun $n\geq N$. 
Sicher $\exists n_k>N$ mit $k\geq K$. Deswegen, f\"ur $n>N$,
  \[\abs{a-a_n}=\abs{a-a_{n_k}+a_{n_k}-a_n}
\leq\abs{a-a_{n_k}}+\abs{a_{n_k}-a_n}\stackrel{\eqref{e:est1} \&
\eqref{e:est2}}{<} \varepsilon\, .\]
Das beweist die Bemerkung \ref{b:h_gen}.

Deswegen, um den Satz zu beweisen, es gen\"ugt die Existenz
eines H\"aufungspunkts zu zeigen. Nach Bolzano-Weiestrass,
die beschr\"anktheit der Folge impliziert die Existenz eines
H\"aufungspunts. Wähle nun $\varepsilon=1$. 
  \[\exists \bar N:\abs{a_n-a_m}<1\s\forall n,m\geq \bar N\, .\]
Deswegen
\[\abs{a_n}\leq \abs{a_n-a_{\bar N}}+\abs{a_{\bar N}}<\abs{a_{\bar N}}+1\s
\qquad\forall n
\geq \bar N\]
Sei nun
\[M:=\max\left( \left\{ \abs{a_k}:k<\bar N \right\}\cup\left\{ \abs{a_{\bar N}+1} 
\right\} \right)\]
Dann $\abs{a_n}\leq M$ und die Folge ist beschr\"ankt.
\end{proof}

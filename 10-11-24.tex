\subsection{Differentation einer Potenzreihe}
\begin{Sat}
  Sei $f(x)=\sum^\infty_{n=0}a_nx^n$ mit Konvergenz $R$. Dann ist $f$ differenzierbar auf $]-R,R[$ und
  \[f'(x)=\sum^\infty_{n=1}n a_n x^{n-1}\]
\end{Sat}
\begin{Bem}
  $f_n(x)=a_nx^n$, $\rho < R$. Dann:
  \[\sum f_n, \sum f_n'\s\text{konvergieren normal auf}\s\overbrace{\left[ \rho, -\rho \right]}^{I:=}\]
  d.h.
  \[\sum\Norm{f_n}_{C^\circ(I)}<+\infty\]
  \[\sum\Norm{f_n'}_{C^\circ(I)}<+\infty\]
  wobei
  \[\Norm{f_n'}_{C^\circ(K)}:=\max \abs{f}\]
\end{Bem}
\begin{Sat}
  Falls $\sum f_n$ und $\sum f_n'$ normal konvergieren auf $[a,b]$, dann ist $f=\sum f_n$ auf $I$ und $f'(x)=\sum f_n'(x)$ $\forall x\in I$ differenzierbar (überall!).
\end{Sat}
\begin{Bew}
  Sei $x\in I$. Die Ableitung an der Stelle $x$:
  \[\Limi{h}\abs{\frac{f(x+h)-f(x)}{h}-f'(x)}=0\]
  d.h.
  \[\Limi{h}\abs{\sum^\infty_{h=0}\left( \frac{f_n(x+h)-f_n(x)}{h}-f_n'(x) \right)}=0\]
  \[\sum^\infty_{h=0}\left( \frac{f_n(x+h)-f(x)}{h}-f_n'(x) \right)\]
  \[\leq \underbrace{\abs{\sum_{n=0}^N\frac{f_n(x+h)-f_n(x)}{h}-f_n(x)}}_{A} + \underbrace{\abs{\sum_{n=N+1}^\infty\frac{f_n(x+h)-f_n(x)}{h}-f_n(x)}}_{B}\]
  $\forall \varepsilon>0$ wählen wir $N$ und $\bar h$ für $\abs{h} < \bar h$.
  $A<\frac{\varepsilon}{2}$ und $B<\frac{\varepsilon}{2}$, fertig
  \[B\leq \sum\infty_{n=N+1}\left\{ \abs{\frac{f_n(x+h)-f_n(x)}{h}}+\abs{f_n'(x)} \right\}\]
  \[\leq \sum^\infty_{n=N+1}\left\{ \Norm{f_n'}_{C^\circ(I)}+\Norm{f_n'}_{C^\circ(I)} \right\}\]
  \[\sum^\infty_{n=N+1}2\Norm{f_n'}_{C^\circ(I)}=2\left\{\overbrace{\sum^\infty_{n=0} \Norm{f_n'}_{C^\circ(I)}}^{b} - \overbrace{\sum^N_{n=0}\Norm{f_n'}_{C^\circ(I)}}^{b_n\to b}\right\}\]
  \[<\frac{\varepsilon}{2}\s\text{für $N$ gross genug}\]
  \[A=\abs{\sum_{n=0}^N\underbrace{\left( \frac{f_n(x+h)-f_n(x)}{h}-f_n'(x) \right)}_{\rightarrow 0}}\]
  \[<\frac{\varepsilon}{2}\s\text{für}\s\abs{h}<\bar h\]
\end{Bew}
\begin{Def}
  Eine Funktion $f$ ist 2 mal differenzierbar wenn an einer Stelle $x\in I$:
  \begin{itemize}
    \item $f'$ existiert $\forall y\in J$ wobei $x\in J$ ($x\in ]a,b[$ $\leftrightarrow$ Im Innern), ($J=[x,\tilde c[$ $\leftrightarrow$ $I=[x,c[$)
    \item $f'$ differenzierbar in $x$.
  \end{itemize}
  \[(f')'=:f''(x)\s\text{Ableitung zweiter Ordnung}\]
  Induktiv: $f$ $n$-mal differenzierbar in $x$ falls:
  \begin{itemize}
    \item die ($n-1$) Ableitung von $f$ in einer Umgebung von $x$ existiert
    \item $f^{(n-1)}$ differenzierbar in $x$, $f^{(n)}(x):=\left( f^{(n-1)} \right)'(x)$
  \end{itemize}
\end{Def}
\begin{Bem}
  $f(x)=\sum^\infty_{n=0}a_nx^n$ dann ist $f$ $k$-mal differezierbar $\forall k$.
  \[f(0)=a_0\]
  \[f'(0)=a_1\]
  \[f^{(k)}(0)=k!a_k\]
  \[f^{(k)}(x)=\sum^\infty_{n=k}n(n-1)(n-2)\cdots(n-k+1)a_nx^{n-k}\]
\end{Bem}
\begin{Def}
  Eine Funktion $f$ heisst analytisch an einer Stelle $x_0$, falls auf einem Intervall $]x_0-\rho, x_0+\rho[$ gilt
  \[f(x)=\sum a_n(x-x_0)^n\]
\end{Def}
\begin{Kor}
  Sei $f$ analytischin $x_0$, dann ist $f$ beliebig differenzierbar und
  \[f(x)=\sum\infty_{n=0}\frac{f^{(n)}(x_0)}{n!}(x-x_0)^n\]
  Beliebig mal differenzierbar $\not\implies$ analytisch!
\end{Kor}
\begin{Bsp}
  \[x_0=0\]
  \[f(x)=e^x=f'(x)=f''(x)=\cdots\implies f^{(k)}(x)=e^x\]
  \[\implies f^n(0)=1\implies e^x=\sum\frac{x^k}{k!}\]
\end{Bsp}
\subsection{Konvexität}
\url{http://de.wikipedia.org/wiki/Konvexe_und_konkave_Funktionen}
\begin{Def}
  Eine $f:I\to R$ heisst konvex, wenn:
  $\forall x_1 < x_2\in I$
  \begin{equation}\label{e:kk}
    f(x)\leq\frac{x-x_1}{x_2-x_1}f(x_2)+\frac{x_2-x}{x_2-x_1}f(x_1)=g(x)\s\forall x\in ]x_1,x_2[
  \end{equation}
  \begin{tabular}{l|c|l}
    streng konvex & $<$\\
    konkav & $\geq$& in \ref{e:kk}\\
    streng konkav & $=$\\
  \end{tabular}
\end{Def}
\begin{Bem}
  konvex $\not\implies$ differenzierbar
\end{Bem}
\begin{Sat}\label{s:1011242}
  Sei $f:I\to\mb{R}$ stetig und differenzierbar im Inneren
  \[f\s\text{konvex} \iff f'(x_1)\leq f'(x_2)\s\forall x_1<x_2\]
  \[f\s\text{streng konvex} \iff f'(x_1) < f'(x_2)\s\forall x_1<x_2\]
\end{Sat}
\begin{Kor}
  Sei $f$ wie im Satz \ref{s:1011242} aber 2-mal differenzierbar im Inneren
  \[f\s\text{konvex}\iff f''\geq 0\]
  \[f\s\text{streng konvex}\Leftarrow f''> 0\]
\end{Kor}
\begin{Bsp}
  $f(x)=x^4$ $f$ ist stetig konvex $f''(x)=12x^2$ $(f''(0)=0)$
\end{Bsp}
\begin{Bem}
  Sei $f$ differenzierbar überall und 2 mal differenzierbar ain einer Stelle $x_0$ mit $f'(x_0)=0$. Falls 
  \begin{itemize}
    \item $f''(x_0)>0$ $\implies$ $x_0$ ist ein lokales Minimum
    \item $f''(x_0)<0$ $\implies$ $x_0$ ist ein lokales Maximum
  \end{itemize}
\end{Bem}
\begin{Bsp}
  $f'(x_0)=0$, $f''(x_0)>0$. $\exists \varepsilon$ so dass
  \[f'(x)>0\s\forall x\in ]x_0-\varepsilon,x_0[\]
  \[f'(x)<0\s\forall x\in ]x_0,x_0+\varepsilon[\]
  \[\Lim{x}{x_0}\frac{f'(x)-f'(x_0)}{x-x_0}=f''(x_0)\implies\Lim{x}{x_0}=f''(x_0)>0\]
  \[\implies\exists\varepsilon:\frac{f(x)-\overbrace{f(x_0)}^{=0}}{x-x_0}>0\s\forall x\in]x_0-\varepsilon,x_0+\varepsilon[\setminus\left\{ x_0 \right\}\]
  \[\implies'(x)>0\s\forall x\in ]x_0,x_0+\varepsilon[\]
  \[\implies'(x)<0\s\forall x\in ]x_0-\varepsilon,x_0[\]
\end{Bsp}
\begin{Lem}
  \[\ref{e:kk}\iff f(\lambda x_1+(1-\lambda)x_2)\leq\lambda f(x_1)+(1-\lambda)f(x_2)\s\forall x_1 < x_2\s\forall \lambda\in ]0,1[\]
\end{Lem}
\begin{Bew}
  $x_1<x_2$
  \[f(x)=\frac{x_2-x}{x_2-x_1}f(x_1)+\frac{x-x_1}{x_2-x_1}f(x_2)\s\forall x\in ]x_1,x_2[\]
  Wir setzen $\lambda\frac{x_2-x}{x_2-x_1}$
  \[\forall x\in ]x_1,x_2[\implies \lambda=\frac{x_2-x}{x_2-x_1}\in ]0,1[\]
  \[\forall \lambda\in ]0,1[\implies x=\lambda x_1+(1-\lambda)x_2\in ]x_1,x_2[\]
  \[\lambda=\frac{x_2-x}{x_2-x_1}\iff \lambda(x_2-x_1)=x_2-x\iff x=\lambda x_1+(1-\lambda)x_2\]
  \[ ]0,1[\ni\lambda\mapsto \lambda x_1+(1-\lambda x_2)\in  ]x_1,x_2[\]
  \[\lambda=\frac{x_2-x}{x_2-x_1}\iff 1-\lambda=1-\frac{x_2-x}{x_2-x_1}=\frac{\not x_2-x_1-\not x_2+x}{x_2-x_1}=\frac{x-x_1}{x_2-x_1}\]
  \[f(\lambda x_1+(1-\lambda)x_2)\leq \lambda f(x_1)+(1-\lambda)f(x_2)\]
\end{Bew}
\begin{Lem}
  $f:I\to \mb{R}$ ist genau dan konvex wenn für jedes Tripel $x_1<x<x_2\in I$ die folgende Ungleichung gilt:
  \[\frac{f(x)-f(x_1)}{x-x_1}\leq \frac{f(x_2)-f(x)}{x_2-x}\]
\end{Lem}
\begin{Bew}
  \[\frac{f(x)-f(x_1)}{x-x_1}\leq \frac{f(x_2)-f(x)}{x_2-x}\]
  \[f(x)\left( \frac{1}{x-x_1}+\frac{1}{x_2-x} \right)\leq \frac{f(x_1)}{x-x_1}+\frac{f(x_2)}{x_2-x}\]
  \[f(x)\frac{x_2-x+x-x_1}{(x-x_1)(x_2-x)}\left( \frac{(x_2-x)(x-x_1)}{x_2-x_1} \right)\]
  \[\leq f(x_1)\frac{x_2-x}{x_2-x_1}+f(x_2)\frac{x-x_1}{x_2-x_1}\]
  \[f(x)\leq f(x_1)\frac{x_2-x}{x_2-x_1}+f(x_2)\frac{x-x_1}{x_2-x_1}\]
\end{Bew}

\begin{Def}
  Eine Kurve ist eine Abbildung $\gamma:[a,b]\to\mb{R}^n$ (d.h. $\forall t\s\gamma(t)\in\mb{R}^n$ 
  \[\gamma(t)=(\gamma_1(t),\cdots,\gamma_n(t))\]
  deswegen $t\to\gamma_i(t)\in\mb{R}$. Die Kurve $\gamma$ heisst differenzierbar wenn jede $\gamma_i$ differenzierbar ist.
  \[\gamma'=(\gamma'(t),\cdots,\gamma_n'(t))\]
\end{Def}
\begin{Sat}(Kettenregel 1. Version) Sei $f:U\to\mb{R}$ mit $U$ Umgebung von $x$ und $f$ differenzierbar in $x$. Sei $\gamma:[a,b]\to U$ eine differenzierbare Kurve mit $\gamma(t_0)=x$. Sei $g=f\circ \gamma$
  \[g(t)=f(\gamma(t))\]
  Sei $g$ in $t_0$ differenzierbar. Dann
  \[g'(t_0)=\md|_\gamma(t_0)(\dot\gamma(t_0))=\seq{\nabla f(\gamma(t_0)),\dot\gamma(t_0)}\]
  % TODO
\end{Sat}
\begin{Bew}
  Das Ziel:
  \[\Limo{h} \frac{g(t_0+h)-g(t_0)-h\left[ \md f|_{\gamma(t_0)}(\dot\gamma(t_0)) \right]}{h}=0\]
  \begin{equation}
    \label{e:1103161}
    R(h)=g(t_0+h)-g(t_0)-g(t_0)-h\left[ \md f|_{\gamma(t_0)}(\dot\gamma(t_0)) \right]
  \end{equation}
  \begin{equation}
    \label{e:1103162}
    \Limo{h}\frac{R(h)}{h}=0
  \end{equation}
  Neue Notation
  \[\ref{e:1103162}\iff R(h)=o(h)\]
  \[x_0=\gamma(t_0)\]
  Annahmen: Differenzierbarkeit von $f$
  \[\Limo{k}\frac{f(x_0+k)-f(x_0)-\md f|_{x_0}(k)}{\Norm{k}}\left( =\frac{r(k)}{\Norm{k}} \right)=0\]
  \[\left( r(k)=o(\Norm{k}) \right)\]
  Differenzierbarkeit von $\gamma$:
  \[\Limo{k}\frac{\gamma(x_0+k)-\gamma(x_0)-\md h\gamma'|_{x_0}(k)}{h}\left( =\frac{p(k)}{\Norm{k}} \right)=0\]
  \[p(h)=o(h)\]
  \[\gamma(t_0+h)=\gamma(t_0)+k\left( =\gamma(t_0+h)-\underbrace{\gamma(t_0)}_{x_0} \right)\]
  \[g(t_0+h)-g(t_0)=f(\gamma(t_0+h))-g(\overbrace{\gamma(t_0)}^{x_0})\]
  \[=f(\gamma(t_0)-k)-f(\gamma(t_0))=\md f|_{\gamma(t_0)}(k)+r(k)\]
  \[=\md f|_{\gamma(t_0)}(\gamma(t_0+h)-\gamma(t_0))+r(k)\]
  \[=\md f|_{\gamma(t_0)}(h\dot\gamma(t_0)+p(h))+r(k)\]
  \[\stackrel{\text{Linearität von $\md f$}}{=}h\md f|_{\gamma(t_0)}(\dot\gamma(t_0))+\md f|_{\gamma(t_0)}(p(h))+r(k)\]
  \[g(t_0+h)-g(t_0)-h\md f|_{\gamma(t_0)}(\dot\gamma(t_0))\]
  \[=f|_{\gamma(t_0)}(p(h))+r(\gamma(t_0+h)-\gamma(t_0))=R(h)\]
  \[\abs{R(h)}\leq \frac{\underbrace{\overbrace{\abs{f|_{\gamma(t_0)}(p(h))}}^{L}+r(\gamma(t_0+h)-\gamma(t_0))}}{\Norm{h}}\]
  \[\leq\Norm{L}\frac{p(h)}{\Norm{h}}+\frac{r(\gamma(t_0+h)-\gamma(t_0)}{\Norm{h}}\]
  Ziel
  \[\Limo{h}\frac{r(\gamma(t_0+h)-\gamma(t_0)}{\abs{h}}\]
  Falls 
  \[r(\gamma(t_0+h)-\gamma(t_0)=0\]
  dann $r(0)=0$. Wenn 
  \[r(\gamma(t_0+h)-\gamma(t_0)\neq 0\]
  \[=\frac{r(\gamma(t_0+h)-\gamma(t_0)}{\Norm{\gamma(t_0+h)-\gamma(t_0)}}\frac{\Norm{t_0+h)-\gamma(t_0)}}{\abs{h}}\]
  \[\frac{r(\gamma(t_0+h)-\gamma(t_0)}{\Norm{\gamma(t_0+h)-\gamma(t_0)}}=\frac{r(k)}{\Norm{k}}\to 0\]
  \ldots wenn $\Norm{k}\to 0$ und $h\to 0$. Es fehlt die Beschränktheit von
  \[\frac{\Norm{t_0+h)-\gamma(t_0)}}{\abs{h}}\]
  \[\frac{t_0+h)-\gamma(t_0)}{h}-\frac{\not h\dot \gamma(t_0)}{\not h}=\frac{p(h)}{h}\]
  \[\frac{\gamma(t_0+h)-\gamma(t_0)}{h}=\underbrace{\dot\gamma(t_0)}_{\text{konstant}}+\underbrace{\frac{p(h)}{h}}_{\to 0}\]
  Deswegen
  \[\Limo{h}\frac{\Norm{\gamma(t_0+h)-\gamma(t_0)}}{\abs{h}}=\Norm{\dot\gamma(t_0)}\]
  \[\implies \frac{\abs{R(h)}}{\Norm{h}}\to 0\]
  $\implies$ Differenzierbarkeit und Kettenregel!
\end{Bew}
\begin{Bem}
  Der Gradient ist orthogonal zur Niveaumenge (Höhenlinien).
\end{Bem}
\begin{Def}
  Sei $\gamma:[a,b]\to U$ eine differenzierbare Kurve, $U$ offen. Sei $f:U\to\mb{R}$ differenzierbar. Wenn $f(\gamma(t))=c_0$ ($c_0$ hängt nicht von $t$ ab). Dann
  \[\nabla f(\gamma(t))\bot \dot\gamma(t)\]
  d.h.
  \[\seq{\nabla f(\gamma(t)), \dot\gamma(t) }=0\]
  \[0=g'(t)=(f(\gamma(t)))'\stackrel{\text{Kettenregel}}{=}\seq{\nabla f(\gamma(t)), \dot\gamma(t)}\]
\end{Def}
\subsection{Mittelwertsatz und Schrankensatz}
  $f:[a,b]\to\mb{R}$, $\xi\in ]a,b[$
  \[f(b)-f(a)=f'(\xi)(b-a)\]
  Sei nun:
  \[f:U\mapsto\mb{R}\s\text{differenzierbar auf $U$}\]
  \[x,y\in U\s\text{so dass das Segment}\s[x,y]\subset U\]
  Was ist ein Segment? Gerade durch $x$ und $y$
  \[\left\{ x+t(y-x)|t\in \mb{R} \right\}\]
  \[\left[ [x,y] \right]=\left\{ x+t(y-x)|t\in \left[ 0,1 \right] \right\}\]
  \[\gamma(t):= x+t(y-x)\]
  \[f(y)-f(X)=\left( x_1+t(y_1-x_1),\cdots,x_n+t(y_n-x_n) \right)\]
  $\gamma$ ist differenzierbar.
  \[g=f\circ \gamma g(t)=f(\gamma(t))\]
  \[g(1)-g(0)=g'(\tau)\s\text{für}\s \tau\in ]0,1[\]
  \[f(y)-f(x)=\md f|_{\gamma(\tau)}(\dot\gamma(\tau))\]
  \[\dot\gamma(\tau)=(\gamma_1'(\tau),\cdots,\gamma_n'(\tau))\]
  \[=(y_1-x_1,\cdots,y_n-x_n)=y-x\]
  \[\gamma(\tau)=\xi\]
  \begin{equation}
    \label{e:1103163}
    f(y)-f(x)=\md f|_\xi(y-x)=\partial_{y-x}f(\xi)
  \end{equation}
\begin{Sat}
  (Mittelwertsatz) $U$ offen, $[x,y]\subset U$ und $f:U\to\mb{R}$ differenzierbar. Dann $\exists \xi\in ]x,y[$ so das \ref{e:1103163} gilt.
\end{Sat}
\begin{Def}
  $U$ sternförmig: wenn $0\in U$ und $[x,0]\subset U$ $\forall x\in U$. Sternförmig mit Zentrum $x_0$ wenn $x_0\in U$ $[x,x_0]\subset U$ $\forall x\in U$
\end{Def}
\begin{Sat}
  (Schrankensatz) Sei $U$ eine offene Menge, die sternförmig ist und $f:U\to\mb{R}$ eine differenzierbare Funktion mit
  \[\sup_{x\in U}\Norm{\md f|_x}_{O}=S<\infty \left( =\sup_{x\in U}\Norm{\nabla f(x)} \right)\]
  Dann
  \[\abs{f(x)-f(0)}\leq S\Norm{x}\]
  Wenn $U$ konvex ist, d.h. das Segment $[x,y]\subset U$ $\forall x,y\in U$, dann
  \[\abs{f(x)-f(y)}\leq S\Norm{y-x}\]
\end{Sat}
\begin{Def}
  $f:\underbrace{K}_{\in \mb{R}^n}\to\mb{R}$ heisst Lipschitz wenn $\exists L[0,+\infty[$ so dass
  \[\abs{f(y)-f(x)}\leq L\Norm{y-x}\s\forall x,y\in K\]
  Wenn $f:(X,d)\to\mb{R}$ Lipschitz bedeutet die Existenz eines $L$ so dass
  \[\abs{f(y)-f(x)}\leq L d(y-x)\s\forall x,y\in K\]
\end{Def}

\begin{Sat}\label{s:umkehr}[Differentiation der Umkehrfunktion]
  Sei $g$ die Umkehrfunktion einer streng monotonen Funktion $f:I\to\mb{R}$. 
Falls $f$ in $x_0$ differenzierbar ist und $f'(x_0)\neq 0$, 
dann ist $g$ in $y_0 = f(x_0)$ differenzierbar und
  \[g'(y_0)=\frac{1}{f(x_0)}\left( =\frac{1}{f'(g(y_0))} \right)\]
\end{Sat}
\begin{proof}[Beweis]
  \[f(x)-f(x_0)=\phi(x)(x-x_0)\]
  wobei
  \begin{itemize}
    \item $\phi$ ist stetig in $x_0$
    \item $\phi(x_0) = f'(x_0)$
  \end{itemize}
  \begin{eqnarray*}
    x=g(y)\\
    x_0=g(y_0)
  \end{eqnarray*}
  $\implies$
  \[y-y_0=\phi(g(y))(g(y)-g(y_0))\]
  \[g(y)-g(y_0)=\frac{1}{\phi(g(y))}( y - y_0) \quad \mbox{falls $\phi (g(y))\neq 0$.}\]
  Aber:
  \[\phi(g(y_0))=\phi(x_0)=f'(x_0)\neq 0\]
  $\phi$ ist stetig in $x_0$ und $g$ ist stetig in $y_0$ $\implies$ $\phi(g)$ ist stetig in $y_0$
  \[\exists\varepsilon>0: \abs{y-y_0}<\varepsilon\implies \phi(g(y))\neq 0\]
  Sei
  \[\psi = \begin{cases}
    \frac{1}{\phi(g(y))}&\abs{y-y_0}<\varepsilon\\
    \frac{g(y)-g(y_0)}{y-y_0}&\abs{y-y_0}>\varepsilon
  \end{cases}\]
  \[\implies g(y)-g(y_0)=\psi(y)(y-y_0)\]
  und $\psi$ ist stetig an der Stelle $y_0$. $g$ ist differenzierbar in $y_0$ und deswegen
  \[\psi(y_0)=g'(y_0)\]
  \[=\frac{1}{\phi(g(y_0))}=\frac{1}{\phi(x_0)}=\frac{1}{f'(x_0)}=\frac{1}{f'(g(y_0))}\]
\end{proof}
\begin{Bem}
  Sei $f:I\to\mb{R}$ streng monoton und stetig. Sei $g$ die Umkehrfunktion von $f$ $g:J\to I$.
 Angenommen dass beide Funktionen differenzierbar sind, die Kettenregel impliziert
  \[(f\circ g)'(x_0)=f'(g(x_0))g'(x_0)=1\, .\]
Falls $f' (g(x_0))\neq 0$, wir schliessen $g'(x_0)=\frac{1}{f'(g(x_0))}$. Das ist
aber kein Beweis vom Satz \ref{s:umkehr}, da die Differenzierbarkeit von $g$
angenommen und nicht bewiesen wird.
\end{Bem}
\begin{Bsp}
  (Übung: arcsin', arccos')
  \[\tan'(y_0)=\frac{1}{\cos^2(y_0)}\neq 0\]
  \[(\arctan)'(x_0)=\frac{1}{\tan'(\arctan(x_0))}=\frac{1}{\frac{1}{\cos^2(\arctan(x_0))}}\]
  \[=\cos^2(\arctan(x_0))\]
  \[\cos^2=\frac{1}{1+\tan^2}\left( =\frac{1}{1+\frac{\sin^2}{\cos^2}}=\frac{1}{\frac{\cos^2+\sin^2}{\cos^2}}=\cos^2 \right)\]
  \[\cos^2(\arctan(x_0))=\frac{1}{1+(\tan(\arctan(x_0)))^2}=\frac{1}{1+x_0^2}\]
  \[\implies \arctan'(x)=\frac{1}{1+x^2}\]
\end{Bsp}

\subsection{Die S\"atze von Rolle und Lagrange}

\begin{Sat}
  Sei $f:I\to\mb{R}$ eine überall differenzierbare Funktion. 
Sei $x_0\in I$ ein Maximum (bzw. ein Minium)
  \begin{itemize}
    \item $x_0$ im Inneren $\implies$ $f'(x_0)=0$
    \item $x_0$ ist das rechte Extremum von $I$ $\implies$
      \[f'(x_0)\geq 0\]
      bzw. bei Minima:
      \[f'(x_0)\leq 0\]
    \item $x_0$ ist das linke Extremum von $I$ $\implies$
      \[f'(x_0)\leq 0\]
      bzw. bei Minima:
      \[f'(x_0)\geq 0\]
  \end{itemize}
\end{Sat}
\begin{proof}[Beweis]
  $x_0$ im Innern.
  \[\begin{cases}
    \lim_{x\downarrow x_0} \frac{\overbrace{f(x)-f(x_0)}^{\leq 0}}{\underbrace{x-x_0}_{\geq 0}}\leq 0\\
    \lim_{x\uparrow x_0} \frac{\overbrace{f(x)-f(x_0)}^{\leq 0}}{\underbrace{x-x_0}_{\leq 0}}\geq 0
  \end{cases}\]
Deswegen $f'(x_0)=0$.
  $x_0$ ist das linke Extremum und eine Maximumstelle:
  \[f'(x_0)=\lim_{x\downarrow x_0}\frac{f(x)-f(x_0)}{x-x_0}\leq 0\, .\]
Die anderen F\"alle sind \"ahnlich.
\end{proof}
\begin{Sat}[Mittelwertsatz, Lagrange]\label{s:lagrange}
  Sei $f[a,b]\to\mb{R}$ stetig (überall) und differenzierbar in $]a,b[$. Dann $\exists \xi\in ]a,b[$ mit
  \[f'(\xi)=\frac{f(b)-f(a)}{b-a}\]
\end{Sat}
\begin{Sat}[Rolle]\label{s:rolle}
  Sei $f$ wie oben mit $f(b)=f(a)$. Dann $\exists\underbrace{\xi}_{\in ]a,b[}:f'(\xi)=0$.
\end{Sat}

Der Satz von Rolle ist ein Fall des Satzes von Lagrange. Aber wir werden
zuerst den Satz von Rolle beweisen und dann den von Lagrange daraus schliessen.

\begin{proof}[Beweis vom Satz \ref{s:rolle}]
  \[f(b)=f(a)\implies \begin{cases}
    \exists x\in ]a,b[\s\text{mit}\s f(x)<f(b)\\
    \exists x\in ]a,b[\s\text{mit}\s f(x)>f(b)\\
    f(x)=f(b)\s\forall x\in ]a,b[
  \end{cases}\]
  Dritte Möglichkeit $\implies$ $f$ ist Konstant!
  \[f'(\xi)=0\s\forall \xi\in ]a,b[ \]
  Erste Möglichkeit $\implies$ Sei $x_0$ eine Maximumstelle von $f$ in $[a,b]$
  \[\implies x_0\in ]a,b[ \mbox{ (weil $f (x_0) > f(a) = f(b)$) } \implies f'(x_0)=0\]
  Zweite Möglichkeit: Sei $x_0$ eine Maximumstelle:
  \[x_0\in ]a,b[\implies f'(x_0)=0\]
\end{proof}
\begin{proof}[Beweis vom Satz \ref{s:lagrange}]
  Sei
  \[g(x)=f(a)+\frac{x-a}{b-a}(f(b)-f(a))\]
  $g(b)=f(b)$ und $g(a)=f(a)$ $\implies$ Sei $h:=f-g$. $h(a)=0$, $h(b)=0$. 
  \[\stackrel{\text{Satz von Rolle}}{\implies}\exists \xi\in ]a,b[\s\text{mit}\s h'(\xi)=0\]
  \[\implies f'(\xi)-g'(\xi)=\frac{f(b)-f(a)}{a-b}\, .\]
\end{proof}
\begin{Kor}\label{k:monot}
  Sei $f:[a,b]\to\mb{R}$ eine differenzierbare Funktion.
  \begin{itemize}
    \item $f'\geq 0$ $\implies$ $f$ ist wachsend.
    \item $f'> 0$ $\implies$ $f$ ist wachsend, streng monoton.
    \item $f'\leq 0$ $\implies$ $f$ ist fallend.
    \item $f'< 0$ $\implies$ $f$ ist fallend, streng monoton.
  \end{itemize}
\end{Kor}
\begin{proof}[Beweis]
  Seien $c<d\in [a,b]$
  \[\stackrel{\text{Mittelwertsatz}}{\implies}\exists \xi\in ]c,d[\s\text{mit}\]
  \[f(d)-f(c)= f'(\xi)\underbrace{(d-c)}_{>0}\]
  $\geq 0$ im ersten Fall, $>0$ im zweiten Fall, usw.
\end{proof}
\begin{Kor}
  Sei $f:]a,b[\to\mb{R}$ differenzierbar. Falls:
  \begin{itemize}
    \item $f'(x)<0$ $\forall x>x_0$
    \item $f'(x)>0$ $\forall x<x_0$
  \end{itemize}
  dann ist $x_0$ das Maximum von $f$ auf $]a,b[$.
\end{Kor}
\begin{Kor}
  Sei $f: ]a,b[\to \mb{R}$ differenzierbar mit $f'\equiv 0$. Dann $f=\text{konst}$.
\end{Kor}
\begin{Bsp}\label{b:tan_in}
  $\tan$ ist streng monoton auf $]-\frac{\pi}{2},\frac{\pi}{2}[$:
  \[\tan'=\frac{1}{\cos^2}>0\]
NB: $\tan$ ist nicht monoton auf 
$\mb{R}\setminus\left\{ \frac{\pi}{2}+k\pi:k\in\mb{Z} \right\}$,
weil (z.B.) $-1 = \tan -\frac{\pi}{4} < 1= \tan \frac{\pi}{4}
> -1 = \tan{3\pi}{4}$. In diesem Fall ist Korollar \ref{k:monot}
nicht anwendbar, weil $\{\frac{\pi}{2}+k\pi:k\in\mb{Z}\}$ kein Intervall ist.
\end{Bsp}
\subsection{Anwendungen des Mittelwertsatzes: Schrankensatz und De L'Hôpitalsche Regel}
\begin{Sat}[Schrankensatz]
  Sei $f:[a,b]\to\mb{R}$ stetig (überall) und differenzierbar in $]a,b[$ mit
  \[\abs{f'(\xi)}\leq M\qquad\forall \xi\in ]a,b[\, .\]
  Dann ist $f$ Lipschitzstetig und
  \[\abs{f(y)-f(x)}\leq M\abs{x-y}\qquad\forall x,y\in [a,b]\, .\]
\end{Sat}
\begin{proof}[Beweis]
  $\forall y\neq x$ (OBdA: $y>x$)
  \[\exists \xi\in ]x,y[\subset ]a,b[: f(y)-f(x)=f(\xi)(y-x)\]
  \[\implies \abs{f(y)-f(x)}=\abs{f'(\xi)}\abs{y-x}\leq M\abs{y-x}.\]
\end{proof}

Die bekannte Funktionen die wir schon gesehen haben sind alle
Differenzierbar mit stetigen Ableitungen. Deswegen, wenn eingeschr\"ankt auf
einem geschlossenen Intervall, ist die Ableitung beschr\"ankt. Der Schrankensatz
impliziert dann die Lipschitzstetigkeit.

\begin{Sat}[Cauchy]
  Seien $f,g:[a,b]\to\mb{R}$ überall stetig und differenzierbar in $]a,b[$. Ausserdem $g'(x)\neq 0$ $\forall x\in ]a,b[$. Dann
  \[\exists \underbrace{\xi}_{\in ]a,b[}:\frac{f(b)-f(a)}{g(b)-g(a)}=\frac{f'(\xi)}{g'(\xi)}\, .\]
\end{Sat}
\begin{Bem} Der Mittelwertsatz ist ein Fall des Satzes von Cauchy: setzen wir
$g(x)=x$. Dann $g' (x)=1$ $\forall x$ und deswegen:
\[
\frac{f(b)-f(a)}{b-a} = \frac{f(b)-f(a)}{g(b)-g(a)}
= \frac{f'(\xi)}{g'(\xi)} = \frac{f'(\xi)}{1} = f'(\xi)\, .
\]
\end{Bem}
\begin{proof}[Beweis] Wie der Satz von Lagrange auch der 
Satz von Cauchy kann man auf dem Satz von Rolle herleiten. 
Wie setzten: 
  \[F(x)=f(x)-\frac{f(b)-f(a)}{g(b)-g(a)}(g(x)-g(a))\]
  \[F(a)=f(a)=F(b)\stackrel{Rolle}{\implies} 
\exists \xi: F'(\xi)=0\implies f'(\xi) = \frac{f(b)-f(a)}{g(b)-g(a)}g'(\xi)\, .\]
\end{proof}

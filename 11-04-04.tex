Sei $f:\underbrace{U}_{\subset \mb{R}^n}\times [a,b]\to\mb{R}$ stetig ($U$ offen) $\forall x\in U$ sei
\[F(x)=\int_a^bf(x,t)\md t\]
\begin{Sat}
  (Differentationssatz) Falls
  \begin{enumerate}
    \item $\forall t\in \left[ x,b \right]$ ist $x\mapsto f(x,t)$ nach $x_i$ partiell differenzierbar
      \[\exists\Part{f}{x_i}(x,t)\s\forall (x,t)\in U\times \left[ a,b \right]\]
    \item und $\Part{f}{x_i}$ ist stetig
  \end{enumerate}
  dann $\exists$ auch $\Part{F}{x_i}(x)$ und
  \[\Part{F}{x_i}(x)=\int_a^b\Part{f}{x_i}(x,t)\md t\]
\end{Sat}
Der Satz bedeutet also dass, mit den obigen Annahmen, d\"urfen wir die
Abletitung und das Integral vertauschen:
\[
\Part{}{x_i}\int_a^bf(x,t)\md t=\int_a^b\Part{}{x_i}f(x,t)\md t
\]
\begin{Bew}
  Sei $x\in U$ und $e_i=(0,\cdots,\underbrace{1}_{i\text{te Stelle}},\cdots,0)$. Wir rechnen
  \begin{eqnarray*}
\Part{F}{x_i} (x)&=&\Limo{\varepsilon}\frac{F(x+\varepsilon e_i)-F(x)}{\varepsilon}\\
&=&\Limo{\varepsilon}\frac{1}{\varepsilon}\left\{ \int_a^bf(x+\varepsilon e_i ,t)\md t-\int_a^bf(x,t)\md t \right\}\\
  &=&\Limo{\varepsilon}\int_a^b\frac{f(x+\varepsilon e_i,t)-f(x,t)}{\varepsilon}\md t
\end{eqnarray*}
Deswegen
  \[\Part{F}{x_i}(x,t)=\int_a^b\Part{f}{x_i}(x,t)\md t\iff\]
  \[\iff \Limo{\varepsilon}\left\{ \int_a^b\frac{f(x+\varepsilon e_i,t)-f(x,t)}{\varepsilon}\md t-\int_a^b\Part{f}{x_i}(x,t)\md t \right\}=0\]
  \[\iff \Limo{\varepsilon}\left\{ \int_a^b\left[ \frac{f(x+\varepsilon e_i,t)-f(x,t)}{\varepsilon}-\Part{f}{x_i}(x,t)\md t \right] \right\}=0\]
  Wir behaupten mehr, d.h.
  \[A (\varepsilon) := \int_a^b\Bigg|\underbrace{\frac{f(x+\varepsilon e_i,t)-f(x,t)}{\varepsilon}}_{\text{(Mittelwetsatz)} = \Part{f}{x_i}(\xi_\varepsilon (t),t)}-\Part{f}{x_i}(x,t)\Bigg|\md t\stackrel{\varepsilon\to 0}{\to}0\]
  wobei $\xi_\varepsilon(t)\in \left[ x,x+\varepsilon e_i \right]$. Also,
  \[ A(\varepsilon)=\int_a^b\Abs{\Part{f}{x_i}\left( \xi(t),t \right)-\Part{f}{x_i}(x,t)}\md t\, .\]
Wir bemerken dass
  \[\Limo{\varepsilon}\xi_\varepsilon(t)=x\]
  und, wegen der Stetigkeit von $\Part{f}{x_i}$,
  \[\Part{f}{x_i}\left( \xi_\varepsilon(t),t \right)\to \Part{f}{x_i}(x,t)\, .\]
In der Tat ist diese Konvergenz gleichm\"assig, d.h.
  \begin{Beh}\label{b:gleichm}
    $\forall \delta>0$ $\exists \varepsilon_0>0$ so dass
    \[\Abs{\varepsilon}\leq \varepsilon_0\implies \sup_{t\in [a,b]}\Abs{\Part{f}{x_i}(\xi_\varepsilon (t),t)-\Part{f}{x_i}(x,t)}<\delta\]
  \end{Beh}
  $\implies$
  \[\limsup_{\varepsilon\to 0} A(\varepsilon)\leq \sup_{\Abs{\varepsilon}<\varepsilon_0}A(\varepsilon)
  \leq \int_a^b\delta\md t=\delta(b-a)\]
  $\delta$ ist beliebig
  \[\Limo{\varepsilon}A(\varepsilon)=0\]
Die Behauptung \ref{b:gleichm} folgt aus dem n\"achsten Lemma.
\end{Bew}

  \begin{Lem}
    Sei $g:U\times [a,b]\to\mb{R}$ stetig (wobei $U\subset\mb{R}^n$ offen ist). Sei $x\in U$ Dann $\forall \delta >0$ $\exists \varepsilon>0$ mit
    \[\sup_{y\in B_\varepsilon(x)}\Abs{g(y,t)-g(x,t)}<\delta\s\forall t\in [a,b]\]
    % TODO Zeichnung
    Betrachte $x$ als ``Parameter'' $\forall y$ sei $t\mapsto g(y,t)=g_y(t)$. Dann $g_y\to g_x$ gleichmässig für $y\to x$.
 \end{Lem}
  \begin{Bem}
    Das Lemma nutzt nur die Kompaktheit von $[a,b]$ (in der Behauptung können wir $[a,b]$ durch eine beliebige kompakte Menge $K\subset\mb{R}$ ersetzen)
  \end{Bem}
  \begin{Bew}
    Sei $\varepsilon>0$ gegeben $\forall (x,t)$ $\exists \delta(x,t)>0$ so dass
    \[\Abs{g(\xi, \tau)-g(x,t)}<\frac{\varepsilon}{10}\s\forall (\xi, \tau)\]
    mit
    \[\Norm{\underbrace{(\xi,\tau)}_{\in \mb{R}^{n+1}}-\underbrace{(x,t)}_{\in \mb{R}^{n+1}}}<\delta(x,t)\]
    \[\Norm{(\xi,\tau)-(x,t)}=\sqrt{\Norm{\xi-x}^2+(t-\tau)^2}\]
$\forall (x,t)$ Sei
  \[U_{x,t}=\underbrace{B_{\frac{\sqrt{2}}{2}\delta(x,t)}}_{\subset\mb{R}^n}(x)\times \left] t-\frac{\sqrt{2}}{2}\delta(x,t),t+\frac{\sqrt{2}}{2}\delta(x,t) \right[\]
\[(y,\tau)\in U_{x,t}\implies \Norm{y-x}\leq\frac{\sqrt{2}}{2}\delta(x,t)\s\text{und}\s\Abs{t-\tau}<\frac{\sqrt{2}}{2}\delta(x,t)\]
  \[\Norm{(y,t)-(x,\tau)}<\sqrt{\frac{1}{2}\delta^2(x,t)+\frac{1}{2}\delta^2(x,t)}=\delta(x,t)\]
  \[\implies (y,t)\in B_{\delta(x,t)}(x,t)\]
 Deswegen  $U_{x,t}\subset B_{\delta(x,t)}(x,t)$. Wir bemerken dass $K$ kompakt istm, weil
    \[\mb{R}\ni t\mapsto(x,t)\]
 eine stetige Funktion ist und $K$ das Bild von $[a,b]$ durch diese Abbildung ist.
  $\left\{ U_{x,t}:t\in [a,b] \right\}$ ist eine offene Überdeckung von $K$. Kompaktheit $\implies$ $\exists \left\{ U_{x_i,t_i}:i\in \left\{ 1,\cdots,N \right\} \right\}$ Überdeckung von $K$. Sei 
  \[\delta=\min\left\{ \frac{\sqrt{2}}{2}\delta(x_i,t_i):i\in\left\{ 1,\cdots,N \right\} \right\}>0\]
  Sei $t\in [a,b]$, $(x,t)\in U_{x_i,t_i}$ für mindestens ein $i\in \left\{ i,\cdots,N \right\}$. Sei $y$ so dass $y-x<\delta$
  \[(x,t),(y,t)\in U_{x_i,t_i}\subset B_{\delta(x_i,t_i}(x_i,t_i)\]
  \[\implies \Abs{g(y,t)-g(x_i,t_i)}<\frac{\varepsilon}{10}\]
  und
  \[\implies \Abs{g(x,t)-g(x_i,t_i)}<\frac{\varepsilon}{10}\]
  \[\implies \Abs{g(x,t)-g(y,t)}<\frac{\varepsilon}{5}\]
  \[\implies \sup_{y\in B_\delta(x)}\Abs{g(x,t)-g(y-t)}\leq\frac{\varepsilon}{5}<\varepsilon\s\forall t\in [a,b]\]
\end{Bew}
\begin{Kor}
  Sei $g:U\times [a,b]\to\mb{R}$ stetig. Dann
  \[F(x)=\int_a^bg(x,t)\md t\]
  ist eine stetige Funktion
\end{Kor}
\begin{Bew}
  Seien $x\in U$ und $\varepsilon >0$. Das letzte Lemma $\implies$ $\exists \delta >0$ so dass
  \[\Abs{g(x,t)-g(y,t)}\leq \frac{\varepsilon}{b-a}\]
  $\forall t$ und $\forall y$ mit $\Norm{y-x}<\delta$. Deswegen für $\Norm{y-x}<\delta$
  \begin{eqnarray*}
\Abs{F(y)-F(x)}&=&\Abs{\int_a^b(g(x,t)-g(y,t))\md t}
\leq\int_a^b\Abs{g(x,t)-g(y,t)}\md t\\
  &<&\int_a^b\frac{\varepsilon}{b-a}\md t=\varepsilon\, .
\end{eqnarray*}
\end{Bew}
\begin{Bem}
  Im Differentiationssatz ist $\Part{f}{x_i}$ eine stetige Funktion. Da
  \[\Part{f}{x_i}(x)=\int_a^b\Part{f}{x_i}(x,t)\md t\]
  ist $\Part{F}{x_i}$ stetig.
\end{Bem}

  Der folgende Satz ist eine sehr wichtige Konsequenz der Differentiationssatz.